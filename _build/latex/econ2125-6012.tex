%% Generated by Sphinx.
\def\sphinxdocclass{jupyterBook}
\documentclass[letterpaper,10pt,english]{jupyterBook}
\ifdefined\pdfpxdimen
   \let\sphinxpxdimen\pdfpxdimen\else\newdimen\sphinxpxdimen
\fi \sphinxpxdimen=.75bp\relax
\ifdefined\pdfimageresolution
    \pdfimageresolution= \numexpr \dimexpr1in\relax/\sphinxpxdimen\relax
\fi
%% let collapsible pdf bookmarks panel have high depth per default
\PassOptionsToPackage{bookmarksdepth=5}{hyperref}
%% turn off hyperref patch of \index as sphinx.xdy xindy module takes care of
%% suitable \hyperpage mark-up, working around hyperref-xindy incompatibility
\PassOptionsToPackage{hyperindex=false}{hyperref}
%% memoir class requires extra handling
\makeatletter\@ifclassloaded{memoir}
{\ifdefined\memhyperindexfalse\memhyperindexfalse\fi}{}\makeatother

\PassOptionsToPackage{warn}{textcomp}

\catcode`^^^^00a0\active\protected\def^^^^00a0{\leavevmode\nobreak\ }
\usepackage{cmap}
\usepackage{fontspec}
\defaultfontfeatures[\rmfamily,\sffamily,\ttfamily]{}
\usepackage{amsmath,amssymb,amstext}
\usepackage{polyglossia}
\setmainlanguage{english}



\setmainfont{FreeSerif}[
  Extension      = .otf,
  UprightFont    = *,
  ItalicFont     = *Italic,
  BoldFont       = *Bold,
  BoldItalicFont = *BoldItalic
]
\setsansfont{FreeSans}[
  Extension      = .otf,
  UprightFont    = *,
  ItalicFont     = *Oblique,
  BoldFont       = *Bold,
  BoldItalicFont = *BoldOblique,
]
\setmonofont{FreeMono}[
  Extension      = .otf,
  UprightFont    = *,
  ItalicFont     = *Oblique,
  BoldFont       = *Bold,
  BoldItalicFont = *BoldOblique,
]



\usepackage[Bjarne]{fncychap}
\usepackage[,numfigreset=1,mathnumfig]{sphinx}

\fvset{fontsize=\small}
\usepackage{geometry}


% Include hyperref last.
\usepackage{hyperref}
% Fix anchor placement for figures with captions.
\usepackage{hypcap}% it must be loaded after hyperref.
% Set up styles of URL: it should be placed after hyperref.
\urlstyle{same}


\usepackage{sphinxmessages}



        % Start of preamble defined in sphinx-jupyterbook-latex %
         \usepackage[Latin,Greek]{ucharclasses}
        \usepackage{unicode-math}
        % fixing title of the toc
        \addto\captionsenglish{\renewcommand{\contentsname}{Contents}}
        \hypersetup{
            pdfencoding=auto,
            psdextra
        }
        % End of preamble defined in sphinx-jupyterbook-latex %
        

\title{ECON2125/6012}
\date{Aug 03, 2023}
\release{}
\author{Fedor Iskhakov}
\newcommand{\sphinxlogo}{\vbox{}}
\renewcommand{\releasename}{}
\makeindex
\begin{document}

\pagestyle{empty}
\sphinxmaketitle
\pagestyle{plain}
\sphinxtableofcontents
\pagestyle{normal}
\phantomsection\label{\detokenize{00.index::doc}}


\begin{DUlineblock}{0em}
\item[] \sphinxstylestrong{\Large Preliminary schedule}
\end{DUlineblock}


\begin{savenotes}\sphinxattablestart
\centering
\begin{tabulary}{\linewidth}[t]{|T|T|T|T|}
\hline
\sphinxstyletheadfamily 
\sphinxAtStartPar
Week
&\sphinxstyletheadfamily 
\sphinxAtStartPar
Date
&\sphinxstyletheadfamily 
\sphinxAtStartPar
Topic
&\sphinxstyletheadfamily 
\sphinxAtStartPar
Notes
\\
\hline
\sphinxAtStartPar
1
&
\sphinxAtStartPar
July 27
&
\sphinxAtStartPar
{\hyperref[\detokenize{01.introduction::doc}]{\sphinxcrossref{\DUrole{doc,std,std-doc}{Introduction}}}}
&
\sphinxAtStartPar
Recorded lecture
\\
\hline
\sphinxAtStartPar
2
&
\sphinxAtStartPar
Aug 3
&
\sphinxAtStartPar
{\hyperref[\detokenize{02.optimization_intro::doc}]{\sphinxcrossref{\DUrole{doc,std,std-doc}{Univariate and bivariate optimization}}}}
&
\sphinxAtStartPar
Tutorials start
\\
\hline
\sphinxAtStartPar
3
&
\sphinxAtStartPar
Aug 10
&
\sphinxAtStartPar
{\hyperref[\detokenize{03.set_theory::doc}]{\sphinxcrossref{\DUrole{doc,std,std-doc}{Elements of set theory and analysis}}}}
&
\sphinxAtStartPar

\\
\hline
\sphinxAtStartPar
4
&
\sphinxAtStartPar
Aug 17
&
\sphinxAtStartPar
{\hyperref[\detokenize{04.linear_algebra::doc}]{\sphinxcrossref{\DUrole{doc,std,std-doc}{Elements of linear algebra}}}}
&
\sphinxAtStartPar

\\
\hline
\sphinxAtStartPar
Test
&
\sphinxAtStartPar

&
\sphinxAtStartPar
15\%
&
\sphinxAtStartPar
Submit by Aug 23
\\
\hline
\sphinxAtStartPar
5
&
\sphinxAtStartPar
Aug 24
&
\sphinxAtStartPar
{\hyperref[\detokenize{05.probability::doc}]{\sphinxcrossref{\DUrole{doc,std,std-doc}{Elements of Probability}}}}
&
\sphinxAtStartPar

\\
\hline
\sphinxAtStartPar
6
&
\sphinxAtStartPar
Aug 31
&
\sphinxAtStartPar
{\hyperref[\detokenize{06.optimization_fundamentals::doc}]{\sphinxcrossref{\DUrole{doc,std,std-doc}{Fundamentals of optimization}}}}
&
\sphinxAtStartPar

\\
\hline
\sphinxAtStartPar
Test
&
\sphinxAtStartPar

&
\sphinxAtStartPar
15\%
&
\sphinxAtStartPar
Submit by Sept 3
\\
\hline
\sphinxAtStartPar
Break
&
\sphinxAtStartPar

&
\sphinxAtStartPar

&
\sphinxAtStartPar
2 weeks
\\
\hline
\sphinxAtStartPar
7
&
\sphinxAtStartPar
Sept 21
&
\sphinxAtStartPar
{\hyperref[\detokenize{07.unconstrained::doc}]{\sphinxcrossref{\DUrole{doc,std,std-doc}{Unconstrained optimization}}}}
&
\sphinxAtStartPar

\\
\hline
\sphinxAtStartPar
8
&
\sphinxAtStartPar
Sept 28
&
\sphinxAtStartPar
{\hyperref[\detokenize{08.constrained::doc}]{\sphinxcrossref{\DUrole{doc,std,std-doc}{Constrained optimization}}}}
&
\sphinxAtStartPar

\\
\hline
\sphinxAtStartPar
Test
&
\sphinxAtStartPar

&
\sphinxAtStartPar
15\%
&
\sphinxAtStartPar
Submit by Oct 4
\\
\hline
\sphinxAtStartPar
9
&
\sphinxAtStartPar
Oct 5
&
\sphinxAtStartPar
{\hyperref[\detokenize{09.practical_session::doc}]{\sphinxcrossref{\DUrole{doc,std,std-doc}{Practical session/invited speaker}}}}
&
\sphinxAtStartPar
TBA
\\
\hline
\sphinxAtStartPar
10
&
\sphinxAtStartPar
Oct 12
&
\sphinxAtStartPar
{\hyperref[\detokenize{10.envelope_maximum::doc}]{\sphinxcrossref{\DUrole{doc,std,std-doc}{Envelope and maximum theorems}}}}
&
\sphinxAtStartPar

\\
\hline
\sphinxAtStartPar
11
&
\sphinxAtStartPar
Oct 19
&
\sphinxAtStartPar
{\hyperref[\detokenize{11.dynamic::doc}]{\sphinxcrossref{\DUrole{doc,std,std-doc}{Dynamic optimization}}}}
&
\sphinxAtStartPar

\\
\hline
\sphinxAtStartPar
12
&
\sphinxAtStartPar
Oct 26
&
\sphinxAtStartPar
{\hyperref[\detokenize{12.revision::doc}]{\sphinxcrossref{\DUrole{doc,std,std-doc}{Revision}}}}
&
\sphinxAtStartPar

\\
\hline
\sphinxAtStartPar
Exam
&
\sphinxAtStartPar

&
\sphinxAtStartPar
55\%
&
\sphinxAtStartPar
During exam period
\\
\hline
\end{tabulary}
\par
\sphinxattableend\end{savenotes}

\begin{DUlineblock}{0em}
\item[] \sphinxstylestrong{\large ANU course pages}
\end{DUlineblock}

\sphinxAtStartPar
\sphinxhref{https://wattlecourses.anu.edu.au/course/view.php?id=41102}{Course Wattle page}
Schedule, announcements, teaching team contacts, recordings, assignement, grades

\sphinxAtStartPar
\sphinxhref{https://programsandcourses.anu.edu.au/2023/course/ECON2125\#terms}{Course overview}
\sphinxhref{https://programsandcourses.anu.edu.au/course/ECON2125/Second\%20Semester/6275}{Class summary}
General course description in ANU Programs and Courses



\sphinxstepscope


\chapter{Welcome}
\label{\detokenize{01.introduction:welcome}}\label{\detokenize{01.introduction::doc}}
\sphinxAtStartPar
Course title: \sphinxstylestrong{“Optimization for Economics and Financial Economics”}
\begin{itemize}
\item {} 
\sphinxAtStartPar
Elective second year course in the \sphinxstyleemphasis{Bachelor of Economics} program ECON2125

\item {} 
\sphinxAtStartPar
Compulsory second math course in the \sphinxstyleemphasis{Master of Economics} program ECON6012

\end{itemize}

\sphinxAtStartPar
The two courses are identical in content and assessment, but final grades may be adjusted depending on your program.


\section{Plan for this lecture}
\label{\detokenize{01.introduction:plan-for-this-lecture}}\begin{enumerate}
\sphinxsetlistlabels{\arabic}{enumi}{enumii}{}{.}%
\item {} 
\sphinxAtStartPar
Organization

\item {} 
\sphinxAtStartPar
Administrative topics

\item {} 
\sphinxAtStartPar
Course content

\item {} 
\sphinxAtStartPar
Self\sphinxhyphen{}learning materials

\end{enumerate}


\section{Instructor}
\label{\detokenize{01.introduction:instructor}}
\sphinxAtStartPar
\sphinxstylestrong{Fedor Iskhakov}
Professor of Economics at RSE
\begin{itemize}
\item {} 
\sphinxAtStartPar
Office: 1021 HW Arndt Building

\item {} 
\sphinxAtStartPar
Email: \sphinxhref{mailto:fedor.iskhakov@anu.edu.au}{fedor.iskhakov@anu.edu.au}

\item {} 
\sphinxAtStartPar
Web: \sphinxhref{https://fedor.iskh.me}{fedor.iskh.me}

\item {} 
\sphinxAtStartPar
Contact hours: Thursday 9:30\sphinxhyphen{}11:30

\end{itemize}


\section{Timetable}
\label{\detokenize{01.introduction:timetable}}
\sphinxAtStartPar
\sphinxstylestrong{Face\sphinxhyphen{}to\sphinxhyphen{}face:}
\begin{itemize}
\item {} 
\sphinxAtStartPar
Lectures: Thursday 15:30 — 17:30

\item {} 
\sphinxAtStartPar
Location: \sphinxstylestrong{DNF Dunbar Lecture Theatre, Physics Bldg 39A}

\end{itemize}

\sphinxAtStartPar
\sphinxstylestrong{Online:}
\begin{itemize}
\item {} 
\sphinxAtStartPar
Echo\sphinxhyphen{}360 recordings on Wattle

\item {} 
\sphinxAtStartPar
All notes and materials on \sphinxstylestrong{\sphinxhref{http://optim.iskh.me}{optim.iskh.me}}

\end{itemize}

\sphinxAtStartPar
Face\sphinxhyphen{}to\sphinxhyphen{}face is strictly preferred


\section{Course web pages}
\label{\detokenize{01.introduction:course-web-pages}}\begin{itemize}
\item {} 
\sphinxAtStartPar
\sphinxhref{https://wattlecourses.anu.edu.au/course/view.php?id=41102}{Wattle}
Schedule, announcements, teaching team contacts, recordings, assignment, grades

\item {} 
\sphinxAtStartPar
\sphinxhref{https://optim.iskh.me}{Online notes}
Lecture notes, slides, assignment tasks

\item {} 
\sphinxAtStartPar
Lecture slides should appear online the previous day before the lecture

\item {} 
\sphinxAtStartPar
Details on assessment including the exam instructions will appear on Wattle

\end{itemize}


\section{Tutorials}
\label{\detokenize{01.introduction:tutorials}}\begin{itemize}
\item {} 
\sphinxAtStartPar
Enrollments open on \sphinxstyleemphasis{Wattle}

\end{itemize}

\sphinxAtStartPar
Tutorial questions
\begin{itemize}
\item {} 
\sphinxAtStartPar
posted on the course website

\item {} 
\sphinxAtStartPar
not assessed, help you learn and prepare

\end{itemize}

\sphinxAtStartPar
Tutorials start on week 2


\section{Tutors}
\label{\detokenize{01.introduction:tutors}}
\sphinxAtStartPar
\sphinxstylestrong{Wending Liu}
\begin{itemize}
\item {} 
\sphinxAtStartPar
Email: \sphinxhref{mailto:Wending.Liu@anu.edu.au}{Wending.Liu@anu.edu.au}

\item {} 
\sphinxAtStartPar
Room: Room 2084, Copland Bld (24)

\item {} 
\sphinxAtStartPar
Office hours: \sphinxstylestrong{Friday 1pm\sphinxhyphen{}3pm}

\end{itemize}

\sphinxAtStartPar
\sphinxstylestrong{Chien Yeh}
\begin{itemize}
\item {} 
\sphinxAtStartPar
Email: \sphinxhref{mailto:Chien.Yeh@anu.edu.au}{Chien.Yeh@anu.edu.au}

\item {} 
\sphinxAtStartPar
Room: Room 2106, Copland Bld (24)

\item {} 
\sphinxAtStartPar
Office hours: \sphinxstylestrong{Monday 2pm\sphinxhyphen{}4pm}

\end{itemize}


\section{Prerequisites}
\label{\detokenize{01.introduction:prerequisites}}
\sphinxAtStartPar
See \sphinxhref{https://programsandcourses.anu.edu.au/2023/course/ECON2125\#terms}{Course overview} and
\sphinxhref{https://programsandcourses.anu.edu.au/course/ECON2125/Second\%20Semester/6275}{Class summary}

\sphinxAtStartPar
What you actually need to know:
\begin{itemize}
\item {} 
\sphinxAtStartPar
basic algebra

\item {} 
\sphinxAtStartPar
basic calculus

\item {} 
\sphinxAtStartPar
some idea of what a matrix is, etc.

\end{itemize}

\sphinxAtStartPar
≈ content of EMET1001/EMET7001 math course


\section{Focus?}
\label{\detokenize{01.introduction:focus}}
\sphinxAtStartPar
\sphinxstyleemphasis{Q:} Is this optimization or a general math\sphinxhyphen{}econ course?

\sphinxAtStartPar
\sphinxstyleemphasis{A:} A general course on mathematical modeling for economics and financial economics. Optimization will be an important and recurring theme.


\section{Assessment}
\label{\detokenize{01.introduction:assessment}}\begin{itemize}
\item {} 
\sphinxAtStartPar
3 timed open book tests (15\% each)

\item {} 
\sphinxAtStartPar
Final exam (55\%)

\end{itemize}

\sphinxAtStartPar
The three tests spread out through the semester will check the knowledge of the immediately preceding material. The final closed book in\sphinxhyphen{}person exam will cover the entire course.


\section{Questions}
\label{\detokenize{01.introduction:questions}}\begin{enumerate}
\sphinxsetlistlabels{\arabic}{enumi}{enumii}{}{.}%
\item {} 
\sphinxAtStartPar
Administrative questions: RSE admin

\end{enumerate}
\begin{itemize}
\item {} 
\sphinxAtStartPar
\sphinxstylestrong{Bronwyn Cammack} Senior School Administrator

\item {} 
\sphinxAtStartPar
Email: \sphinxhref{mailto:enquiries.rse@anu.edu.au}{enquiries.rse@anu.edu.au}

\item {} 
\sphinxAtStartPar
“I can not register for the tutorial group”

\end{itemize}
\begin{enumerate}
\sphinxsetlistlabels{\arabic}{enumi}{enumii}{}{.}%
\setcounter{enumi}{1}
\item {} 
\sphinxAtStartPar
Content related questions: please, refer to the tutors

\end{enumerate}
\begin{itemize}
\item {} 
\sphinxAtStartPar
“I don’t understand why this function is convex”

\end{itemize}
\begin{enumerate}
\sphinxsetlistlabels{\arabic}{enumi}{enumii}{}{.}%
\setcounter{enumi}{2}
\item {} 
\sphinxAtStartPar
Other questions: to Fedor

\end{enumerate}
\begin{itemize}
\item {} 
\sphinxAtStartPar
“I’m working hard but still can not keep up”

\item {} 
\sphinxAtStartPar
“Can I please have extra assignment for more practice”

\end{itemize}


\section{Attendance}
\label{\detokenize{01.introduction:attendance}}\begin{itemize}
\item {} 
\sphinxAtStartPar
Please, \sphinxstylestrong{do not} use email for \sphinxstyleemphasis{instructional} questions\textbackslash{}Instead make use of the office hours

\item {} 
\sphinxAtStartPar
Attendance of tutorials is \sphinxstyleemphasis{very highly} recommended\\
You will make your life much easier this way

\item {} 
\sphinxAtStartPar
Attendance of lectures is \sphinxstyleemphasis{highly} recommended\\
But not mandatory

\end{itemize}


\section{Comments for lectures notes/slides}
\label{\detokenize{01.introduction:comments-for-lectures-notes-slides}}\begin{itemize}
\item {} 
\sphinxAtStartPar
Cover exactly what you are required to know

\item {} 
\sphinxAtStartPar
Code inserts are the exception, they are not assessable

\end{itemize}

\sphinxAtStartPar
In particular, you need to know:
\begin{itemize}
\item {} 
\sphinxAtStartPar
The definitions from the notes

\item {} 
\sphinxAtStartPar
The facts from the notes

\item {} 
\sphinxAtStartPar
How to apply facts and definitions

\end{itemize}

\sphinxAtStartPar
If a concept in not in the lecture notes, it is not assessable


\section{Definitions and facts}
\label{\detokenize{01.introduction:definitions-and-facts}}
\sphinxAtStartPar
The lectures notes/slides are full of definitions and facts.

\begin{sphinxadmonition}{note}{Definition}

\sphinxAtStartPar
Functions \(f: \mathbb{R} \rightarrow \mathbb{R}\) is called \sphinxstyleemphasis{continuous at} \(x\) if, for any sequence \(\{x_n\}\) converging to \(x\), we have \(f(x_n) \rightarrow f(x)\).
\end{sphinxadmonition}

\sphinxAtStartPar
Possible exam question: “Show  that if functions \(f\) and \(g\) are continuous at \(x\), so is \(f+g\).”

\sphinxAtStartPar
You should start the answer with the definition of continuity:

\sphinxAtStartPar
“Let \(\{x_n\}\) be any sequence converging to \(x\). We need to show that \(f(x_n) + g(x_n) \rightarrow f(x) + g(x)\). To see this, note that …”


\section{Facts}
\label{\detokenize{01.introduction:facts}}
\sphinxAtStartPar
In the lecture notes/slides you will often see

\begin{sphinxadmonition}{note}{Fact}

\sphinxAtStartPar
The only \(N\)\sphinxhyphen{}dimensional subset of \(\mathbb{R}^N\) is \(\mathbb{R}^N\).
\end{sphinxadmonition}

\sphinxAtStartPar
This means either:
\begin{itemize}
\item {} 
\sphinxAtStartPar
theorem

\item {} 
\sphinxAtStartPar
proposition

\item {} 
\sphinxAtStartPar
lemma

\item {} 
\sphinxAtStartPar
true statement

\end{itemize}

\sphinxAtStartPar
All well known results. You need to remember them, have some intuition for, and be able to apply.


\section{Note on Assessments}
\label{\detokenize{01.introduction:note-on-assessments}}
\sphinxAtStartPar
Assessable = definitions and facts + last year level math + a few simple steps of logic

\sphinxAtStartPar
Exams and tests will award:
\begin{itemize}
\item {} 
\sphinxAtStartPar
Hard work

\item {} 
\sphinxAtStartPar
Deeper understanding of the concepts

\end{itemize}

\sphinxAtStartPar
In each question there will be a \sphinxstyleemphasis{easy} path to the solution


\section{Reading materials}
\label{\detokenize{01.introduction:reading-materials}}
\sphinxAtStartPar
\sphinxstylestrong{Primary reference:} lecture slides

\sphinxAtStartPar
\sphinxstylestrong{Books:}

\noindent\sphinxincludegraphics[height=100\sphinxpxdimen]{{simon_blume}.png}

\noindent\sphinxincludegraphics[height=100\sphinxpxdimen]{{sundaram}.png}

\noindent\sphinxincludegraphics[height=100\sphinxpxdimen]{{stachurski}.png}
\begin{itemize}
\item {} 
\sphinxAtStartPar
“Mathematics for Economists” (1994) by Simon, C. and L. Blume

\item {} 
\sphinxAtStartPar
“A First Course in Optimization” (1996) Theory by Rangarajan Sundaram

\item {} 
\sphinxAtStartPar
“A Primer in Econometric Theory” (2016) by John Stachurski

\end{itemize}

\sphinxAtStartPar
Readings are supplementary but will provide a more detailed explanation with additional examples.
\begin{itemize}
\item {} 
\sphinxAtStartPar
Each lecture will reference book chapters

\end{itemize}


\section{Key points for the administrative part}
\label{\detokenize{01.introduction:key-points-for-the-administrative-part}}\begin{itemize}
\item {} 
\sphinxAtStartPar
Tutorials start next week, \sphinxstylestrong{please register before the next lecture}

\item {} 
\sphinxAtStartPar
Course content = what’s in lecture notes/slides

\item {} 
\sphinxAtStartPar
Lecture slides are available online and will be updated throughout the semester

\item {} 
\sphinxAtStartPar
Optimization is a recurring theme but not the only topic

\end{itemize}


\section{What you will learn in the course}
\label{\detokenize{01.introduction:what-you-will-learn-in-the-course}}\begin{itemize}
\item {} 
\sphinxAtStartPar
The lecture plan is on the course website \sphinxhref{https://optim.iskh.me}{optim.iskh.me} and \sphinxhref{https://programsandcourses.anu.edu.au/course/ECON2125/Second\%20Semester/6275}{Class summary}

\item {} 
\sphinxAtStartPar
See the list of topics on the left

\end{itemize}

\sphinxAtStartPar
Essentially:
\begin{enumerate}
\sphinxsetlistlabels{\arabic}{enumi}{enumii}{}{.}%
\item {} 
\sphinxAtStartPar
\sphinxstylestrong{Mathematical foundations}

\end{enumerate}
\begin{itemize}
\item {} 
\sphinxAtStartPar
elements of analysis

\item {} 
\sphinxAtStartPar
elements of linear algebra

\item {} 
\sphinxAtStartPar
elements of probability

\end{itemize}
\begin{enumerate}
\sphinxsetlistlabels{\arabic}{enumi}{enumii}{}{.}%
\setcounter{enumi}{1}
\item {} 
\sphinxAtStartPar
\sphinxstylestrong{Optimization theory}

\end{enumerate}
\begin{itemize}
\item {} 
\sphinxAtStartPar
when solution exists

\item {} 
\sphinxAtStartPar
unconstrained optimization

\item {} 
\sphinxAtStartPar
optimization with equality constraints

\item {} 
\sphinxAtStartPar
optimization with inequality constraints

\end{itemize}
\begin{enumerate}
\sphinxsetlistlabels{\arabic}{enumi}{enumii}{}{.}%
\setcounter{enumi}{2}
\item {} 
\sphinxAtStartPar
\sphinxstylestrong{Further topics}

\end{enumerate}
\begin{itemize}
\item {} 
\sphinxAtStartPar
Parameterized optimization problems

\item {} 
\sphinxAtStartPar
Optimization in dynamics

\end{itemize}


\section{Further material and self\sphinxhyphen{}learning}
\label{\detokenize{01.introduction:further-material-and-self-learning}}\begin{itemize}
\item {} 
\sphinxAtStartPar
Each lecture will suggest some material for further reading and learning

\item {} 
\sphinxAtStartPar
Today: \sphinxstylestrong{The Wason Selection Task} logical problem

\item {} 
\sphinxAtStartPar
Mathematics relies on rules of logic

\item {} 
\sphinxAtStartPar
Yet, for human brain applying mathematical logic may be difficult, and dependent on the domain

\end{itemize}

\sphinxAtStartPar
Please, watch the video and try to solve the puzzle yourself
\sphinxhref{https://youtu.be/iR97LBgpsl8}{youtu.be/iR97LBgpsl8}

\sphinxstepscope


\chapter{Univariate and bivariate optimization}
\label{\detokenize{02.optimization_intro:univariate-and-bivariate-optimization}}\label{\detokenize{02.optimization_intro::doc}}
\sphinxAtStartPar
\sphinxstylestrong{ECON2125/6012 Lecture 2}
Fedor Iskhakov


\section{Announcements \& Reminders}
\label{\detokenize{02.optimization_intro:announcements-reminders}}\begin{itemize}
\item {} 
\sphinxAtStartPar
\sphinxstylestrong{Tutorials start tomorrow (Aug 4)}

\item {} 
\sphinxAtStartPar
Register for tutorials on \sphinxhref{https://wattlecourses.anu.edu.au/course/view.php?id=41102}{Wattle} if you have not done so already

\item {} 
\sphinxAtStartPar
Office hours of the tutors are updated:
\begin{itemize}
\item {} 
\sphinxAtStartPar
\sphinxstylestrong{Wending Liu}
\begin{itemize}
\item {} 
\sphinxAtStartPar
Email: \sphinxhref{mailto:Wending.Liu@anu.edu.au}{Wending.Liu@anu.edu.au}

\item {} 
\sphinxAtStartPar
Room: Room 2084, Copland Bld (24) (\sphinxstyleemphasis{updated!})

\item {} 
\sphinxAtStartPar
Office hours: \sphinxstylestrong{Friday 1pm\sphinxhyphen{}3pm}

\end{itemize}

\item {} 
\sphinxAtStartPar
\sphinxstylestrong{Chien Yeh}
\begin{itemize}
\item {} 
\sphinxAtStartPar
Email: \sphinxhref{mailto:Chien.Yeh@anu.edu.au}{Chien.Yeh@anu.edu.au}

\item {} 
\sphinxAtStartPar
Room: Room 2106, Copland Bld (24)

\item {} 
\sphinxAtStartPar
Office hours: \sphinxstylestrong{Monday 2pm\sphinxhyphen{}4pm}

\end{itemize}

\end{itemize}

\item {} 
\sphinxAtStartPar
Reminder on how to ask questions:
\begin{enumerate}
\sphinxsetlistlabels{\arabic}{enumi}{enumii}{}{.}%
\item {} 
\sphinxAtStartPar
Administrative: RSE admin

\item {} 
\sphinxAtStartPar
Content/understanding: tutors

\item {} 
\sphinxAtStartPar
Other: to Fedor

\end{enumerate}

\end{itemize}


\section{Plan for this lecture}
\label{\detokenize{02.optimization_intro:plan-for-this-lecture}}\begin{enumerate}
\sphinxsetlistlabels{\arabic}{enumi}{enumii}{}{.}%
\item {} 
\sphinxAtStartPar
Motivation (math vs. computing)

\item {} 
\sphinxAtStartPar
Univariate optimization

\item {} 
\sphinxAtStartPar
Working with bivariate functions

\item {} 
\sphinxAtStartPar
Bivariate optimization

\end{enumerate}

\sphinxAtStartPar
\sphinxstylestrong{Supplementary reading:}
\begin{itemize}
\item {} 
\sphinxAtStartPar
Simon \& Blume: part 1 (revision)

\item {} 
\sphinxAtStartPar
Sundaram: sections 1.1, 1.4, chapter 2, chapter 4

\end{itemize}


\section{Computing}
\label{\detokenize{02.optimization_intro:computing}}
\sphinxAtStartPar
The \sphinxstyleemphasis{classic} way we do mathematics is pencil and paper
\begin{quote}

\sphinxAtStartPar
In 1944, Hans Bethe solved following problem \sphinxstyleemphasis{by hand}:\\
Will detonating an atom bomb ignite the atmosphere and
thereby destroy life on earth?\\
\sphinxhref{https://inis.iaea.org/search/search.aspx?orig\_q=RN:25070731}{source}
\end{quote}

\sphinxAtStartPar
These days we rarely calculate with actual numbers

\sphinxAtStartPar
Almost all calculations are done on computers

\begin{sphinxadmonition}{note}{Example: numerical integration}
\begin{equation*}
\begin{split}
\frac{1}{\sqrt{2\pi}} 
\int_{-2}^2 
\exp\left\{ - \frac{x^2}{2} \right\} dx
\end{split}
\end{equation*}\end{sphinxadmonition}

\begin{sphinxuseclass}{cell}\begin{sphinxVerbatimInput}

\begin{sphinxuseclass}{cell_input}
\begin{sphinxVerbatim}[commandchars=\\\{\}]
\PYG{k+kn}{from} \PYG{n+nn}{scipy}\PYG{n+nn}{.}\PYG{n+nn}{stats} \PYG{k+kn}{import} \PYG{n}{norm}
\PYG{k+kn}{from} \PYG{n+nn}{scipy}\PYG{n+nn}{.}\PYG{n+nn}{integrate} \PYG{k+kn}{import} \PYG{n}{quad}
\PYG{n}{phi} \PYG{o}{=} \PYG{n}{norm}\PYG{p}{(}\PYG{p}{)}
\PYG{n}{value}\PYG{p}{,} \PYG{n}{error} \PYG{o}{=} \PYG{n}{quad}\PYG{p}{(}\PYG{n}{phi}\PYG{o}{.}\PYG{n}{pdf}\PYG{p}{,} \PYG{o}{\PYGZhy{}}\PYG{l+m+mi}{2}\PYG{p}{,} \PYG{l+m+mi}{2}\PYG{p}{)}
\PYG{n+nb}{print}\PYG{p}{(}\PYG{l+s+s1}{\PYGZsq{}}\PYG{l+s+s1}{Integral value =}\PYG{l+s+s1}{\PYGZsq{}}\PYG{p}{,}\PYG{n}{value}\PYG{p}{)}
\end{sphinxVerbatim}

\end{sphinxuseclass}\end{sphinxVerbatimInput}
\begin{sphinxVerbatimOutput}

\begin{sphinxuseclass}{cell_output}
\begin{sphinxVerbatim}[commandchars=\\\{\}]
Integral value = 0.9544997361036417
\end{sphinxVerbatim}

\end{sphinxuseclass}\end{sphinxVerbatimOutput}

\end{sphinxuseclass}
\begin{sphinxadmonition}{note}{Example: Numerical optimization}
\begin{equation*}
\begin{split}
f(x) = - \exp
\left\{-\frac{(x - 5.0)^4}{1.5} \right\}
\rightarrow \min
\end{split}
\end{equation*}\end{sphinxadmonition}

\begin{sphinxuseclass}{cell}\begin{sphinxVerbatimInput}

\begin{sphinxuseclass}{cell_input}
\begin{sphinxVerbatim}[commandchars=\\\{\}]
\PYG{k+kn}{from} \PYG{n+nn}{scipy}\PYG{n+nn}{.}\PYG{n+nn}{optimize} \PYG{k+kn}{import} \PYG{n}{fminbound}
\PYG{k+kn}{import} \PYG{n+nn}{numpy} \PYG{k}{as} \PYG{n+nn}{np}
\PYG{n}{f} \PYG{o}{=} \PYG{k}{lambda} \PYG{n}{x}\PYG{p}{:} \PYG{o}{\PYGZhy{}}\PYG{n}{np}\PYG{o}{.}\PYG{n}{exp}\PYG{p}{(}\PYG{o}{\PYGZhy{}}\PYG{p}{(}\PYG{n}{x} \PYG{o}{\PYGZhy{}} \PYG{l+m+mf}{5.0}\PYG{p}{)}\PYG{o}{*}\PYG{o}{*}\PYG{l+m+mi}{4} \PYG{o}{/} \PYG{l+m+mf}{1.5}\PYG{p}{)}
\PYG{n}{res} \PYG{o}{=} \PYG{n}{fminbound}\PYG{p}{(}\PYG{n}{f}\PYG{p}{,} \PYG{o}{\PYGZhy{}}\PYG{l+m+mi}{10}\PYG{p}{,} \PYG{l+m+mi}{10}\PYG{p}{)}  \PYG{c+c1}{\PYGZsh{} find approx solution}
\PYG{n+nb}{print}\PYG{p}{(}\PYG{l+s+s1}{\PYGZsq{}}\PYG{l+s+s1}{Minimum value is attained approximately at}\PYG{l+s+s1}{\PYGZsq{}}\PYG{p}{,} \PYG{n}{res}\PYG{p}{)}
\end{sphinxVerbatim}

\end{sphinxuseclass}\end{sphinxVerbatimInput}
\begin{sphinxVerbatimOutput}

\begin{sphinxuseclass}{cell_output}
\begin{sphinxVerbatim}[commandchars=\\\{\}]
Minimum value is attained approximately at 4.999941901210501
\end{sphinxVerbatim}

\end{sphinxuseclass}\end{sphinxVerbatimOutput}

\end{sphinxuseclass}
\begin{sphinxadmonition}{note}{Example: Visualization}

\sphinxAtStartPar
What does this function look like?
\begin{equation*}
\begin{split}
f(x, y) = \frac{\cos(x^2 + y^2)}{1 + x^2 + y^2}
\end{split}
\end{equation*}\end{sphinxadmonition}

\begin{sphinxuseclass}{cell}\begin{sphinxVerbatimInput}

\begin{sphinxuseclass}{cell_input}
\begin{sphinxVerbatim}[commandchars=\\\{\}]
\PYG{k+kn}{import} \PYG{n+nn}{matplotlib}\PYG{n+nn}{.}\PYG{n+nn}{pyplot} \PYG{k}{as} \PYG{n+nn}{plt}
\PYG{k+kn}{from} \PYG{n+nn}{mpl\PYGZus{}toolkits}\PYG{n+nn}{.}\PYG{n+nn}{mplot3d}\PYG{n+nn}{.}\PYG{n+nn}{axes3d} \PYG{k+kn}{import} \PYG{n}{Axes3D}
\PYG{k+kn}{import} \PYG{n+nn}{numpy} \PYG{k}{as} \PYG{n+nn}{np}
\PYG{k+kn}{from} \PYG{n+nn}{matplotlib} \PYG{k+kn}{import} \PYG{n}{cm}
\PYG{n}{f} \PYG{o}{=} \PYG{k}{lambda} \PYG{n}{x}\PYG{p}{,} \PYG{n}{y}\PYG{p}{:} \PYG{n}{np}\PYG{o}{.}\PYG{n}{cos}\PYG{p}{(}\PYG{n}{x}\PYG{o}{*}\PYG{o}{*}\PYG{l+m+mi}{2} \PYG{o}{+} \PYG{n}{y}\PYG{o}{*}\PYG{o}{*}\PYG{l+m+mi}{2}\PYG{p}{)} \PYG{o}{/} \PYG{p}{(}\PYG{l+m+mi}{1} \PYG{o}{+} \PYG{n}{x}\PYG{o}{*}\PYG{o}{*}\PYG{l+m+mi}{2} \PYG{o}{+} \PYG{n}{y}\PYG{o}{*}\PYG{o}{*}\PYG{l+m+mi}{2}\PYG{p}{)}
\PYG{n}{xgrid} \PYG{o}{=} \PYG{n}{np}\PYG{o}{.}\PYG{n}{linspace}\PYG{p}{(}\PYG{o}{\PYGZhy{}}\PYG{l+m+mi}{3}\PYG{p}{,} \PYG{l+m+mi}{3}\PYG{p}{,} \PYG{l+m+mi}{50}\PYG{p}{)}
\PYG{n}{ygrid} \PYG{o}{=} \PYG{n}{xgrid}
\PYG{n}{x}\PYG{p}{,} \PYG{n}{y} \PYG{o}{=} \PYG{n}{np}\PYG{o}{.}\PYG{n}{meshgrid}\PYG{p}{(}\PYG{n}{xgrid}\PYG{p}{,} \PYG{n}{ygrid}\PYG{p}{)}
\PYG{n}{fig} \PYG{o}{=} \PYG{n}{plt}\PYG{o}{.}\PYG{n}{figure}\PYG{p}{(}\PYG{n}{figsize}\PYG{o}{=}\PYG{p}{(}\PYG{l+m+mi}{8}\PYG{p}{,} \PYG{l+m+mi}{6}\PYG{p}{)}\PYG{p}{)}
\PYG{n}{ax} \PYG{o}{=} \PYG{n}{fig}\PYG{o}{.}\PYG{n}{add\PYGZus{}subplot}\PYG{p}{(}\PYG{l+m+mi}{111}\PYG{p}{,} \PYG{n}{projection}\PYG{o}{=}\PYG{l+s+s1}{\PYGZsq{}}\PYG{l+s+s1}{3d}\PYG{l+s+s1}{\PYGZsq{}}\PYG{p}{)}
\PYG{n}{ax}\PYG{o}{.}\PYG{n}{plot\PYGZus{}surface}\PYG{p}{(}\PYG{n}{x}\PYG{p}{,}
                \PYG{n}{y}\PYG{p}{,}
                \PYG{n}{f}\PYG{p}{(}\PYG{n}{x}\PYG{p}{,} \PYG{n}{y}\PYG{p}{)}\PYG{p}{,}
                \PYG{n}{rstride}\PYG{o}{=}\PYG{l+m+mi}{2}\PYG{p}{,} \PYG{n}{cstride}\PYG{o}{=}\PYG{l+m+mi}{2}\PYG{p}{,}
                \PYG{n}{cmap}\PYG{o}{=}\PYG{n}{cm}\PYG{o}{.}\PYG{n}{jet}\PYG{p}{,}
                \PYG{n}{alpha}\PYG{o}{=}\PYG{l+m+mf}{0.7}\PYG{p}{,}
                \PYG{n}{linewidth}\PYG{o}{=}\PYG{l+m+mf}{0.25}\PYG{p}{)}
\PYG{n}{ax}\PYG{o}{.}\PYG{n}{set\PYGZus{}zlim}\PYG{p}{(}\PYG{o}{\PYGZhy{}}\PYG{l+m+mf}{0.5}\PYG{p}{,} \PYG{l+m+mf}{1.0}\PYG{p}{)}
\PYG{n}{plt}\PYG{o}{.}\PYG{n}{show}\PYG{p}{(}\PYG{p}{)}
\end{sphinxVerbatim}

\end{sphinxuseclass}\end{sphinxVerbatimInput}
\begin{sphinxVerbatimOutput}

\begin{sphinxuseclass}{cell_output}
\noindent{\hspace*{\fill}\sphinxincludegraphics[width=0.800\linewidth]{{7e667ededfed90e7cfdd5e04185b793e6f2f689cfecb6d4850b534d03f325af8}.png}\hspace*{\fill}}

\end{sphinxuseclass}\end{sphinxVerbatimOutput}

\end{sphinxuseclass}
\begin{sphinxadmonition}{note}{Example: Symbolic calculations }

\sphinxAtStartPar
Differentiate \(f(x) = (1 + 2x)^5\).\\
Forgotten how?  No problems, just ask a computer for \sphinxstyleemphasis{symbolic} derivative
\end{sphinxadmonition}

\begin{sphinxuseclass}{cell}\begin{sphinxVerbatimInput}

\begin{sphinxuseclass}{cell_input}
\begin{sphinxVerbatim}[commandchars=\\\{\}]
\PYG{k+kn}{import} \PYG{n+nn}{sympy} \PYG{k}{as} \PYG{n+nn}{sp}
\PYG{n}{x} \PYG{o}{=} \PYG{n}{sp}\PYG{o}{.}\PYG{n}{Symbol}\PYG{p}{(}\PYG{l+s+s1}{\PYGZsq{}}\PYG{l+s+s1}{x}\PYG{l+s+s1}{\PYGZsq{}}\PYG{p}{)}
\PYG{n}{fx} \PYG{o}{=} \PYG{p}{(}\PYG{l+m+mi}{1} \PYG{o}{+} \PYG{l+m+mi}{2} \PYG{o}{*} \PYG{n}{x}\PYG{p}{)}\PYG{o}{*}\PYG{o}{*}\PYG{l+m+mi}{5}
\PYG{n+nb}{print}\PYG{p}{(}\PYG{l+s+s2}{\PYGZdq{}}\PYG{l+s+s2}{Derivative of}\PYG{l+s+s2}{\PYGZdq{}}\PYG{p}{,}\PYG{n}{fx}\PYG{p}{,}\PYG{l+s+s2}{\PYGZdq{}}\PYG{l+s+s2}{is}\PYG{l+s+s2}{\PYGZdq{}}\PYG{p}{,}\PYG{n}{fx}\PYG{o}{.}\PYG{n}{diff}\PYG{p}{(}\PYG{n}{x}\PYG{p}{)}\PYG{p}{)}
\end{sphinxVerbatim}

\end{sphinxuseclass}\end{sphinxVerbatimInput}
\begin{sphinxVerbatimOutput}

\begin{sphinxuseclass}{cell_output}
\begin{sphinxVerbatim}[commandchars=\\\{\}]
Derivative of (2*x + 1)**5 is 10*(2*x + 1)**4
\end{sphinxVerbatim}

\end{sphinxuseclass}\end{sphinxVerbatimOutput}

\end{sphinxuseclass}
\sphinxAtStartPar
So if computers can do our maths for us, why learn maths?

\sphinxAtStartPar
The difficulty is
\begin{itemize}
\item {} 
\sphinxAtStartPar
giving them the right inputs and instructions

\item {} 
\sphinxAtStartPar
interpreting what comes out

\end{itemize}

\sphinxAtStartPar
The skills we need are
\begin{itemize}
\item {} 
\sphinxAtStartPar
Understanding of fundamental concepts

\item {} 
\sphinxAtStartPar
Sound deductive reasoning

\end{itemize}

\sphinxAtStartPar
\sphinxstylestrong{These are the focus of the course}


\subsection{Computer Code in the Lectures}
\label{\detokenize{02.optimization_intro:computer-code-in-the-lectures}}
\sphinxAtStartPar
While computation is not a formal part of the course\\
there will be little bits of code in the lectures to illustrate the kinds of things we can do.
\begin{itemize}
\item {} 
\sphinxAtStartPar
All the code will be written in the Python programming language

\item {} 
\sphinxAtStartPar
It is not assessable

\end{itemize}

\sphinxAtStartPar
You might find value in actually running the code shown in lectures\\
If you want to do so please refer to \sphinxstylestrong{linked GitHub repository} in \sphinxhref{https://optim.iskh.me}{optim.iskh.me}


\section{Univariate Optimization}
\label{\detokenize{02.optimization_intro:univariate-optimization}}
\sphinxAtStartPar
Let \(f \colon [a, b] \to \mathbb{R}\) be a differentiable (smooth) function
\begin{itemize}
\item {} 
\sphinxAtStartPar
\([a, b]\) is all \(x\) with \(a \leq x \leq b\)

\item {} 
\sphinxAtStartPar
\(\mathbb{R}\) is “all numbers”

\item {} 
\sphinxAtStartPar
\(f\) takes \(x \in [a, b]\) and returns number \(f(x)\)

\item {} 
\sphinxAtStartPar
derivative \(f'(x)\) exists for all \(x\) with \(a < x < b\)

\end{itemize}

\begin{sphinxadmonition}{note}{Definition}

\sphinxAtStartPar
A  point \(x^* \in [a, b]\) is called a
\begin{itemize}
\item {} 
\sphinxAtStartPar
\sphinxstyleemphasis{\sphinxstylestrong{maximizer}} of \(f\) on \([a, b]\) if \(f(x^*) \geq f(x)\) for all \(x \in [a,b]\)

\item {} 
\sphinxAtStartPar
\sphinxstyleemphasis{\sphinxstylestrong{minimizer}} of \(f\) on \([a, b]\) if \(f(x^*) \leq f(x)\) for all \(x \in [a,b]\)

\end{itemize}
\end{sphinxadmonition}

\begin{sphinxadmonition}{note}{Example}

\sphinxAtStartPar
Let
\begin{itemize}
\item {} 
\sphinxAtStartPar
\(f(x) = -(x-4)^2 + 10\)

\item {} 
\sphinxAtStartPar
\(a = 2\) and \(b=8\)

\end{itemize}

\sphinxAtStartPar
Then
\begin{itemize}
\item {} 
\sphinxAtStartPar
\(x^* = 4\) is a maximizer of \(f\) on \([2, 8]\)

\item {} 
\sphinxAtStartPar
\(x^{**} = 8\) is a minimizer of \(f\) on \([2, 8]\)

\end{itemize}

\begin{figure}[H]
\centering
\capstart

\noindent\sphinxincludegraphics[width=0.800\linewidth]{{d3801ce557e3b38cc0132fb26c560870763137bf39b611b69e7302075a6fce3d}.png}
\caption{Maximizer on \([a, b] = [2, 8]\) is \(x^* = 4\)}\label{\detokenize{02.optimization_intro:id1}}\end{figure}

\begin{figure}[H]
\centering
\capstart

\noindent\sphinxincludegraphics[width=0.800\linewidth]{{2751088387d35324fface5d67b117cffec47bfada7f51cc420592f5fccfadc8c}.png}
\caption{Minimizer on \([a, b] = [2, 8]\) is \(x^{**} = 8\)}\label{\detokenize{02.optimization_intro:id2}}\end{figure}
\end{sphinxadmonition}

\sphinxAtStartPar
The set of maximizers/minimizers can be
\begin{itemize}
\item {} 
\sphinxAtStartPar
empty

\item {} 
\sphinxAtStartPar
a singleton (contains one element)

\item {} 
\sphinxAtStartPar
infinite (contains infinitely many elements)

\end{itemize}

\begin{sphinxadmonition}{note}{Example: infinite maximizers}

\sphinxAtStartPar
\(f \colon [0, 1] \to \mathbb{R}\) defined by \(f(x) =1\)\\
has infinitely many maximizers and minimizers on \([0, 1]\)
\end{sphinxadmonition}

\begin{sphinxadmonition}{note}{Example: no maximizers}

\sphinxAtStartPar
The following function has no maximizers on \([0, 2]\)
\begin{equation*}
\begin{split}
f(x) = 
\begin{cases}
x^2 &  \text{ if } x < 1
\\
1/2 &  \text{ otherwise}
\end{cases}
\end{split}
\end{equation*}
\begin{figure}[H]
\centering
\capstart

\noindent\sphinxincludegraphics[width=0.800\linewidth]{{d3625b746fc2a0f0c7b4f776552674214beed900f88c11ec0d8fb77970831006}.png}
\caption{No maximizer on \([0, 2]\)}\label{\detokenize{02.optimization_intro:id3}}\end{figure}
\end{sphinxadmonition}

\begin{sphinxadmonition}{note}{Definition}

\sphinxAtStartPar
Point  \(x\) is called \sphinxstyleemphasis{\sphinxstylestrong{interior}} to \([a, b]\) if \(a < x < b\)
\end{sphinxadmonition}

\sphinxAtStartPar
The set of all interior points is written \((a, b)\)

\sphinxAtStartPar
We refer to \(x^* \in [a, b]\) as
\begin{itemize}
\item {} 
\sphinxAtStartPar
\sphinxstyleemphasis{\sphinxstylestrong{interior maximizer}} if both a maximizer and interior

\item {} 
\sphinxAtStartPar
\sphinxstyleemphasis{\sphinxstylestrong{interior minimizer}} if both a minimizer and interior

\end{itemize}


\section{Finding optima}
\label{\detokenize{02.optimization_intro:finding-optima}}
\begin{sphinxadmonition}{note}{Definition}

\sphinxAtStartPar
A \sphinxstyleemphasis{\sphinxstylestrong{stationary point}} of \(f\) on \([a, b]\) is an interior point \(x\) with \(f'(x) = 0\)
\end{sphinxadmonition}

\begin{figure}[htbp]
\centering
\capstart

\noindent\sphinxincludegraphics[width=0.800\linewidth]{{stationary}.png}
\caption{Both \(x^*\) and \(x^{**}\) are stationary}\label{\detokenize{02.optimization_intro:id4}}\end{figure}

\begin{sphinxadmonition}{note}{Fact}

\sphinxAtStartPar
If \(f\) is differentiable and \(x^*\) is either an interior minimizer
or an interior maximizer of \(f\) on \([a, b]\), then \(x^*\) is stationary
\end{sphinxadmonition}

\sphinxAtStartPar
Sketch of proof, for maximizers:
\begin{equation*}
\begin{split}
f'(x^*) = \, \lim_{h \to 0} \, \frac{f(x^* + h) - f(x^*)}{h}
\qquad \text{(by def.)}
\end{split}
\end{equation*}\begin{equation*}
\begin{split}
\Rightarrow f(x^* + h) \approx f(x^*) + f'(x^*) h 
\qquad \text{for small } h 
\end{split}
\end{equation*}
\sphinxAtStartPar
If \(f'(x^*) \ne 0\) then exists small \(h\) such that \(f(x^* + h) > f(x^*)\)

\sphinxAtStartPar
Hence interior maximizers must be stationary — otherwise we can do better

\sphinxAtStartPar
\(\Rightarrow\) any interior maximizer stationary\\
\(\Rightarrow\) set of interior maximizers \(\subset\) set of stationary points\\
\(\Rightarrow\) maximizers \(\subset\) stationary points \(\cup \{a\} \cup \{b\}\)

\sphinxAtStartPar
Usage:
\begin{enumerate}
\sphinxsetlistlabels{\arabic}{enumi}{enumii}{}{.}%
\item {} 
\sphinxAtStartPar
Locate stationary points

\item {} 
\sphinxAtStartPar
Evaluate \(y = f(x)\) for each stationary \(x\) and for \(a\), \(b\)

\item {} 
\sphinxAtStartPar
Pick point giving largest \(y\) value

\end{enumerate}

\sphinxAtStartPar
Minimization: same idea

\begin{sphinxadmonition}{note}{Example}

\sphinxAtStartPar
Let’s solve
\begin{equation*}
\begin{split} 
\max_{-2 \leq x \leq 5} f(x) 
\quad \text{where} \quad
f(x) = x^3 - 6x^2 + 4x + 8
\end{split}
\end{equation*}
\sphinxAtStartPar
Steps
\begin{itemize}
\item {} 
\sphinxAtStartPar
Differentiate to get \(f'(x) = 3x^2 - 12x + 4\)

\item {} 
\sphinxAtStartPar
Solve \(3x^2 - 12x + 4 = 0\) to get stationary \(x\)

\item {} 
\sphinxAtStartPar
Discard any stationary points outside \([-2, 5]\)

\item {} 
\sphinxAtStartPar
Eval \(f\) at remaining points plus end points \(-2\) and \(5\)

\item {} 
\sphinxAtStartPar
Pick point giving largest value

\end{itemize}
\end{sphinxadmonition}

\begin{sphinxuseclass}{cell}\begin{sphinxVerbatimInput}

\begin{sphinxuseclass}{cell_input}
\begin{sphinxVerbatim}[commandchars=\\\{\}]
\PYG{k+kn}{from} \PYG{n+nn}{sympy} \PYG{k+kn}{import} \PYG{o}{*}
\PYG{n}{x} \PYG{o}{=} \PYG{n}{Symbol}\PYG{p}{(}\PYG{l+s+s1}{\PYGZsq{}}\PYG{l+s+s1}{x}\PYG{l+s+s1}{\PYGZsq{}}\PYG{p}{)}
\PYG{n}{points} \PYG{o}{=} \PYG{p}{[}\PYG{o}{\PYGZhy{}}\PYG{l+m+mi}{2}\PYG{p}{,} \PYG{l+m+mi}{5}\PYG{p}{]}
\PYG{n}{f} \PYG{o}{=} \PYG{n}{x}\PYG{o}{*}\PYG{o}{*}\PYG{l+m+mi}{3} \PYG{o}{\PYGZhy{}} \PYG{l+m+mi}{6}\PYG{o}{*}\PYG{n}{x}\PYG{o}{*}\PYG{o}{*}\PYG{l+m+mi}{2} \PYG{o}{+} \PYG{l+m+mi}{4}\PYG{o}{*}\PYG{n}{x} \PYG{o}{+} \PYG{l+m+mi}{8}
\PYG{n}{fp} \PYG{o}{=} \PYG{n}{diff}\PYG{p}{(}\PYG{n}{f}\PYG{p}{,} \PYG{n}{x}\PYG{p}{)}
\PYG{n}{spoints} \PYG{o}{=} \PYG{n}{solve}\PYG{p}{(}\PYG{n}{fp}\PYG{p}{,} \PYG{n}{x}\PYG{p}{)}
\PYG{n}{points}\PYG{o}{.}\PYG{n}{extend}\PYG{p}{(}\PYG{n}{spoints}\PYG{p}{)}
\PYG{n}{v} \PYG{o}{=} \PYG{p}{[}\PYG{n}{f}\PYG{o}{.}\PYG{n}{subs}\PYG{p}{(}\PYG{n}{x}\PYG{p}{,} \PYG{n}{c}\PYG{p}{)}\PYG{o}{.}\PYG{n}{evalf}\PYG{p}{(}\PYG{p}{)} \PYG{k}{for} \PYG{n}{c} \PYG{o+ow}{in} \PYG{n}{points}\PYG{p}{]}
\PYG{n}{maximizer} \PYG{o}{=} \PYG{n}{points}\PYG{p}{[}\PYG{n}{v}\PYG{o}{.}\PYG{n}{index}\PYG{p}{(}\PYG{n+nb}{max}\PYG{p}{(}\PYG{n}{v}\PYG{p}{)}\PYG{p}{)}\PYG{p}{]}
\PYG{n+nb}{print}\PYG{p}{(}\PYG{l+s+s2}{\PYGZdq{}}\PYG{l+s+s2}{Maximizer =}\PYG{l+s+s2}{\PYGZdq{}}\PYG{p}{,} \PYG{n+nb}{str}\PYG{p}{(}\PYG{n}{maximizer}\PYG{p}{)}\PYG{p}{,}\PYG{l+s+s1}{\PYGZsq{}}\PYG{l+s+s1}{=}\PYG{l+s+s1}{\PYGZsq{}}\PYG{p}{,}\PYG{n}{maximizer}\PYG{o}{.}\PYG{n}{evalf}\PYG{p}{(}\PYG{p}{)}\PYG{p}{)}
\end{sphinxVerbatim}

\end{sphinxuseclass}\end{sphinxVerbatimInput}
\begin{sphinxVerbatimOutput}

\begin{sphinxuseclass}{cell_output}
\begin{sphinxVerbatim}[commandchars=\\\{\}]
Maximizer = 2 \PYGZhy{} 2*sqrt(6)/3 = 0.367006838144548
\end{sphinxVerbatim}

\end{sphinxuseclass}\end{sphinxVerbatimOutput}

\end{sphinxuseclass}
\begin{sphinxuseclass}{cell}
\begin{sphinxuseclass}{tag_hide-input}\begin{sphinxVerbatimOutput}

\begin{sphinxuseclass}{cell_output}
\noindent{\hspace*{\fill}\sphinxincludegraphics[width=0.800\linewidth]{{83f903ccee15fdbe59aaa8f3cad4ad66e1d89c202520eeced59c3947756e680b}.png}\hspace*{\fill}}

\end{sphinxuseclass}\end{sphinxVerbatimOutput}

\end{sphinxuseclass}
\end{sphinxuseclass}

\section{Shape Conditions and Sufficiency}
\label{\detokenize{02.optimization_intro:shape-conditions-and-sufficiency}}
\sphinxAtStartPar
When is \(f'(x^*) = 0\) sufficient for \(x^*\) to be a maximizer?

\sphinxAtStartPar
One answer: When \(f\) is concave

\begin{sphinxuseclass}{cell}
\begin{sphinxuseclass}{tag_hide-input}\begin{sphinxVerbatimOutput}

\begin{sphinxuseclass}{cell_output}
\noindent{\hspace*{\fill}\sphinxincludegraphics[width=0.800\linewidth]{{0643e7397394ed7d767b2f7c26a238493b608e42274e3969c758f0283e74a9ca}.png}\hspace*{\fill}}

\end{sphinxuseclass}\end{sphinxVerbatimOutput}

\end{sphinxuseclass}
\end{sphinxuseclass}
\sphinxAtStartPar
(Full definition deferred)

\begin{sphinxadmonition}{note}{Sufficient conditions for \sphinxstyleemphasis{concavity} in one dimension}

\sphinxAtStartPar
Let \(f \colon [a, b] \to \mathbb{R}\)
\begin{itemize}
\item {} 
\sphinxAtStartPar
If \(f''(x) \leq 0\) for all \(x \in (a, b)\) then \(f\) is concave on \((a, b)\)

\item {} 
\sphinxAtStartPar
If \(f''(x) < 0\) for all \(x \in (a, b)\) then \(f\) is \sphinxstylestrong{strictly} concave on \((a, b)\)

\end{itemize}
\end{sphinxadmonition}

\begin{sphinxadmonition}{note}{Example}
\begin{itemize}
\item {} 
\sphinxAtStartPar
\(f(x) = a + b x\) is concave on \(\mathbb{R}\) but not strictly

\item {} 
\sphinxAtStartPar
\(f(x) = \log(x)\) is strictly concave on \((0, \infty)\)

\end{itemize}
\end{sphinxadmonition}

\sphinxAtStartPar
When is \(f'(x^*) = 0\) sufficient for \(x^*\) to be a minimizer?

\sphinxAtStartPar
One answer: When \(f\) is convex

\begin{sphinxuseclass}{cell}
\begin{sphinxuseclass}{tag_hide-input}\begin{sphinxVerbatimOutput}

\begin{sphinxuseclass}{cell_output}
\noindent{\hspace*{\fill}\sphinxincludegraphics[width=0.800\linewidth]{{0a2934f61b0beb5e674a16ddd69a364120e5944f75d0dcc17ed81f13a31d073f}.png}\hspace*{\fill}}

\end{sphinxuseclass}\end{sphinxVerbatimOutput}

\end{sphinxuseclass}
\end{sphinxuseclass}
\sphinxAtStartPar
(Full definition deferred)

\begin{sphinxadmonition}{note}{Sufficient conditions for \sphinxstyleemphasis{convexity} in one dimension}

\sphinxAtStartPar
Let \(f \colon [a, b] \to \mathbb{R}\)
\begin{itemize}
\item {} 
\sphinxAtStartPar
If \(f''(x) \geq 0\) for all \(x \in (a, b)\) then \(f\) is convex on \((a, b)\)

\item {} 
\sphinxAtStartPar
If \(f''(x) > 0\) for all \(x \in (a, b)\) then \(f\) is \sphinxstylestrong{strictly}
convex on \((a, b)\)

\end{itemize}
\end{sphinxadmonition}

\begin{sphinxadmonition}{note}{Example}
\begin{itemize}
\item {} 
\sphinxAtStartPar
\(f(x) = a + b x\) is convex on \(\mathbb{R}\) but not strictly

\item {} 
\sphinxAtStartPar
\(f(x) = x^2\) is strictly convex on \(\mathbb{R}\)

\end{itemize}
\end{sphinxadmonition}


\subsection{Sufficiency and uniqueness with shape conditions}
\label{\detokenize{02.optimization_intro:sufficiency-and-uniqueness-with-shape-conditions}}
\begin{sphinxadmonition}{note}{Fact}

\sphinxAtStartPar
For maximizers:
\begin{itemize}
\item {} 
\sphinxAtStartPar
If \(f \colon [a,b] \to \mathbb{R}\) is concave and \(x^* \in (a, b)\) is
stationary then \(x^*\) is a maximizer

\item {} 
\sphinxAtStartPar
If, in addition, \(f\) is strictly concave, then \(x^*\) is the
unique maximizer

\end{itemize}
\end{sphinxadmonition}

\begin{sphinxadmonition}{note}{Fact}

\sphinxAtStartPar
For minimizers:
\begin{itemize}
\item {} 
\sphinxAtStartPar
If \(f \colon [a,b] \to \mathbb{R}\) is convex and \(x^* \in (a, b)\) is
stationary then \(x^*\) is a minimizer

\item {} 
\sphinxAtStartPar
If, in addition, \(f\) is strictly convex, then \(x^*\) is the
unique minimizer

\end{itemize}
\end{sphinxadmonition}

\begin{sphinxadmonition}{note}{Example}

\sphinxAtStartPar
A price taking firm faces output price \(p > 0\), input price \(w >0\)

\sphinxAtStartPar
Maximize profits with respect to input \(\ell\)
\begin{equation*}
\begin{split}
\max_{\ell \ge 0} \pi(\ell) = p f(\ell) - w \ell,
\end{split}
\end{equation*}
\sphinxAtStartPar
where the production technology is given by
\begin{equation*}
\begin{split}
f(\ell) = \ell^{\alpha}, 0 < \alpha < 1.
\end{split}
\end{equation*}\end{sphinxadmonition}

\sphinxAtStartPar
Evidently
\begin{equation*}
\begin{split}
\pi'(\ell) = \alpha p \ell^{\alpha - 1} - w,
\end{split}
\end{equation*}
\sphinxAtStartPar
so unique stationary point is
\begin{equation*}
\begin{split}
\ell^* = (\alpha p/w)^{1/(1 - \alpha)}
\end{split}
\end{equation*}
\sphinxAtStartPar
Moreover,
\begin{equation*}
\begin{split}
\pi''(\ell) = \alpha (\alpha - 1) p \ell^{\alpha - 2} < 0
\end{split}
\end{equation*}
\sphinxAtStartPar
for all \(\ell \ge 0\) so \(\ell^*\) is unique maximizer.

\begin{figure}[htbp]
\centering
\capstart

\noindent\sphinxincludegraphics[width=0.800\linewidth]{{4d0128849c9b4d98dc41b942a10906327f01cfb09802cdae5a0db626b035973a}.png}
\caption{Profit maximization with \(p=2\), \(w=1\), \(\alpha=0.6\), \(\ell^*=\)\DUrole{output,text_plain}{1.5774}}\label{\detokenize{02.optimization_intro:id5}}\end{figure}


\section{Functions of two variables}
\label{\detokenize{02.optimization_intro:functions-of-two-variables}}
\sphinxAtStartPar
Let’s have a look at some functions of two variables
\begin{itemize}
\item {} 
\sphinxAtStartPar
How to visualize them

\item {} 
\sphinxAtStartPar
Slope, contours, etc.

\end{itemize}

\begin{sphinxadmonition}{note}{Example: Cobb\sphinxhyphen{}Douglas production function}

\sphinxAtStartPar
Consider production function
\begin{equation*}
\begin{split}
f(k, \ell) = k^{\alpha} \ell^{\beta}\\
\alpha \ge 0, \, \beta \ge 0, \, \alpha + \beta < 1
\end{split}
\end{equation*}
\sphinxAtStartPar
Let’s graph it in two dimensions.
\end{sphinxadmonition}

\begin{figure}[htbp]
\centering
\capstart

\noindent\sphinxincludegraphics[width=0.800\linewidth]{{prod2d}.png}
\caption{Production function with \(\alpha=0.4\), \(\beta=0.5\) (a)}\label{\detokenize{02.optimization_intro:id6}}\end{figure}

\begin{figure}[htbp]
\centering
\capstart

\noindent\sphinxincludegraphics[width=0.800\linewidth]{{prod2d_1}.png}
\caption{Production function with \(\alpha=0.4\), \(\beta=0.5\) (b)}\label{\detokenize{02.optimization_intro:id7}}\end{figure}

\begin{figure}[htbp]
\centering
\capstart

\noindent\sphinxincludegraphics[width=0.800\linewidth]{{prod2d_2}.png}
\caption{Production function with \(\alpha=0.4\), \(\beta=0.5\) (c)}\label{\detokenize{02.optimization_intro:id8}}\end{figure}

\sphinxAtStartPar
Like many 3D plots it’s hard to get a good understanding

\sphinxAtStartPar
Let’s try again with contours plus heat map

\begin{figure}[htbp]
\centering
\capstart

\noindent\sphinxincludegraphics[width=0.800\linewidth]{{prodcontour}.png}
\caption{Production function with \(\alpha=0.4\), \(\beta=0.5\), contours}\label{\detokenize{02.optimization_intro:id9}}\end{figure}

\sphinxAtStartPar
In this context the contour lines are called \sphinxstyleemphasis{\sphinxstylestrong{isoquants}}

\sphinxAtStartPar
Can you see how \(\alpha < \beta\) shows up in the slope of the contours?

\sphinxAtStartPar
We can drop the colours to see the numbers more clearly

\begin{figure}[htbp]
\centering
\capstart

\noindent\sphinxincludegraphics[width=0.800\linewidth]{{prodcontour2}.png}
\caption{Production function with \(\alpha=0.4\), \(\beta=0.5\)}\label{\detokenize{02.optimization_intro:id10}}\end{figure}

\begin{sphinxadmonition}{note}{Example: log\sphinxhyphen{}utility}

\sphinxAtStartPar
Let \(u(x_1,x_2)\) be “utility” gained from \(x_1\) units of good 1 and \(x_2\) units of good 2

\sphinxAtStartPar
We take
\begin{equation*}
\begin{split}
u(x_1, x_2) = \alpha \log(x_1) + \beta \log(x_2)
\end{split}
\end{equation*}
\sphinxAtStartPar
where
\begin{itemize}
\item {} 
\sphinxAtStartPar
\(\alpha\) and \(\beta\) are parameters

\item {} 
\sphinxAtStartPar
we assume \(\alpha>0, \, \beta > 0\)

\item {} 
\sphinxAtStartPar
The log functions mean “diminishing returns” in each good

\end{itemize}
\end{sphinxadmonition}

\begin{figure}[htbp]
\centering
\capstart

\noindent\sphinxincludegraphics[width=0.800\linewidth]{{log_util}.png}
\caption{Log utility with \(\alpha=0.4\), \(\beta=0.5\)}\label{\detokenize{02.optimization_intro:id11}}\end{figure}

\sphinxAtStartPar
Let’s look at the contour lines

\sphinxAtStartPar
For utility functions, contour lines called \sphinxstyleemphasis{\sphinxstylestrong{indifference curves}}

\begin{figure}[htbp]
\centering
\capstart

\noindent\sphinxincludegraphics[width=0.800\linewidth]{{log_util_contour}.png}
\caption{Indifference curves of log utility with \(\alpha=0.4\), \(\beta=0.5\)}\label{\detokenize{02.optimization_intro:id12}}\end{figure}

\begin{sphinxadmonition}{note}{Example: quasi\sphinxhyphen{}linear utility}
\begin{equation*}
\begin{split}
u(x_1, x_2) = x_1 + \log(x_2)
\end{split}
\end{equation*}\begin{itemize}
\item {} 
\sphinxAtStartPar
Called quasi\sphinxhyphen{}linear because linear in good 1

\end{itemize}
\end{sphinxadmonition}

\begin{figure}[htbp]
\centering
\capstart

\noindent\sphinxincludegraphics[width=0.800\linewidth]{{ql_utility}.png}
\caption{Quasi\sphinxhyphen{}linear utility}\label{\detokenize{02.optimization_intro:id13}}\end{figure}

\begin{figure}[htbp]
\centering
\capstart

\noindent\sphinxincludegraphics[width=0.800\linewidth]{{ql_utility_contour}.png}
\caption{Indifference curves of quasi\sphinxhyphen{}linear utility}\label{\detokenize{02.optimization_intro:id14}}\end{figure}

\begin{sphinxadmonition}{note}{Example: quadratic utility}
\begin{equation*}
\begin{split}
u(x_1, x_2) = - (x_1 - b_1)^2 - (x_2 - b_2)^2
\end{split}
\end{equation*}
\sphinxAtStartPar
Here
\begin{itemize}
\item {} 
\sphinxAtStartPar
\(b_1\) is a “satiation” or “bliss” point for \(x_1\)

\item {} 
\sphinxAtStartPar
\(b_2\) is a “satiation” or “bliss” point for \(x_2\)

\end{itemize}
\end{sphinxadmonition}

\sphinxAtStartPar
Dissatisfaction increases with deviations from the bliss points

\begin{figure}[htbp]
\centering
\capstart

\noindent\sphinxincludegraphics[width=0.800\linewidth]{{quad_util}.png}
\caption{Quadratic utility with \(b_1 = 3\) and \(b_2 = 2\)}\label{\detokenize{02.optimization_intro:id15}}\end{figure}

\begin{figure}[htbp]
\centering
\capstart

\noindent\sphinxincludegraphics[width=0.800\linewidth]{{quad_util_contour}.png}
\caption{Indifference curves quadratic utility with \(b_1 = 3\) and \(b_2 = 2\)}\label{\detokenize{02.optimization_intro:id16}}\end{figure}


\section{Bivariate Optimization}
\label{\detokenize{02.optimization_intro:bivariate-optimization}}
\sphinxAtStartPar
Consider \(f \colon I \to \mathbb{R}\) where \(I \subset \mathbb{R}^2\)

\sphinxAtStartPar
The set \(\mathbb{R}^2\) is all \((x_1, x_2)\) pairs

\begin{sphinxadmonition}{note}{Definition}

\sphinxAtStartPar
A point \((x_1^*, x_2^*) \in I\) is called a \sphinxstyleemphasis{\sphinxstylestrong{maximizer}} of \(f\) on \(I\) if
\begin{equation*}
\begin{split}
f(x_1^*, x_2^*) \geq f(x_1, x_2) 
\quad \text{for all} \quad
(x_1, x_2) \in I
\end{split}
\end{equation*}\end{sphinxadmonition}

\begin{sphinxadmonition}{note}{Definition}

\sphinxAtStartPar
A point \((x_1^*, x_2^*) \in I\) is called a \sphinxstyleemphasis{\sphinxstylestrong{minimizer}} of \(f\) on \(I\) if
\begin{equation*}
\begin{split}
f(x_1^*, x_2^*) \leq f(x_1, x_2) 
\quad \text{for all} \quad
(x_1, x_2) \in I
\end{split}
\end{equation*}\end{sphinxadmonition}

\sphinxAtStartPar
When they exist, the partial derivatives at \((x_1, x_2) \in I\) are
\begin{equation*}
\begin{split}
f_1(x_1, x_2) = \frac{\partial}{\partial x_1} f(x_1, x_2)
\\
f_2(x_1, x_2) = \frac{\partial}{\partial x_2} f(x_1, x_2)
\end{split}
\end{equation*}
\begin{sphinxadmonition}{note}{Example}

\sphinxAtStartPar
When \(f(k, \ell) = k^\alpha \ell^\beta\),
\begin{equation*}
\begin{split}
f_1(k, \ell) 
= \frac{\partial}{\partial k} f(k, \ell)
= \frac{\partial}{\partial k} k^\alpha \ell^\beta
= \alpha k^{\alpha-1} \ell^\beta
\end{split}
\end{equation*}\end{sphinxadmonition}

\begin{sphinxadmonition}{note}{Definition}

\sphinxAtStartPar
An interior point \((x_1, x_2) \in I\) is called \sphinxstyleemphasis{\sphinxstylestrong{stationary}} for \(f\) if
\begin{equation*}
\begin{split}
f_1(x_1, x_2) = f_2(x_1, x_2) = 0
\end{split}
\end{equation*}\end{sphinxadmonition}

\begin{sphinxadmonition}{note}{Fact}

\sphinxAtStartPar
Let \(f \colon I \to \mathbb{R}\) be a continuously differentiable function.
If \((x_1^*, x_2^*)\) is either
\begin{itemize}
\item {} 
\sphinxAtStartPar
an interior maximizer of \(f\) on \(I\), or

\item {} 
\sphinxAtStartPar
an interior minimizer of \(f\) on \(I\),

\end{itemize}

\sphinxAtStartPar
then \((x_1^*, x_2^*)\) is a stationary point of \(f\)
\end{sphinxadmonition}

\sphinxAtStartPar
Usage, for maximization:
\begin{enumerate}
\sphinxsetlistlabels{\arabic}{enumi}{enumii}{}{.}%
\item {} 
\sphinxAtStartPar
Compute partials

\item {} 
\sphinxAtStartPar
Set partials to zero to find \(S =\) all stationary points

\item {} 
\sphinxAtStartPar
Evaluate candidates in \(S\) and boundary of \(I\)

\item {} 
\sphinxAtStartPar
Select point \((x^*_1, x_2^*)\) yielding highest value

\end{enumerate}

\begin{sphinxadmonition}{note}{Example}
\begin{equation*}
\begin{split}
f(x_1, x_2) = x_1^2 + 4 x_2^2 \rightarrow \min
\quad \mathrm{s.t.} \quad
x_1 + x_2 \leq 1
\end{split}
\end{equation*}\end{sphinxadmonition}

\sphinxAtStartPar
Setting
\begin{equation*}
\begin{split}
f_1(x_1, x_2) = 2 x_1 = 0 
\quad \text{and} \quad
f_2(x_1, x_2) = 8 x_2 = 0 
\end{split}
\end{equation*}
\sphinxAtStartPar
gives the unique stationary point \((0, 0)\), at which \(f(0, 0) = 0\)

\sphinxAtStartPar
On the boundary we have \(x_1 + x_2 = 1\), so
\begin{equation*}
\begin{split}
f(x_1, x_2) 
= f(x_1, 1 - x_1) 
= x_1^2 + 4 (1 - x_1)^2
\end{split}
\end{equation*}
\sphinxAtStartPar
\sphinxstylestrong{Exercise:} Show right hand side \(> 0\) for any \(x_1\)

\sphinxAtStartPar
Hence minimizer is \((x_1^*, x_2^*) = (0, 0)\)


\subsection{Nasty secrets}
\label{\detokenize{02.optimization_intro:nasty-secrets}}
\sphinxAtStartPar
Solving for \((x_1, x_2)\) such that \(f_1(x_1, x_2) = 0\) and \(f_2(x_1, x_2) = 0\) can be hard
\begin{itemize}
\item {} 
\sphinxAtStartPar
System of nonlinear equations

\item {} 
\sphinxAtStartPar
Might have no analytical solution

\item {} 
\sphinxAtStartPar
Set of solutions can be a continuum

\end{itemize}

\begin{sphinxadmonition}{note}{Example}

\sphinxAtStartPar
(Don’t) try to find all stationary  points of
\begin{equation*}
\begin{split}
f(x_1, x_2) = \frac{\cos(x_1^2 + x_2^2) + x_1^2 + x_1}{2 +
p(-x_1^2) + \sin^2(x_2)}
\end{split}
\end{equation*}\end{sphinxadmonition}

\sphinxAtStartPar
Also:
\begin{itemize}
\item {} 
\sphinxAtStartPar
Boundary is often a continuum, not just two points

\item {} 
\sphinxAtStartPar
Things get even harder in higher dimensions

\end{itemize}

\sphinxAtStartPar
On the other hand:
\begin{itemize}
\item {} 
\sphinxAtStartPar
Most classroom examples are chosen to avoid these problems

\item {} 
\sphinxAtStartPar
Life is still pretty easy if we have concavity / convexity

\item {} 
\sphinxAtStartPar
Clever tricks have been found for certain kinds of problems

\end{itemize}


\section{Second Order Partials}
\label{\detokenize{02.optimization_intro:second-order-partials}}
\sphinxAtStartPar
Let \(f \colon I \to \mathbb{R}\) and, when they exist, denote
\begin{equation*}
\begin{split}
f_{11}(x_1, x_2) 
= \frac{\partial^2}{\partial x_1^2} 
f(x_1, x_2)
\end{split}
\end{equation*}\begin{equation*}
\begin{split}
f_{12}(x_1, x_2) 
= \frac{\partial^2}{\partial x_1 \partial x_2} 
f(x_1, x_2)
\end{split}
\end{equation*}\begin{equation*}
\begin{split}
f_{21}(x_1, x_2) 
= \frac{\partial^2}{\partial x_2 \partial x_1} 
f(x_1, x_2)
\end{split}
\end{equation*}\begin{equation*}
\begin{split}
f_{22}(x_1, x_2) 
= \frac{\partial^2}{\partial x_2^2} 
f(x_1, x_2)
\end{split}
\end{equation*}
\begin{sphinxadmonition}{note}{Example: Cobb\sphinxhyphen{}Douglas technology with linear costs}

\sphinxAtStartPar
If \(\pi(k, \ell) = p k^{\alpha} \ell^{\beta} - w \ell - r k\) then
\begin{equation*}
\begin{split}
\pi_{11}(k, \ell) = p \alpha(\alpha-1) k^{\alpha-2} \ell^{\beta}
\end{split}
\end{equation*}\begin{equation*}
\begin{split}
\pi_{12}(k, \ell) = p \alpha\beta k^{\alpha-1} \ell^{\beta-1}
\end{split}
\end{equation*}\begin{equation*}
\begin{split}
\pi_{21}(k, \ell) = p \alpha\beta k^{\alpha-1} \ell^{\beta-1}
\end{split}
\end{equation*}\begin{equation*}
\begin{split}
\pi_{22}(k, \ell) = p \beta(\beta-1) k^{\alpha} \ell^{\beta-2}
\end{split}
\end{equation*}\end{sphinxadmonition}

\begin{sphinxadmonition}{note}{Fact}

\sphinxAtStartPar
If \(f \colon I \to \mathbb{R}\) is twice continuously differentiable at \((x_1, x_2)\), then
\begin{equation*}
\begin{split}
f_{12}(x_1, x_2) = f_{21}(x_1, x_2)
\end{split}
\end{equation*}\end{sphinxadmonition}

\sphinxAtStartPar
\sphinxstylestrong{Exercise:} Confirm the results in the exercise above.


\section{Shape conditions in 2D}
\label{\detokenize{02.optimization_intro:shape-conditions-in-2d}}
\sphinxAtStartPar
Let \(I\) be an “open” set (only interior points – formalities next week)

\sphinxAtStartPar
Let \(f \colon I \to \mathbb{R}\) be twice continuously differentiable

\sphinxAtStartPar
The function \(f\) is strictly \sphinxstylestrong{concave} on \(I\) if, for any \((x_1, x_2) \in I\)
\begin{enumerate}
\sphinxsetlistlabels{\arabic}{enumi}{enumii}{}{.}%
\item {} 
\sphinxAtStartPar
\(f_{11}(x_1, x_2) < 0\)

\item {} 
\sphinxAtStartPar
\(f_{11}(x_1, x_2) \, f_{22}(x_1, x_2) >  f_{12}(x_1, x_2)^2\)

\end{enumerate}

\sphinxAtStartPar
The function \(f\) is strictly \sphinxstylestrong{convex} on \(I\) if, for any \((x_1, x_2) \in I\)
\begin{enumerate}
\sphinxsetlistlabels{\arabic}{enumi}{enumii}{}{.}%
\item {} 
\sphinxAtStartPar
\(f_{11}(x_1, x_2) > 0\)

\item {} 
\sphinxAtStartPar
\(f_{11}(x_1, x_2) \, f_{22}(x_1, x_2) >  f_{12}(x_1, x_2)^2\)

\end{enumerate}

\sphinxAtStartPar
When is stationarity sufficient?

\begin{sphinxadmonition}{note}{Fact}

\sphinxAtStartPar
If \(f\) is differentiable and strictly concave on \(I\), then any
stationary point of \(f\) is also a unique maximizer of \(f\) on \(I\)
\end{sphinxadmonition}

\begin{sphinxadmonition}{note}{Fact}

\sphinxAtStartPar
If \(f\) is differentiable and strictly convex on \(I\), then any
stationary point of \(f\) is also a unique minimizer of \(f\) on \(I\)
\end{sphinxadmonition}

\begin{figure}[htbp]
\centering
\capstart

\noindent\sphinxincludegraphics[width=0.800\linewidth]{{concave_max}.png}
\caption{Maximizer of a concave function}\label{\detokenize{02.optimization_intro:id17}}\end{figure}

\begin{figure}[htbp]
\centering
\capstart

\noindent\sphinxincludegraphics[width=0.800\linewidth]{{convex_min}.png}
\caption{Minimizer of a convex function}\label{\detokenize{02.optimization_intro:id18}}\end{figure}

\begin{sphinxadmonition}{note}{Example: unconstrained maximization of quadratic utility}
\begin{equation*}
\begin{split}
u(x_1, x_2) = - (x_1 - b_1)^2 - (x_2 - b_2)^2
\rightarrow \max_{x_1, x_2}
\end{split}
\end{equation*}\end{sphinxadmonition}

\sphinxAtStartPar
Intuitively the solution is \(x_1^*=b_1\) and \(x_2^*=b_2\)

\sphinxAtStartPar
Analysis above leads to the same conclusion

\sphinxAtStartPar
First let’s check first order conditions (\sphinxstyleemphasis{F.O.C.})
\begin{equation*}
\begin{split}
\frac{\partial}{\partial x_1}
u(x_1, x_2) = -2 (x_1 - b_1) = 0
\quad \implies \quad
x_1 = b_1
\end{split}
\end{equation*}\begin{equation*}
\begin{split}
\frac{\partial}{\partial x_2}
u(x_1, x_2) = -2 (x_2 - b_2) = 0
\quad \implies \quad
x_2 = b_2
\end{split}
\end{equation*}
\sphinxAtStartPar
How about (strict) concavity?

\sphinxAtStartPar
Sufficient condition is
\begin{enumerate}
\sphinxsetlistlabels{\arabic}{enumi}{enumii}{}{.}%
\item {} 
\sphinxAtStartPar
\(u_{11}(x_1, x_2) < 0\)

\item {} 
\sphinxAtStartPar
\(u_{11}(x_1, x_2)u_{22}(x_1, x_2) > u_{12}(x_1, x_2)^2\)

\end{enumerate}

\sphinxAtStartPar
We have
\begin{itemize}
\item {} 
\sphinxAtStartPar
\(u_{11}(x_1, x_2) = -2\)

\item {} 
\sphinxAtStartPar
\(u_{11}(x_1, x_2)u_{22}(x_1, x_2) = 4 > 0 = u_{12}(x_1, x_2)^2\)

\end{itemize}

\begin{sphinxadmonition}{note}{Example: Profit maximization with two inputs}
\begin{equation*}
\begin{split}
\pi(k, \ell) 
= p k^{\alpha} \ell^{\beta} - w \ell - r k
\rightarrow \max_{k, \ell}
\end{split}
\end{equation*}
\sphinxAtStartPar
where \( \alpha, \beta, p, w\) are all positive and \(\alpha + \beta < 1\)
\end{sphinxadmonition}

\sphinxAtStartPar
Derivatives:
\begin{itemize}
\item {} 
\sphinxAtStartPar
\(\pi_1(k, \ell) = p \alpha k^{\alpha-1} \ell^{\beta} - r\)

\item {} 
\sphinxAtStartPar
\(\pi_2(k, \ell) = p \beta k^{\alpha} \ell^{\beta-1} - w\)

\item {} 
\sphinxAtStartPar
\(\pi_{11}(k, \ell) = p \alpha(\alpha-1) k^{\alpha-2} \ell^{\beta}\)

\item {} 
\sphinxAtStartPar
\(\pi_{22}(k, \ell) = p \beta(\beta-1) k^{\alpha} \ell^{\beta-2}\)

\item {} 
\sphinxAtStartPar
\(\pi_{12}(k, \ell) = p \alpha \beta k^{\alpha-1} \ell^{\beta-1}\)

\end{itemize}

\sphinxAtStartPar
First order conditions: set
\begin{equation*}
\begin{split}
\pi_1(k, \ell) = 0
\\
\pi_2(k, \ell) = 0
\end{split}
\end{equation*}
\sphinxAtStartPar
and solve simultaneously for \(k, \ell\) to get
\begin{equation*}
\begin{split}
k^* =
\left[ 
p (\alpha/r)^{1 - \beta}  (\beta/w)^{\beta}
\right]^{1 / (1 - \alpha - \beta)}
\\
\ell^* =
\left[ 
p (\beta/w)^{1 - \alpha}  (\alpha/r)^{\alpha}
\right]^{1 / (1 - \alpha - \beta)}
\end{split}
\end{equation*}
\sphinxAtStartPar
\sphinxstylestrong{Exercise:} Verify

\sphinxAtStartPar
Now we check second order conditions, hoping for strict concavity

\sphinxAtStartPar
What we need: for any \(k, \ell > 0\)
\begin{enumerate}
\sphinxsetlistlabels{\arabic}{enumi}{enumii}{}{.}%
\item {} 
\sphinxAtStartPar
\(\pi_{11}(k, \ell) < 0\)

\item {} 
\sphinxAtStartPar
\(\pi_{11}(k, \ell) \, \pi_{22}(k, \ell) >  \pi_{12}(k, \ell)^2\)

\end{enumerate}

\sphinxAtStartPar
\sphinxstylestrong{Exercise:} Show both inequalities satisfied when \(\alpha + \beta < 1\)

\begin{figure}[htbp]
\centering
\capstart

\noindent\sphinxincludegraphics[width=0.800\linewidth]{{optprod}.png}
\caption{Profit function when \(p=5\), \(r=w=2\), \(\alpha=0.4\), \(\beta=0.5\)}\label{\detokenize{02.optimization_intro:id19}}\end{figure}

\begin{figure}[htbp]
\centering
\capstart

\noindent\sphinxincludegraphics[width=0.800\linewidth]{{optprod_contour}.png}
\caption{Optimal choice, \(p=5\), \(r=w=2\), \(\alpha=0.4\), \(\beta=0.5\)}\label{\detokenize{02.optimization_intro:id20}}\end{figure}

\sphinxstepscope


\chapter{Elements of set theory and analysis}
\label{\detokenize{03.set_theory:elements-of-set-theory-and-analysis}}\label{\detokenize{03.set_theory::doc}}
\noindent{\hspace*{\fill}\sphinxincludegraphics[scale=1.0]{{coming_soon}.png}\hspace*{\fill}}

\sphinxstepscope


\chapter{Elements of linear algebra}
\label{\detokenize{04.linear_algebra:elements-of-linear-algebra}}\label{\detokenize{04.linear_algebra::doc}}
\noindent{\hspace*{\fill}\sphinxincludegraphics[scale=1.0]{{coming_soon}.png}\hspace*{\fill}}

\sphinxstepscope


\chapter{Elements of probability}
\label{\detokenize{05.probability:elements-of-probability}}\label{\detokenize{05.probability::doc}}
\noindent{\hspace*{\fill}\sphinxincludegraphics[scale=1.0]{{coming_soon}.png}\hspace*{\fill}}

\sphinxstepscope


\chapter{Fundamentals of optimization}
\label{\detokenize{06.optimization_fundamentals:fundamentals-of-optimization}}\label{\detokenize{06.optimization_fundamentals::doc}}
\noindent{\hspace*{\fill}\sphinxincludegraphics[scale=1.0]{{coming_soon}.png}\hspace*{\fill}}

\sphinxstepscope


\chapter{Unconstrained optimization}
\label{\detokenize{07.unconstrained:unconstrained-optimization}}\label{\detokenize{07.unconstrained::doc}}
\noindent{\hspace*{\fill}\sphinxincludegraphics[scale=1.0]{{coming_soon}.png}\hspace*{\fill}}

\sphinxstepscope


\chapter{Constrained optimization}
\label{\detokenize{08.constrained:constrained-optimization}}\label{\detokenize{08.constrained::doc}}
\noindent{\hspace*{\fill}\sphinxincludegraphics[scale=1.0]{{coming_soon}.png}\hspace*{\fill}}

\sphinxstepscope


\chapter{Practical session}
\label{\detokenize{09.practical_session:practical-session}}\label{\detokenize{09.practical_session::doc}}
\noindent{\hspace*{\fill}\sphinxincludegraphics[scale=1.0]{{coming_soon}.png}\hspace*{\fill}}

\sphinxstepscope


\chapter{Envelope and maximum theorems}
\label{\detokenize{10.envelope_maximum:envelope-and-maximum-theorems}}\label{\detokenize{10.envelope_maximum::doc}}
\noindent{\hspace*{\fill}\sphinxincludegraphics[scale=1.0]{{coming_soon}.png}\hspace*{\fill}}

\sphinxstepscope


\chapter{Dynamic optimization}
\label{\detokenize{11.dynamic:dynamic-optimization}}\label{\detokenize{11.dynamic::doc}}
\noindent{\hspace*{\fill}\sphinxincludegraphics[scale=1.0]{{coming_soon}.png}\hspace*{\fill}}

\sphinxstepscope


\chapter{Revision}
\label{\detokenize{12.revision:revision}}\label{\detokenize{12.revision::doc}}
\noindent{\hspace*{\fill}\sphinxincludegraphics[scale=1.0]{{coming_soon}.png}\hspace*{\fill}}







\renewcommand{\indexname}{Index}
\printindex
\end{document}