%% Generated by Sphinx.
\def\sphinxdocclass{jupyterBook}
\documentclass[letterpaper,10pt,english]{jupyterBook}
\ifdefined\pdfpxdimen
   \let\sphinxpxdimen\pdfpxdimen\else\newdimen\sphinxpxdimen
\fi \sphinxpxdimen=.75bp\relax
\ifdefined\pdfimageresolution
    \pdfimageresolution= \numexpr \dimexpr1in\relax/\sphinxpxdimen\relax
\fi
%% let collapsible pdf bookmarks panel have high depth per default
\PassOptionsToPackage{bookmarksdepth=5}{hyperref}
%% turn off hyperref patch of \index as sphinx.xdy xindy module takes care of
%% suitable \hyperpage mark-up, working around hyperref-xindy incompatibility
\PassOptionsToPackage{hyperindex=false}{hyperref}
%% memoir class requires extra handling
\makeatletter\@ifclassloaded{memoir}
{\ifdefined\memhyperindexfalse\memhyperindexfalse\fi}{}\makeatother

\PassOptionsToPackage{warn}{textcomp}

\catcode`^^^^00a0\active\protected\def^^^^00a0{\leavevmode\nobreak\ }
\usepackage{cmap}
\usepackage{fontspec}
\defaultfontfeatures[\rmfamily,\sffamily,\ttfamily]{}
\usepackage{amsmath,amssymb,amstext}
\usepackage{polyglossia}
\setmainlanguage{english}



\setmainfont{FreeSerif}[
  Extension      = .otf,
  UprightFont    = *,
  ItalicFont     = *Italic,
  BoldFont       = *Bold,
  BoldItalicFont = *BoldItalic
]
\setsansfont{FreeSans}[
  Extension      = .otf,
  UprightFont    = *,
  ItalicFont     = *Oblique,
  BoldFont       = *Bold,
  BoldItalicFont = *BoldOblique,
]
\setmonofont{FreeMono}[
  Extension      = .otf,
  UprightFont    = *,
  ItalicFont     = *Oblique,
  BoldFont       = *Bold,
  BoldItalicFont = *BoldOblique,
]



\usepackage[Bjarne]{fncychap}
\usepackage[,numfigreset=1,mathnumfig]{sphinx}

\fvset{fontsize=\small}
\usepackage{geometry}


% Include hyperref last.
\usepackage{hyperref}
% Fix anchor placement for figures with captions.
\usepackage{hypcap}% it must be loaded after hyperref.
% Set up styles of URL: it should be placed after hyperref.
\urlstyle{same}


\usepackage{sphinxmessages}



        % Start of preamble defined in sphinx-jupyterbook-latex %
         \usepackage[Latin,Greek]{ucharclasses}
        \usepackage{unicode-math}
        % fixing title of the toc
        \addto\captionsenglish{\renewcommand{\contentsname}{Contents}}
        \hypersetup{
            pdfencoding=auto,
            psdextra
        }
        % End of preamble defined in sphinx-jupyterbook-latex %
        

\title{ECON2125/6012}
\date{Aug 09, 2023}
\release{}
\author{Fedor Iskhakov}
\newcommand{\sphinxlogo}{\vbox{}}
\renewcommand{\releasename}{}
\makeindex
\begin{document}

\pagestyle{empty}
\sphinxmaketitle
\pagestyle{plain}
\sphinxtableofcontents
\pagestyle{normal}
\phantomsection\label{\detokenize{00.index::doc}}


\begin{DUlineblock}{0em}
\item[] \sphinxstylestrong{\Large Preliminary schedule}
\end{DUlineblock}


\begin{savenotes}\sphinxattablestart
\centering
\begin{tabulary}{\linewidth}[t]{|T|T|T|T|}
\hline
\sphinxstyletheadfamily 
\sphinxAtStartPar
Week
&\sphinxstyletheadfamily 
\sphinxAtStartPar
Date
&\sphinxstyletheadfamily 
\sphinxAtStartPar
Topic
&\sphinxstyletheadfamily 
\sphinxAtStartPar
Notes
\\
\hline
\sphinxAtStartPar
1
&
\sphinxAtStartPar
July 27
&
\sphinxAtStartPar
{\hyperref[\detokenize{01.introduction::doc}]{\sphinxcrossref{\DUrole{doc,std,std-doc}{Introduction}}}}
&
\sphinxAtStartPar
Recorded lecture
\\
\hline
\sphinxAtStartPar
2
&
\sphinxAtStartPar
Aug 3
&
\sphinxAtStartPar
{\hyperref[\detokenize{02.optimization_intro::doc}]{\sphinxcrossref{\DUrole{doc,std,std-doc}{Univariate and bivariate optimization}}}}
&
\sphinxAtStartPar
Tutorials start
\\
\hline
\sphinxAtStartPar
3
&
\sphinxAtStartPar
Aug 10
&
\sphinxAtStartPar
{\hyperref[\detokenize{03.set_theory::doc}]{\sphinxcrossref{\DUrole{doc,std,std-doc}{Elements of set theory and analysis}}}}
&
\sphinxAtStartPar

\\
\hline
\sphinxAtStartPar
4
&
\sphinxAtStartPar
Aug 17
&
\sphinxAtStartPar
{\hyperref[\detokenize{04.linear_algebra::doc}]{\sphinxcrossref{\DUrole{doc,std,std-doc}{Elements of linear algebra}}}}
&
\sphinxAtStartPar

\\
\hline
\sphinxAtStartPar
Test
&
\sphinxAtStartPar

&
\sphinxAtStartPar
15\%
&
\sphinxAtStartPar
Submit by Aug 23
\\
\hline
\sphinxAtStartPar
5
&
\sphinxAtStartPar
Aug 24
&
\sphinxAtStartPar
{\hyperref[\detokenize{05.probability::doc}]{\sphinxcrossref{\DUrole{doc,std,std-doc}{Elements of Probability}}}}
&
\sphinxAtStartPar

\\
\hline
\sphinxAtStartPar
6
&
\sphinxAtStartPar
Aug 31
&
\sphinxAtStartPar
{\hyperref[\detokenize{06.optimization_fundamentals::doc}]{\sphinxcrossref{\DUrole{doc,std,std-doc}{Fundamentals of optimization}}}}
&
\sphinxAtStartPar

\\
\hline
\sphinxAtStartPar
Test
&
\sphinxAtStartPar

&
\sphinxAtStartPar
15\%
&
\sphinxAtStartPar
Submit by Sept 3
\\
\hline
\sphinxAtStartPar
Break
&
\sphinxAtStartPar

&
\sphinxAtStartPar

&
\sphinxAtStartPar
2 weeks
\\
\hline
\sphinxAtStartPar
7
&
\sphinxAtStartPar
Sept 21
&
\sphinxAtStartPar
{\hyperref[\detokenize{07.unconstrained::doc}]{\sphinxcrossref{\DUrole{doc,std,std-doc}{Unconstrained optimization}}}}
&
\sphinxAtStartPar

\\
\hline
\sphinxAtStartPar
8
&
\sphinxAtStartPar
Sept 28
&
\sphinxAtStartPar
{\hyperref[\detokenize{08.constrained::doc}]{\sphinxcrossref{\DUrole{doc,std,std-doc}{Constrained optimization}}}}
&
\sphinxAtStartPar

\\
\hline
\sphinxAtStartPar
Test
&
\sphinxAtStartPar

&
\sphinxAtStartPar
15\%
&
\sphinxAtStartPar
Submit by Oct 4
\\
\hline
\sphinxAtStartPar
9
&
\sphinxAtStartPar
Oct 5
&
\sphinxAtStartPar
{\hyperref[\detokenize{09.practical_session::doc}]{\sphinxcrossref{\DUrole{doc,std,std-doc}{Practical session/invited speaker}}}}
&
\sphinxAtStartPar
TBA
\\
\hline
\sphinxAtStartPar
10
&
\sphinxAtStartPar
Oct 12
&
\sphinxAtStartPar
{\hyperref[\detokenize{10.envelope_maximum::doc}]{\sphinxcrossref{\DUrole{doc,std,std-doc}{Envelope and maximum theorems}}}}
&
\sphinxAtStartPar

\\
\hline
\sphinxAtStartPar
11
&
\sphinxAtStartPar
Oct 19
&
\sphinxAtStartPar
{\hyperref[\detokenize{11.dynamic::doc}]{\sphinxcrossref{\DUrole{doc,std,std-doc}{Dynamic optimization}}}}
&
\sphinxAtStartPar

\\
\hline
\sphinxAtStartPar
12
&
\sphinxAtStartPar
Oct 26
&
\sphinxAtStartPar
{\hyperref[\detokenize{12.revision::doc}]{\sphinxcrossref{\DUrole{doc,std,std-doc}{Revision}}}}
&
\sphinxAtStartPar

\\
\hline
\sphinxAtStartPar
Exam
&
\sphinxAtStartPar

&
\sphinxAtStartPar
55\%
&
\sphinxAtStartPar
During exam period
\\
\hline
\end{tabulary}
\par
\sphinxattableend\end{savenotes}

\begin{DUlineblock}{0em}
\item[] \sphinxstylestrong{\large ANU course pages}
\end{DUlineblock}

\sphinxAtStartPar
\sphinxhref{https://wattlecourses.anu.edu.au/course/view.php?id=41102}{Course Wattle page}
Schedule, announcements, teaching team contacts, recordings, assignement, grades

\sphinxAtStartPar
\sphinxhref{https://programsandcourses.anu.edu.au/2023/course/ECON2125\#terms}{Course overview}
\sphinxhref{https://programsandcourses.anu.edu.au/course/ECON2125/Second\%20Semester/6275}{Class summary}
General course description in ANU Programs and Courses



\sphinxstepscope


\chapter{Welcome}
\label{\detokenize{01.introduction:welcome}}\label{\detokenize{01.introduction::doc}}
\sphinxAtStartPar
Course title: \sphinxstylestrong{“Optimization for Economics and Financial Economics”}
\begin{itemize}
\item {} 
\sphinxAtStartPar
Elective second year course in the \sphinxstyleemphasis{Bachelor of Economics} program ECON2125

\item {} 
\sphinxAtStartPar
Compulsory second math course in the \sphinxstyleemphasis{Master of Economics} program ECON6012

\end{itemize}

\sphinxAtStartPar
The two courses are identical in content and assessment, but final grades may be adjusted depending on your program.


\section{Plan for this lecture}
\label{\detokenize{01.introduction:plan-for-this-lecture}}\begin{enumerate}
\sphinxsetlistlabels{\arabic}{enumi}{enumii}{}{.}%
\item {} 
\sphinxAtStartPar
Organization

\item {} 
\sphinxAtStartPar
Administrative topics

\item {} 
\sphinxAtStartPar
Course content

\item {} 
\sphinxAtStartPar
Self\sphinxhyphen{}learning materials

\end{enumerate}


\section{Instructor}
\label{\detokenize{01.introduction:instructor}}
\sphinxAtStartPar
\sphinxstylestrong{Fedor Iskhakov}
Professor of Economics at RSE
\begin{itemize}
\item {} 
\sphinxAtStartPar
Office: 1021 HW Arndt Building

\item {} 
\sphinxAtStartPar
Email: \sphinxhref{mailto:fedor.iskhakov@anu.edu.au}{fedor.iskhakov@anu.edu.au}

\item {} 
\sphinxAtStartPar
Web: \sphinxhref{https://fedor.iskh.me}{fedor.iskh.me}

\item {} 
\sphinxAtStartPar
Contact hours: Thursday 9:30\sphinxhyphen{}11:30

\end{itemize}


\section{Timetable}
\label{\detokenize{01.introduction:timetable}}
\sphinxAtStartPar
\sphinxstylestrong{Face\sphinxhyphen{}to\sphinxhyphen{}face:}
\begin{itemize}
\item {} 
\sphinxAtStartPar
Lectures: Thursday 15:30 — 17:30

\item {} 
\sphinxAtStartPar
Location: \sphinxstylestrong{DNF Dunbar Lecture Theatre, Physics Bldg 39A}

\end{itemize}

\sphinxAtStartPar
\sphinxstylestrong{Online:}
\begin{itemize}
\item {} 
\sphinxAtStartPar
Echo\sphinxhyphen{}360 recordings on Wattle

\item {} 
\sphinxAtStartPar
All notes and materials on \sphinxstylestrong{\sphinxhref{http://optim.iskh.me}{optim.iskh.me}}

\end{itemize}

\sphinxAtStartPar
Face\sphinxhyphen{}to\sphinxhyphen{}face is strictly preferred


\section{Course web pages}
\label{\detokenize{01.introduction:course-web-pages}}\begin{itemize}
\item {} 
\sphinxAtStartPar
\sphinxhref{https://wattlecourses.anu.edu.au/course/view.php?id=41102}{Wattle}
Schedule, announcements, teaching team contacts, recordings, assignment, grades

\item {} 
\sphinxAtStartPar
\sphinxhref{https://optim.iskh.me}{Online notes}
Lecture notes, slides, assignment tasks

\item {} 
\sphinxAtStartPar
Lecture slides should appear online the previous day before the lecture

\item {} 
\sphinxAtStartPar
Details on assessment including the exam instructions will appear on Wattle

\end{itemize}


\section{Tutorials}
\label{\detokenize{01.introduction:tutorials}}\begin{itemize}
\item {} 
\sphinxAtStartPar
Enrollments open on \sphinxstyleemphasis{Wattle}

\end{itemize}

\sphinxAtStartPar
Tutorial questions
\begin{itemize}
\item {} 
\sphinxAtStartPar
posted on the course website

\item {} 
\sphinxAtStartPar
not assessed, help you learn and prepare

\end{itemize}

\sphinxAtStartPar
Tutorials start on week 2


\section{Tutors}
\label{\detokenize{01.introduction:tutors}}
\sphinxAtStartPar
\sphinxstylestrong{Wending Liu}
\begin{itemize}
\item {} 
\sphinxAtStartPar
Email: \sphinxhref{mailto:Wending.Liu@anu.edu.au}{Wending.Liu@anu.edu.au}

\item {} 
\sphinxAtStartPar
Room: Room 2084, Copland Bld (24)

\item {} 
\sphinxAtStartPar
Office hours: \sphinxstylestrong{Friday 1pm\sphinxhyphen{}3pm}

\end{itemize}

\sphinxAtStartPar
\sphinxstylestrong{Chien Yeh}
\begin{itemize}
\item {} 
\sphinxAtStartPar
Email: \sphinxhref{mailto:Chien.Yeh@anu.edu.au}{Chien.Yeh@anu.edu.au}

\item {} 
\sphinxAtStartPar
Room: Room 2106, Copland Bld (24)

\item {} 
\sphinxAtStartPar
Office hours: \sphinxstylestrong{Monday 2pm\sphinxhyphen{}4pm}

\end{itemize}


\section{Prerequisites}
\label{\detokenize{01.introduction:prerequisites}}
\sphinxAtStartPar
See \sphinxhref{https://programsandcourses.anu.edu.au/2023/course/ECON2125\#terms}{Course overview} and
\sphinxhref{https://programsandcourses.anu.edu.au/course/ECON2125/Second\%20Semester/6275}{Class summary}

\sphinxAtStartPar
What you actually need to know:
\begin{itemize}
\item {} 
\sphinxAtStartPar
basic algebra

\item {} 
\sphinxAtStartPar
basic calculus

\item {} 
\sphinxAtStartPar
some idea of what a matrix is, etc.

\end{itemize}

\sphinxAtStartPar
≈ content of EMET1001/EMET7001 math course


\section{Focus?}
\label{\detokenize{01.introduction:focus}}
\sphinxAtStartPar
\sphinxstyleemphasis{Q:} Is this optimization or a general math\sphinxhyphen{}econ course?

\sphinxAtStartPar
\sphinxstyleemphasis{A:} A general course on mathematical modeling for economics and financial economics. Optimization will be an important and recurring theme.


\section{Assessment}
\label{\detokenize{01.introduction:assessment}}\begin{itemize}
\item {} 
\sphinxAtStartPar
3 timed open book tests (15\% each)

\item {} 
\sphinxAtStartPar
Final exam (55\%)

\end{itemize}

\sphinxAtStartPar
The three tests spread out through the semester will check the knowledge of the immediately preceding material. The final closed book in\sphinxhyphen{}person exam will cover the entire course.


\section{Questions}
\label{\detokenize{01.introduction:questions}}\begin{enumerate}
\sphinxsetlistlabels{\arabic}{enumi}{enumii}{}{.}%
\item {} 
\sphinxAtStartPar
Administrative questions: RSE admin

\end{enumerate}
\begin{itemize}
\item {} 
\sphinxAtStartPar
\sphinxstylestrong{Bronwyn Cammack} Senior School Administrator

\item {} 
\sphinxAtStartPar
Email: \sphinxhref{mailto:enquiries.rse@anu.edu.au}{enquiries.rse@anu.edu.au}

\item {} 
\sphinxAtStartPar
“I can not register for the tutorial group”

\end{itemize}
\begin{enumerate}
\sphinxsetlistlabels{\arabic}{enumi}{enumii}{}{.}%
\setcounter{enumi}{1}
\item {} 
\sphinxAtStartPar
Content related questions: please, refer to the tutors

\end{enumerate}
\begin{itemize}
\item {} 
\sphinxAtStartPar
“I don’t understand why this function is convex”

\end{itemize}
\begin{enumerate}
\sphinxsetlistlabels{\arabic}{enumi}{enumii}{}{.}%
\setcounter{enumi}{2}
\item {} 
\sphinxAtStartPar
Other questions: to Fedor

\end{enumerate}
\begin{itemize}
\item {} 
\sphinxAtStartPar
“I’m working hard but still can not keep up”

\item {} 
\sphinxAtStartPar
“Can I please have extra assignment for more practice”

\end{itemize}


\section{Attendance}
\label{\detokenize{01.introduction:attendance}}\begin{itemize}
\item {} 
\sphinxAtStartPar
Please, \sphinxstylestrong{do not} use email for \sphinxstyleemphasis{instructional} questions\textbackslash{}Instead make use of the office hours

\item {} 
\sphinxAtStartPar
Attendance of tutorials is \sphinxstyleemphasis{very highly} recommended\\
You will make your life much easier this way

\item {} 
\sphinxAtStartPar
Attendance of lectures is \sphinxstyleemphasis{highly} recommended\\
But not mandatory

\end{itemize}


\section{Comments for lectures notes/slides}
\label{\detokenize{01.introduction:comments-for-lectures-notes-slides}}\begin{itemize}
\item {} 
\sphinxAtStartPar
Cover exactly what you are required to know

\item {} 
\sphinxAtStartPar
Code inserts are the exception, they are not assessable

\end{itemize}

\sphinxAtStartPar
In particular, you need to know:
\begin{itemize}
\item {} 
\sphinxAtStartPar
The definitions from the notes

\item {} 
\sphinxAtStartPar
The facts from the notes

\item {} 
\sphinxAtStartPar
How to apply facts and definitions

\end{itemize}

\sphinxAtStartPar
If a concept in not in the lecture notes, it is not assessable


\section{Definitions and facts}
\label{\detokenize{01.introduction:definitions-and-facts}}
\sphinxAtStartPar
The lectures notes/slides are full of definitions and facts.

\begin{sphinxadmonition}{note}{Definition}

\sphinxAtStartPar
Functions \(f: \mathbb{R} \rightarrow \mathbb{R}\) is called \sphinxstyleemphasis{continuous at} \(x\) if, for any sequence \(\{x_n\}\) converging to \(x\), we have \(f(x_n) \rightarrow f(x)\).
\end{sphinxadmonition}

\sphinxAtStartPar
Possible exam question: “Show  that if functions \(f\) and \(g\) are continuous at \(x\), so is \(f+g\).”

\sphinxAtStartPar
You should start the answer with the definition of continuity:

\sphinxAtStartPar
“Let \(\{x_n\}\) be any sequence converging to \(x\). We need to show that \(f(x_n) + g(x_n) \rightarrow f(x) + g(x)\). To see this, note that …”


\section{Facts}
\label{\detokenize{01.introduction:facts}}
\sphinxAtStartPar
In the lecture notes/slides you will often see

\begin{sphinxadmonition}{note}{Fact}

\sphinxAtStartPar
The only \(N\)\sphinxhyphen{}dimensional subset of \(\mathbb{R}^N\) is \(\mathbb{R}^N\).
\end{sphinxadmonition}

\sphinxAtStartPar
This means either:
\begin{itemize}
\item {} 
\sphinxAtStartPar
theorem

\item {} 
\sphinxAtStartPar
proposition

\item {} 
\sphinxAtStartPar
lemma

\item {} 
\sphinxAtStartPar
true statement

\end{itemize}

\sphinxAtStartPar
All well known results. You need to remember them, have some intuition for, and be able to apply.


\section{Note on Assessments}
\label{\detokenize{01.introduction:note-on-assessments}}
\sphinxAtStartPar
Assessable = definitions and facts + last year level math + a few simple steps of logic

\sphinxAtStartPar
Exams and tests will award:
\begin{itemize}
\item {} 
\sphinxAtStartPar
Hard work

\item {} 
\sphinxAtStartPar
Deeper understanding of the concepts

\end{itemize}

\sphinxAtStartPar
In each question there will be a \sphinxstyleemphasis{easy} path to the solution


\section{Reading materials}
\label{\detokenize{01.introduction:reading-materials}}
\sphinxAtStartPar
\sphinxstylestrong{Primary reference:} lecture slides

\sphinxAtStartPar
\sphinxstylestrong{Books:}

\noindent\sphinxincludegraphics[height=100\sphinxpxdimen]{{simon_blume}.png}

\noindent\sphinxincludegraphics[height=100\sphinxpxdimen]{{sundaram}.png}

\noindent\sphinxincludegraphics[height=100\sphinxpxdimen]{{stachurski}.png}
\begin{itemize}
\item {} 
\sphinxAtStartPar
“Mathematics for Economists” (1994) by Simon, C. and L. Blume

\item {} 
\sphinxAtStartPar
“A First Course in Optimization” (1996) Theory by Rangarajan Sundaram

\item {} 
\sphinxAtStartPar
“A Primer in Econometric Theory” (2016) by John Stachurski

\end{itemize}

\sphinxAtStartPar
Readings are supplementary but will provide a more detailed explanation with additional examples.
\begin{itemize}
\item {} 
\sphinxAtStartPar
Each lecture will reference book chapters

\end{itemize}


\section{Key points for the administrative part}
\label{\detokenize{01.introduction:key-points-for-the-administrative-part}}\begin{itemize}
\item {} 
\sphinxAtStartPar
Tutorials start next week, \sphinxstylestrong{please register before the next lecture}

\item {} 
\sphinxAtStartPar
Course content = what’s in lecture notes/slides

\item {} 
\sphinxAtStartPar
Lecture slides are available online and will be updated throughout the semester

\item {} 
\sphinxAtStartPar
Optimization is a recurring theme but not the only topic

\end{itemize}


\section{What you will learn in the course}
\label{\detokenize{01.introduction:what-you-will-learn-in-the-course}}\begin{itemize}
\item {} 
\sphinxAtStartPar
The lecture plan is on the course website \sphinxhref{https://optim.iskh.me}{optim.iskh.me} and \sphinxhref{https://programsandcourses.anu.edu.au/course/ECON2125/Second\%20Semester/6275}{Class summary}

\item {} 
\sphinxAtStartPar
See the list of topics on the left

\end{itemize}

\sphinxAtStartPar
Essentially:
\begin{enumerate}
\sphinxsetlistlabels{\arabic}{enumi}{enumii}{}{.}%
\item {} 
\sphinxAtStartPar
\sphinxstylestrong{Mathematical foundations}

\end{enumerate}
\begin{itemize}
\item {} 
\sphinxAtStartPar
elements of analysis

\item {} 
\sphinxAtStartPar
elements of linear algebra

\item {} 
\sphinxAtStartPar
elements of probability

\end{itemize}
\begin{enumerate}
\sphinxsetlistlabels{\arabic}{enumi}{enumii}{}{.}%
\setcounter{enumi}{1}
\item {} 
\sphinxAtStartPar
\sphinxstylestrong{Optimization theory}

\end{enumerate}
\begin{itemize}
\item {} 
\sphinxAtStartPar
when solution exists

\item {} 
\sphinxAtStartPar
unconstrained optimization

\item {} 
\sphinxAtStartPar
optimization with equality constraints

\item {} 
\sphinxAtStartPar
optimization with inequality constraints

\end{itemize}
\begin{enumerate}
\sphinxsetlistlabels{\arabic}{enumi}{enumii}{}{.}%
\setcounter{enumi}{2}
\item {} 
\sphinxAtStartPar
\sphinxstylestrong{Further topics}

\end{enumerate}
\begin{itemize}
\item {} 
\sphinxAtStartPar
Parameterized optimization problems

\item {} 
\sphinxAtStartPar
Optimization in dynamics

\end{itemize}


\section{Further material and self\sphinxhyphen{}learning}
\label{\detokenize{01.introduction:further-material-and-self-learning}}\begin{itemize}
\item {} 
\sphinxAtStartPar
Each lecture will suggest some material for further reading and learning

\item {} 
\sphinxAtStartPar
Today: \sphinxstylestrong{The Wason Selection Task} logical problem

\item {} 
\sphinxAtStartPar
Mathematics relies on rules of logic

\item {} 
\sphinxAtStartPar
Yet, for human brain applying mathematical logic may be difficult, and dependent on the domain

\end{itemize}

\sphinxAtStartPar
Please, watch the video and try to solve the puzzle yourself
\sphinxhref{https://youtu.be/iR97LBgpsl8}{youtu.be/iR97LBgpsl8}

\sphinxstepscope


\chapter{Lecture 1: Univariate and bivariate optimization}
\label{\detokenize{02.optimization_intro:lecture-1-univariate-and-bivariate-optimization}}\label{\detokenize{02.optimization_intro::doc}}
\sphinxAtStartPar
\sphinxstylestrong{ECON2125/6012 Lecture 2}
Fedor Iskhakov


\section{Announcements \& Reminders}
\label{\detokenize{02.optimization_intro:announcements-reminders}}\begin{itemize}
\item {} 
\sphinxAtStartPar
\sphinxstylestrong{Tutorials start tomorrow (Aug 4)}

\item {} 
\sphinxAtStartPar
Register for tutorials on \sphinxhref{https://wattlecourses.anu.edu.au/course/view.php?id=41102}{Wattle} if you have not done so already

\item {} 
\sphinxAtStartPar
Office hours of the tutors are updated:
\begin{itemize}
\item {} 
\sphinxAtStartPar
\sphinxstylestrong{Wending Liu}
\begin{itemize}
\item {} 
\sphinxAtStartPar
Email: \sphinxhref{mailto:Wending.Liu@anu.edu.au}{Wending.Liu@anu.edu.au}

\item {} 
\sphinxAtStartPar
Room: Room 2084, Copland Bld (24) (\sphinxstyleemphasis{updated!})

\item {} 
\sphinxAtStartPar
Office hours: \sphinxstylestrong{Friday 1pm\sphinxhyphen{}3pm}

\end{itemize}

\item {} 
\sphinxAtStartPar
\sphinxstylestrong{Chien Yeh}
\begin{itemize}
\item {} 
\sphinxAtStartPar
Email: \sphinxhref{mailto:Chien.Yeh@anu.edu.au}{Chien.Yeh@anu.edu.au}

\item {} 
\sphinxAtStartPar
Room: Room 2106, Copland Bld (24)

\item {} 
\sphinxAtStartPar
Office hours: \sphinxstylestrong{Monday 2pm\sphinxhyphen{}4pm}

\end{itemize}

\end{itemize}

\item {} 
\sphinxAtStartPar
Reminder on how to ask questions:
\begin{enumerate}
\sphinxsetlistlabels{\arabic}{enumi}{enumii}{}{.}%
\item {} 
\sphinxAtStartPar
Administrative: RSE admin

\item {} 
\sphinxAtStartPar
Content/understanding: tutors

\item {} 
\sphinxAtStartPar
Other: to Fedor

\end{enumerate}

\end{itemize}


\section{Plan for this lecture}
\label{\detokenize{02.optimization_intro:plan-for-this-lecture}}\begin{enumerate}
\sphinxsetlistlabels{\arabic}{enumi}{enumii}{}{.}%
\item {} 
\sphinxAtStartPar
Motivation (math vs. computing)

\item {} 
\sphinxAtStartPar
Univariate optimization

\item {} 
\sphinxAtStartPar
Working with bivariate functions

\item {} 
\sphinxAtStartPar
Bivariate optimization

\end{enumerate}

\sphinxAtStartPar
\sphinxstylestrong{Supplementary reading:}
\begin{itemize}
\item {} 
\sphinxAtStartPar
Simon \& Blume: part 1 (revision)

\item {} 
\sphinxAtStartPar
Sundaram: sections 1.1, 1.4, chapter 2, chapter 4

\end{itemize}


\section{Computing}
\label{\detokenize{02.optimization_intro:computing}}
\sphinxAtStartPar
The \sphinxstyleemphasis{classic} way we do mathematics is pencil and paper
\begin{quote}

\sphinxAtStartPar
In 1944, Hans Bethe solved following problem \sphinxstyleemphasis{by hand}:\\
Will detonating an atom bomb ignite the atmosphere and
thereby destroy life on earth?\\
\sphinxhref{https://inis.iaea.org/search/search.aspx?orig\_q=RN:25070731}{source}
\end{quote}

\sphinxAtStartPar
These days we rarely calculate with actual numbers

\sphinxAtStartPar
Almost all calculations are done on computers

\begin{sphinxadmonition}{note}{Example: numerical integration}
\begin{equation*}
\begin{split}
\frac{1}{\sqrt{2\pi}} 
\int_{-2}^2 
\exp\left\{ - \frac{x^2}{2} \right\} dx
\end{split}
\end{equation*}\end{sphinxadmonition}

\begin{sphinxuseclass}{cell}\begin{sphinxVerbatimInput}

\begin{sphinxuseclass}{cell_input}
\begin{sphinxVerbatim}[commandchars=\\\{\}]
\PYG{k+kn}{from} \PYG{n+nn}{scipy}\PYG{n+nn}{.}\PYG{n+nn}{stats} \PYG{k+kn}{import} \PYG{n}{norm}
\PYG{k+kn}{from} \PYG{n+nn}{scipy}\PYG{n+nn}{.}\PYG{n+nn}{integrate} \PYG{k+kn}{import} \PYG{n}{quad}
\PYG{n}{phi} \PYG{o}{=} \PYG{n}{norm}\PYG{p}{(}\PYG{p}{)}
\PYG{n}{value}\PYG{p}{,} \PYG{n}{error} \PYG{o}{=} \PYG{n}{quad}\PYG{p}{(}\PYG{n}{phi}\PYG{o}{.}\PYG{n}{pdf}\PYG{p}{,} \PYG{o}{\PYGZhy{}}\PYG{l+m+mi}{2}\PYG{p}{,} \PYG{l+m+mi}{2}\PYG{p}{)}
\PYG{n+nb}{print}\PYG{p}{(}\PYG{l+s+s1}{\PYGZsq{}}\PYG{l+s+s1}{Integral value =}\PYG{l+s+s1}{\PYGZsq{}}\PYG{p}{,}\PYG{n}{value}\PYG{p}{)}
\end{sphinxVerbatim}

\end{sphinxuseclass}\end{sphinxVerbatimInput}
\begin{sphinxVerbatimOutput}

\begin{sphinxuseclass}{cell_output}
\begin{sphinxVerbatim}[commandchars=\\\{\}]
Integral value = 0.9544997361036417
\end{sphinxVerbatim}

\end{sphinxuseclass}\end{sphinxVerbatimOutput}

\end{sphinxuseclass}
\begin{sphinxadmonition}{note}{Example: Numerical optimization}
\begin{equation*}
\begin{split}
f(x) = - \exp
\left\{-\frac{(x - 5.0)^4}{1.5} \right\}
\rightarrow \min
\end{split}
\end{equation*}\end{sphinxadmonition}

\begin{sphinxuseclass}{cell}\begin{sphinxVerbatimInput}

\begin{sphinxuseclass}{cell_input}
\begin{sphinxVerbatim}[commandchars=\\\{\}]
\PYG{k+kn}{from} \PYG{n+nn}{scipy}\PYG{n+nn}{.}\PYG{n+nn}{optimize} \PYG{k+kn}{import} \PYG{n}{fminbound}
\PYG{k+kn}{import} \PYG{n+nn}{numpy} \PYG{k}{as} \PYG{n+nn}{np}
\PYG{n}{f} \PYG{o}{=} \PYG{k}{lambda} \PYG{n}{x}\PYG{p}{:} \PYG{o}{\PYGZhy{}}\PYG{n}{np}\PYG{o}{.}\PYG{n}{exp}\PYG{p}{(}\PYG{o}{\PYGZhy{}}\PYG{p}{(}\PYG{n}{x} \PYG{o}{\PYGZhy{}} \PYG{l+m+mf}{5.0}\PYG{p}{)}\PYG{o}{*}\PYG{o}{*}\PYG{l+m+mi}{4} \PYG{o}{/} \PYG{l+m+mf}{1.5}\PYG{p}{)}
\PYG{n}{res} \PYG{o}{=} \PYG{n}{fminbound}\PYG{p}{(}\PYG{n}{f}\PYG{p}{,} \PYG{o}{\PYGZhy{}}\PYG{l+m+mi}{10}\PYG{p}{,} \PYG{l+m+mi}{10}\PYG{p}{)}  \PYG{c+c1}{\PYGZsh{} find approx solution}
\PYG{n+nb}{print}\PYG{p}{(}\PYG{l+s+s1}{\PYGZsq{}}\PYG{l+s+s1}{Minimum value is attained approximately at}\PYG{l+s+s1}{\PYGZsq{}}\PYG{p}{,} \PYG{n}{res}\PYG{p}{)}
\end{sphinxVerbatim}

\end{sphinxuseclass}\end{sphinxVerbatimInput}
\begin{sphinxVerbatimOutput}

\begin{sphinxuseclass}{cell_output}
\begin{sphinxVerbatim}[commandchars=\\\{\}]
Minimum value is attained approximately at 4.999941901210501
\end{sphinxVerbatim}

\end{sphinxuseclass}\end{sphinxVerbatimOutput}

\end{sphinxuseclass}
\begin{sphinxadmonition}{note}{Example: Visualization}

\sphinxAtStartPar
What does this function look like?
\begin{equation*}
\begin{split}
f(x, y) = \frac{\cos(x^2 + y^2)}{1 + x^2 + y^2}
\end{split}
\end{equation*}\end{sphinxadmonition}

\begin{sphinxuseclass}{cell}\begin{sphinxVerbatimInput}

\begin{sphinxuseclass}{cell_input}
\begin{sphinxVerbatim}[commandchars=\\\{\}]
\PYG{k+kn}{import} \PYG{n+nn}{matplotlib}\PYG{n+nn}{.}\PYG{n+nn}{pyplot} \PYG{k}{as} \PYG{n+nn}{plt}
\PYG{k+kn}{from} \PYG{n+nn}{mpl\PYGZus{}toolkits}\PYG{n+nn}{.}\PYG{n+nn}{mplot3d}\PYG{n+nn}{.}\PYG{n+nn}{axes3d} \PYG{k+kn}{import} \PYG{n}{Axes3D}
\PYG{k+kn}{import} \PYG{n+nn}{numpy} \PYG{k}{as} \PYG{n+nn}{np}
\PYG{k+kn}{from} \PYG{n+nn}{matplotlib} \PYG{k+kn}{import} \PYG{n}{cm}
\PYG{n}{f} \PYG{o}{=} \PYG{k}{lambda} \PYG{n}{x}\PYG{p}{,} \PYG{n}{y}\PYG{p}{:} \PYG{n}{np}\PYG{o}{.}\PYG{n}{cos}\PYG{p}{(}\PYG{n}{x}\PYG{o}{*}\PYG{o}{*}\PYG{l+m+mi}{2} \PYG{o}{+} \PYG{n}{y}\PYG{o}{*}\PYG{o}{*}\PYG{l+m+mi}{2}\PYG{p}{)} \PYG{o}{/} \PYG{p}{(}\PYG{l+m+mi}{1} \PYG{o}{+} \PYG{n}{x}\PYG{o}{*}\PYG{o}{*}\PYG{l+m+mi}{2} \PYG{o}{+} \PYG{n}{y}\PYG{o}{*}\PYG{o}{*}\PYG{l+m+mi}{2}\PYG{p}{)}
\PYG{n}{xgrid} \PYG{o}{=} \PYG{n}{np}\PYG{o}{.}\PYG{n}{linspace}\PYG{p}{(}\PYG{o}{\PYGZhy{}}\PYG{l+m+mi}{3}\PYG{p}{,} \PYG{l+m+mi}{3}\PYG{p}{,} \PYG{l+m+mi}{50}\PYG{p}{)}
\PYG{n}{ygrid} \PYG{o}{=} \PYG{n}{xgrid}
\PYG{n}{x}\PYG{p}{,} \PYG{n}{y} \PYG{o}{=} \PYG{n}{np}\PYG{o}{.}\PYG{n}{meshgrid}\PYG{p}{(}\PYG{n}{xgrid}\PYG{p}{,} \PYG{n}{ygrid}\PYG{p}{)}
\PYG{n}{fig} \PYG{o}{=} \PYG{n}{plt}\PYG{o}{.}\PYG{n}{figure}\PYG{p}{(}\PYG{n}{figsize}\PYG{o}{=}\PYG{p}{(}\PYG{l+m+mi}{8}\PYG{p}{,} \PYG{l+m+mi}{6}\PYG{p}{)}\PYG{p}{)}
\PYG{n}{ax} \PYG{o}{=} \PYG{n}{fig}\PYG{o}{.}\PYG{n}{add\PYGZus{}subplot}\PYG{p}{(}\PYG{l+m+mi}{111}\PYG{p}{,} \PYG{n}{projection}\PYG{o}{=}\PYG{l+s+s1}{\PYGZsq{}}\PYG{l+s+s1}{3d}\PYG{l+s+s1}{\PYGZsq{}}\PYG{p}{)}
\PYG{n}{ax}\PYG{o}{.}\PYG{n}{plot\PYGZus{}surface}\PYG{p}{(}\PYG{n}{x}\PYG{p}{,}
                \PYG{n}{y}\PYG{p}{,}
                \PYG{n}{f}\PYG{p}{(}\PYG{n}{x}\PYG{p}{,} \PYG{n}{y}\PYG{p}{)}\PYG{p}{,}
                \PYG{n}{rstride}\PYG{o}{=}\PYG{l+m+mi}{2}\PYG{p}{,} \PYG{n}{cstride}\PYG{o}{=}\PYG{l+m+mi}{2}\PYG{p}{,}
                \PYG{n}{cmap}\PYG{o}{=}\PYG{n}{cm}\PYG{o}{.}\PYG{n}{jet}\PYG{p}{,}
                \PYG{n}{alpha}\PYG{o}{=}\PYG{l+m+mf}{0.7}\PYG{p}{,}
                \PYG{n}{linewidth}\PYG{o}{=}\PYG{l+m+mf}{0.25}\PYG{p}{)}
\PYG{n}{ax}\PYG{o}{.}\PYG{n}{set\PYGZus{}zlim}\PYG{p}{(}\PYG{o}{\PYGZhy{}}\PYG{l+m+mf}{0.5}\PYG{p}{,} \PYG{l+m+mf}{1.0}\PYG{p}{)}
\PYG{n}{plt}\PYG{o}{.}\PYG{n}{show}\PYG{p}{(}\PYG{p}{)}
\end{sphinxVerbatim}

\end{sphinxuseclass}\end{sphinxVerbatimInput}
\begin{sphinxVerbatimOutput}

\begin{sphinxuseclass}{cell_output}
\noindent{\hspace*{\fill}\sphinxincludegraphics[width=0.800\linewidth]{{7e667ededfed90e7cfdd5e04185b793e6f2f689cfecb6d4850b534d03f325af8}.png}\hspace*{\fill}}

\end{sphinxuseclass}\end{sphinxVerbatimOutput}

\end{sphinxuseclass}
\begin{sphinxadmonition}{note}{Example: Symbolic calculations }

\sphinxAtStartPar
Differentiate \(f(x) = (1 + 2x)^5\).\\
Forgotten how?  No problems, just ask a computer for \sphinxstyleemphasis{symbolic} derivative
\end{sphinxadmonition}

\begin{sphinxuseclass}{cell}\begin{sphinxVerbatimInput}

\begin{sphinxuseclass}{cell_input}
\begin{sphinxVerbatim}[commandchars=\\\{\}]
\PYG{k+kn}{import} \PYG{n+nn}{sympy} \PYG{k}{as} \PYG{n+nn}{sp}
\PYG{n}{x} \PYG{o}{=} \PYG{n}{sp}\PYG{o}{.}\PYG{n}{Symbol}\PYG{p}{(}\PYG{l+s+s1}{\PYGZsq{}}\PYG{l+s+s1}{x}\PYG{l+s+s1}{\PYGZsq{}}\PYG{p}{)}
\PYG{n}{fx} \PYG{o}{=} \PYG{p}{(}\PYG{l+m+mi}{1} \PYG{o}{+} \PYG{l+m+mi}{2} \PYG{o}{*} \PYG{n}{x}\PYG{p}{)}\PYG{o}{*}\PYG{o}{*}\PYG{l+m+mi}{5}
\PYG{n+nb}{print}\PYG{p}{(}\PYG{l+s+s2}{\PYGZdq{}}\PYG{l+s+s2}{Derivative of}\PYG{l+s+s2}{\PYGZdq{}}\PYG{p}{,}\PYG{n}{fx}\PYG{p}{,}\PYG{l+s+s2}{\PYGZdq{}}\PYG{l+s+s2}{is}\PYG{l+s+s2}{\PYGZdq{}}\PYG{p}{,}\PYG{n}{fx}\PYG{o}{.}\PYG{n}{diff}\PYG{p}{(}\PYG{n}{x}\PYG{p}{)}\PYG{p}{)}
\end{sphinxVerbatim}

\end{sphinxuseclass}\end{sphinxVerbatimInput}
\begin{sphinxVerbatimOutput}

\begin{sphinxuseclass}{cell_output}
\begin{sphinxVerbatim}[commandchars=\\\{\}]
Derivative of (2*x + 1)**5 is 10*(2*x + 1)**4
\end{sphinxVerbatim}

\end{sphinxuseclass}\end{sphinxVerbatimOutput}

\end{sphinxuseclass}
\sphinxAtStartPar
So if computers can do our maths for us, why learn maths?

\sphinxAtStartPar
The difficulty is
\begin{itemize}
\item {} 
\sphinxAtStartPar
giving them the right inputs and instructions

\item {} 
\sphinxAtStartPar
interpreting what comes out

\end{itemize}

\sphinxAtStartPar
The skills we need are
\begin{itemize}
\item {} 
\sphinxAtStartPar
Understanding of fundamental concepts

\item {} 
\sphinxAtStartPar
Sound deductive reasoning

\end{itemize}

\sphinxAtStartPar
\sphinxstylestrong{These are the focus of the course}


\subsection{Computer Code in the Lectures}
\label{\detokenize{02.optimization_intro:computer-code-in-the-lectures}}
\sphinxAtStartPar
While computation is not a formal part of the course\\
there will be little bits of code in the lectures to illustrate the kinds of things we can do.
\begin{itemize}
\item {} 
\sphinxAtStartPar
All the code will be written in the Python programming language

\item {} 
\sphinxAtStartPar
It is not assessable

\end{itemize}

\sphinxAtStartPar
You might find value in actually running the code shown in lectures\\
If you want to do so please refer to \sphinxstylestrong{linked GitHub repository} in \sphinxhref{https://optim.iskh.me}{optim.iskh.me}


\section{Univariate Optimization}
\label{\detokenize{02.optimization_intro:univariate-optimization}}
\sphinxAtStartPar
Let \(f \colon [a, b] \to \mathbb{R}\) be a differentiable (smooth) function
\begin{itemize}
\item {} 
\sphinxAtStartPar
\([a, b]\) is all \(x\) with \(a \leq x \leq b\)

\item {} 
\sphinxAtStartPar
\(\mathbb{R}\) is “all numbers”

\item {} 
\sphinxAtStartPar
\(f\) takes \(x \in [a, b]\) and returns number \(f(x)\)

\item {} 
\sphinxAtStartPar
derivative \(f'(x)\) exists for all \(x\) with \(a < x < b\)

\end{itemize}

\begin{sphinxadmonition}{note}{Definition}

\sphinxAtStartPar
A  point \(x^* \in [a, b]\) is called a
\begin{itemize}
\item {} 
\sphinxAtStartPar
\sphinxstyleemphasis{\sphinxstylestrong{maximizer}} of \(f\) on \([a, b]\) if \(f(x^*) \geq f(x)\) for all \(x \in [a,b]\)

\item {} 
\sphinxAtStartPar
\sphinxstyleemphasis{\sphinxstylestrong{minimizer}} of \(f\) on \([a, b]\) if \(f(x^*) \leq f(x)\) for all \(x \in [a,b]\)

\end{itemize}
\end{sphinxadmonition}

\begin{sphinxadmonition}{note}{Example}

\sphinxAtStartPar
Let
\begin{itemize}
\item {} 
\sphinxAtStartPar
\(f(x) = -(x-4)^2 + 10\)

\item {} 
\sphinxAtStartPar
\(a = 2\) and \(b=8\)

\end{itemize}

\sphinxAtStartPar
Then
\begin{itemize}
\item {} 
\sphinxAtStartPar
\(x^* = 4\) is a maximizer of \(f\) on \([2, 8]\)

\item {} 
\sphinxAtStartPar
\(x^{**} = 8\) is a minimizer of \(f\) on \([2, 8]\)

\end{itemize}

\begin{figure}[H]
\centering
\capstart

\noindent\sphinxincludegraphics[width=0.800\linewidth]{{d3801ce557e3b38cc0132fb26c560870763137bf39b611b69e7302075a6fce3d}.png}
\caption{Maximizer on \([a, b] = [2, 8]\) is \(x^* = 4\)}\label{\detokenize{02.optimization_intro:id1}}\end{figure}

\begin{figure}[H]
\centering
\capstart

\noindent\sphinxincludegraphics[width=0.800\linewidth]{{2751088387d35324fface5d67b117cffec47bfada7f51cc420592f5fccfadc8c}.png}
\caption{Minimizer on \([a, b] = [2, 8]\) is \(x^{**} = 8\)}\label{\detokenize{02.optimization_intro:id2}}\end{figure}
\end{sphinxadmonition}

\sphinxAtStartPar
The set of maximizers/minimizers can be
\begin{itemize}
\item {} 
\sphinxAtStartPar
empty

\item {} 
\sphinxAtStartPar
a singleton (contains one element)

\item {} 
\sphinxAtStartPar
infinite (contains infinitely many elements)

\end{itemize}

\begin{sphinxadmonition}{note}{Example: infinite maximizers}

\sphinxAtStartPar
\(f \colon [0, 1] \to \mathbb{R}\) defined by \(f(x) =1\)\\
has infinitely many maximizers and minimizers on \([0, 1]\)
\end{sphinxadmonition}

\begin{sphinxadmonition}{note}{Example: no maximizers}

\sphinxAtStartPar
The following function has no maximizers on \([0, 2]\)
\begin{equation*}
\begin{split}
f(x) = 
\begin{cases}
x^2 &  \text{ if } x < 1
\\
1/2 &  \text{ otherwise}
\end{cases}
\end{split}
\end{equation*}
\begin{figure}[H]
\centering
\capstart

\noindent\sphinxincludegraphics[width=0.800\linewidth]{{d3625b746fc2a0f0c7b4f776552674214beed900f88c11ec0d8fb77970831006}.png}
\caption{No maximizer on \([0, 2]\)}\label{\detokenize{02.optimization_intro:id3}}\end{figure}
\end{sphinxadmonition}

\begin{sphinxadmonition}{note}{Definition}

\sphinxAtStartPar
Point  \(x\) is called \sphinxstyleemphasis{\sphinxstylestrong{interior}} to \([a, b]\) if \(a < x < b\)
\end{sphinxadmonition}

\sphinxAtStartPar
The set of all interior points is written \((a, b)\)

\sphinxAtStartPar
We refer to \(x^* \in [a, b]\) as
\begin{itemize}
\item {} 
\sphinxAtStartPar
\sphinxstyleemphasis{\sphinxstylestrong{interior maximizer}} if both a maximizer and interior

\item {} 
\sphinxAtStartPar
\sphinxstyleemphasis{\sphinxstylestrong{interior minimizer}} if both a minimizer and interior

\end{itemize}


\section{Finding optima}
\label{\detokenize{02.optimization_intro:finding-optima}}
\begin{sphinxadmonition}{note}{Definition}

\sphinxAtStartPar
A \sphinxstyleemphasis{\sphinxstylestrong{stationary point}} of \(f\) on \([a, b]\) is an interior point \(x\) with \(f'(x) = 0\)
\end{sphinxadmonition}

\begin{figure}[htbp]
\centering
\capstart

\noindent\sphinxincludegraphics[width=0.800\linewidth]{{stationary}.png}
\caption{Both \(x^*\) and \(x^{**}\) are stationary}\label{\detokenize{02.optimization_intro:id4}}\end{figure}

\begin{sphinxadmonition}{note}{Fact}

\sphinxAtStartPar
If \(f\) is differentiable and \(x^*\) is either an interior minimizer
or an interior maximizer of \(f\) on \([a, b]\), then \(x^*\) is stationary
\end{sphinxadmonition}

\sphinxAtStartPar
Sketch of proof, for maximizers:
\begin{equation*}
\begin{split}
f'(x^*) = \, \lim_{h \to 0} \, \frac{f(x^* + h) - f(x^*)}{h}
\qquad \text{(by def.)}
\end{split}
\end{equation*}\begin{equation*}
\begin{split}
\Rightarrow f(x^* + h) \approx f(x^*) + f'(x^*) h 
\qquad \text{for small } h 
\end{split}
\end{equation*}
\sphinxAtStartPar
If \(f'(x^*) \ne 0\) then exists small \(h\) such that \(f(x^* + h) > f(x^*)\)

\sphinxAtStartPar
Hence interior maximizers must be stationary — otherwise we can do better

\sphinxAtStartPar
\(\Rightarrow\) any interior maximizer stationary\\
\(\Rightarrow\) set of interior maximizers \(\subset\) set of stationary points\\
\(\Rightarrow\) maximizers \(\subset\) stationary points \(\cup \{a\} \cup \{b\}\)

\sphinxAtStartPar
Usage:
\begin{enumerate}
\sphinxsetlistlabels{\arabic}{enumi}{enumii}{}{.}%
\item {} 
\sphinxAtStartPar
Locate stationary points

\item {} 
\sphinxAtStartPar
Evaluate \(y = f(x)\) for each stationary \(x\) and for \(a\), \(b\)

\item {} 
\sphinxAtStartPar
Pick point giving largest \(y\) value

\end{enumerate}

\sphinxAtStartPar
Minimization: same idea

\begin{sphinxadmonition}{note}{Example}

\sphinxAtStartPar
Let’s solve
\begin{equation*}
\begin{split} 
\max_{-2 \leq x \leq 5} f(x) 
\quad \text{where} \quad
f(x) = x^3 - 6x^2 + 4x + 8
\end{split}
\end{equation*}
\sphinxAtStartPar
Steps
\begin{itemize}
\item {} 
\sphinxAtStartPar
Differentiate to get \(f'(x) = 3x^2 - 12x + 4\)

\item {} 
\sphinxAtStartPar
Solve \(3x^2 - 12x + 4 = 0\) to get stationary \(x\)

\item {} 
\sphinxAtStartPar
Discard any stationary points outside \([-2, 5]\)

\item {} 
\sphinxAtStartPar
Eval \(f\) at remaining points plus end points \(-2\) and \(5\)

\item {} 
\sphinxAtStartPar
Pick point giving largest value

\end{itemize}
\end{sphinxadmonition}

\begin{sphinxuseclass}{cell}\begin{sphinxVerbatimInput}

\begin{sphinxuseclass}{cell_input}
\begin{sphinxVerbatim}[commandchars=\\\{\}]
\PYG{k+kn}{from} \PYG{n+nn}{sympy} \PYG{k+kn}{import} \PYG{o}{*}
\PYG{n}{x} \PYG{o}{=} \PYG{n}{Symbol}\PYG{p}{(}\PYG{l+s+s1}{\PYGZsq{}}\PYG{l+s+s1}{x}\PYG{l+s+s1}{\PYGZsq{}}\PYG{p}{)}
\PYG{n}{points} \PYG{o}{=} \PYG{p}{[}\PYG{o}{\PYGZhy{}}\PYG{l+m+mi}{2}\PYG{p}{,} \PYG{l+m+mi}{5}\PYG{p}{]}
\PYG{n}{f} \PYG{o}{=} \PYG{n}{x}\PYG{o}{*}\PYG{o}{*}\PYG{l+m+mi}{3} \PYG{o}{\PYGZhy{}} \PYG{l+m+mi}{6}\PYG{o}{*}\PYG{n}{x}\PYG{o}{*}\PYG{o}{*}\PYG{l+m+mi}{2} \PYG{o}{+} \PYG{l+m+mi}{4}\PYG{o}{*}\PYG{n}{x} \PYG{o}{+} \PYG{l+m+mi}{8}
\PYG{n}{fp} \PYG{o}{=} \PYG{n}{diff}\PYG{p}{(}\PYG{n}{f}\PYG{p}{,} \PYG{n}{x}\PYG{p}{)}
\PYG{n}{spoints} \PYG{o}{=} \PYG{n}{solve}\PYG{p}{(}\PYG{n}{fp}\PYG{p}{,} \PYG{n}{x}\PYG{p}{)}
\PYG{n}{points}\PYG{o}{.}\PYG{n}{extend}\PYG{p}{(}\PYG{n}{spoints}\PYG{p}{)}
\PYG{n}{v} \PYG{o}{=} \PYG{p}{[}\PYG{n}{f}\PYG{o}{.}\PYG{n}{subs}\PYG{p}{(}\PYG{n}{x}\PYG{p}{,} \PYG{n}{c}\PYG{p}{)}\PYG{o}{.}\PYG{n}{evalf}\PYG{p}{(}\PYG{p}{)} \PYG{k}{for} \PYG{n}{c} \PYG{o+ow}{in} \PYG{n}{points}\PYG{p}{]}
\PYG{n}{maximizer} \PYG{o}{=} \PYG{n}{points}\PYG{p}{[}\PYG{n}{v}\PYG{o}{.}\PYG{n}{index}\PYG{p}{(}\PYG{n+nb}{max}\PYG{p}{(}\PYG{n}{v}\PYG{p}{)}\PYG{p}{)}\PYG{p}{]}
\PYG{n+nb}{print}\PYG{p}{(}\PYG{l+s+s2}{\PYGZdq{}}\PYG{l+s+s2}{Maximizer =}\PYG{l+s+s2}{\PYGZdq{}}\PYG{p}{,} \PYG{n+nb}{str}\PYG{p}{(}\PYG{n}{maximizer}\PYG{p}{)}\PYG{p}{,}\PYG{l+s+s1}{\PYGZsq{}}\PYG{l+s+s1}{=}\PYG{l+s+s1}{\PYGZsq{}}\PYG{p}{,}\PYG{n}{maximizer}\PYG{o}{.}\PYG{n}{evalf}\PYG{p}{(}\PYG{p}{)}\PYG{p}{)}
\end{sphinxVerbatim}

\end{sphinxuseclass}\end{sphinxVerbatimInput}
\begin{sphinxVerbatimOutput}

\begin{sphinxuseclass}{cell_output}
\begin{sphinxVerbatim}[commandchars=\\\{\}]
Maximizer = 2 \PYGZhy{} 2*sqrt(6)/3 = 0.367006838144548
\end{sphinxVerbatim}

\end{sphinxuseclass}\end{sphinxVerbatimOutput}

\end{sphinxuseclass}
\begin{sphinxuseclass}{cell}
\begin{sphinxuseclass}{tag_hide-input}\begin{sphinxVerbatimOutput}

\begin{sphinxuseclass}{cell_output}
\noindent{\hspace*{\fill}\sphinxincludegraphics[width=0.800\linewidth]{{83f903ccee15fdbe59aaa8f3cad4ad66e1d89c202520eeced59c3947756e680b}.png}\hspace*{\fill}}

\end{sphinxuseclass}\end{sphinxVerbatimOutput}

\end{sphinxuseclass}
\end{sphinxuseclass}

\section{Shape Conditions and Sufficiency}
\label{\detokenize{02.optimization_intro:shape-conditions-and-sufficiency}}
\sphinxAtStartPar
When is \(f'(x^*) = 0\) sufficient for \(x^*\) to be a maximizer?

\sphinxAtStartPar
One answer: When \(f\) is concave

\begin{sphinxuseclass}{cell}
\begin{sphinxuseclass}{tag_hide-input}\begin{sphinxVerbatimOutput}

\begin{sphinxuseclass}{cell_output}
\noindent{\hspace*{\fill}\sphinxincludegraphics[width=0.800\linewidth]{{0643e7397394ed7d767b2f7c26a238493b608e42274e3969c758f0283e74a9ca}.png}\hspace*{\fill}}

\end{sphinxuseclass}\end{sphinxVerbatimOutput}

\end{sphinxuseclass}
\end{sphinxuseclass}
\sphinxAtStartPar
(Full definition deferred)

\begin{sphinxadmonition}{note}{Sufficient conditions for \sphinxstyleemphasis{concavity} in one dimension}

\sphinxAtStartPar
Let \(f \colon [a, b] \to \mathbb{R}\)
\begin{itemize}
\item {} 
\sphinxAtStartPar
If \(f''(x) \leq 0\) for all \(x \in (a, b)\) then \(f\) is concave on \((a, b)\)

\item {} 
\sphinxAtStartPar
If \(f''(x) < 0\) for all \(x \in (a, b)\) then \(f\) is \sphinxstylestrong{strictly} concave on \((a, b)\)

\end{itemize}
\end{sphinxadmonition}

\begin{sphinxadmonition}{note}{Example}
\begin{itemize}
\item {} 
\sphinxAtStartPar
\(f(x) = a + b x\) is concave on \(\mathbb{R}\) but not strictly

\item {} 
\sphinxAtStartPar
\(f(x) = \log(x)\) is strictly concave on \((0, \infty)\)

\end{itemize}
\end{sphinxadmonition}

\sphinxAtStartPar
When is \(f'(x^*) = 0\) sufficient for \(x^*\) to be a minimizer?

\sphinxAtStartPar
One answer: When \(f\) is convex

\begin{sphinxuseclass}{cell}
\begin{sphinxuseclass}{tag_hide-input}\begin{sphinxVerbatimOutput}

\begin{sphinxuseclass}{cell_output}
\noindent{\hspace*{\fill}\sphinxincludegraphics[width=0.800\linewidth]{{0a2934f61b0beb5e674a16ddd69a364120e5944f75d0dcc17ed81f13a31d073f}.png}\hspace*{\fill}}

\end{sphinxuseclass}\end{sphinxVerbatimOutput}

\end{sphinxuseclass}
\end{sphinxuseclass}
\sphinxAtStartPar
(Full definition deferred)

\begin{sphinxadmonition}{note}{Sufficient conditions for \sphinxstyleemphasis{convexity} in one dimension}

\sphinxAtStartPar
Let \(f \colon [a, b] \to \mathbb{R}\)
\begin{itemize}
\item {} 
\sphinxAtStartPar
If \(f''(x) \geq 0\) for all \(x \in (a, b)\) then \(f\) is convex on \((a, b)\)

\item {} 
\sphinxAtStartPar
If \(f''(x) > 0\) for all \(x \in (a, b)\) then \(f\) is \sphinxstylestrong{strictly}
convex on \((a, b)\)

\end{itemize}
\end{sphinxadmonition}

\begin{sphinxadmonition}{note}{Example}
\begin{itemize}
\item {} 
\sphinxAtStartPar
\(f(x) = a + b x\) is convex on \(\mathbb{R}\) but not strictly

\item {} 
\sphinxAtStartPar
\(f(x) = x^2\) is strictly convex on \(\mathbb{R}\)

\end{itemize}
\end{sphinxadmonition}


\subsection{Sufficiency and uniqueness with shape conditions}
\label{\detokenize{02.optimization_intro:sufficiency-and-uniqueness-with-shape-conditions}}
\begin{sphinxadmonition}{note}{Fact}

\sphinxAtStartPar
For maximizers:
\begin{itemize}
\item {} 
\sphinxAtStartPar
If \(f \colon [a,b] \to \mathbb{R}\) is concave and \(x^* \in (a, b)\) is
stationary then \(x^*\) is a maximizer

\item {} 
\sphinxAtStartPar
If, in addition, \(f\) is strictly concave, then \(x^*\) is the
unique maximizer

\end{itemize}
\end{sphinxadmonition}

\begin{sphinxadmonition}{note}{Fact}

\sphinxAtStartPar
For minimizers:
\begin{itemize}
\item {} 
\sphinxAtStartPar
If \(f \colon [a,b] \to \mathbb{R}\) is convex and \(x^* \in (a, b)\) is
stationary then \(x^*\) is a minimizer

\item {} 
\sphinxAtStartPar
If, in addition, \(f\) is strictly convex, then \(x^*\) is the
unique minimizer

\end{itemize}
\end{sphinxadmonition}

\begin{sphinxadmonition}{note}{Example}

\sphinxAtStartPar
A price taking firm faces output price \(p > 0\), input price \(w >0\)

\sphinxAtStartPar
Maximize profits with respect to input \(\ell\)
\begin{equation*}
\begin{split}
\max_{\ell \ge 0} \pi(\ell) = p f(\ell) - w \ell,
\end{split}
\end{equation*}
\sphinxAtStartPar
where the production technology is given by
\begin{equation*}
\begin{split}
f(\ell) = \ell^{\alpha}, 0 < \alpha < 1.
\end{split}
\end{equation*}\end{sphinxadmonition}

\sphinxAtStartPar
Evidently
\begin{equation*}
\begin{split}
\pi'(\ell) = \alpha p \ell^{\alpha - 1} - w,
\end{split}
\end{equation*}
\sphinxAtStartPar
so unique stationary point is
\begin{equation*}
\begin{split}
\ell^* = (\alpha p/w)^{1/(1 - \alpha)}
\end{split}
\end{equation*}
\sphinxAtStartPar
Moreover,
\begin{equation*}
\begin{split}
\pi''(\ell) = \alpha (\alpha - 1) p \ell^{\alpha - 2} < 0
\end{split}
\end{equation*}
\sphinxAtStartPar
for all \(\ell \ge 0\) so \(\ell^*\) is unique maximizer.

\begin{figure}[htbp]
\centering
\capstart

\noindent\sphinxincludegraphics[width=0.800\linewidth]{{4d0128849c9b4d98dc41b942a10906327f01cfb09802cdae5a0db626b035973a}.png}
\caption{Profit maximization with \(p=2\), \(w=1\), \(\alpha=0.6\), \(\ell^*=\)\DUrole{output,text_plain}{1.5774}}\label{\detokenize{02.optimization_intro:id5}}\end{figure}


\section{Functions of two variables}
\label{\detokenize{02.optimization_intro:functions-of-two-variables}}
\sphinxAtStartPar
Let’s have a look at some functions of two variables
\begin{itemize}
\item {} 
\sphinxAtStartPar
How to visualize them

\item {} 
\sphinxAtStartPar
Slope, contours, etc.

\end{itemize}

\begin{sphinxadmonition}{note}{Example: Cobb\sphinxhyphen{}Douglas production function}

\sphinxAtStartPar
Consider production function
\begin{equation*}
\begin{split}
f(k, \ell) = k^{\alpha} \ell^{\beta}\\
\alpha \ge 0, \, \beta \ge 0, \, \alpha + \beta < 1
\end{split}
\end{equation*}
\sphinxAtStartPar
Let’s graph it in two dimensions.
\end{sphinxadmonition}

\begin{figure}[htbp]
\centering
\capstart

\noindent\sphinxincludegraphics[width=0.800\linewidth]{{prod2d}.png}
\caption{Production function with \(\alpha=0.4\), \(\beta=0.5\) (a)}\label{\detokenize{02.optimization_intro:id6}}\end{figure}

\begin{figure}[htbp]
\centering
\capstart

\noindent\sphinxincludegraphics[width=0.800\linewidth]{{prod2d_1}.png}
\caption{Production function with \(\alpha=0.4\), \(\beta=0.5\) (b)}\label{\detokenize{02.optimization_intro:id7}}\end{figure}

\begin{figure}[htbp]
\centering
\capstart

\noindent\sphinxincludegraphics[width=0.800\linewidth]{{prod2d_2}.png}
\caption{Production function with \(\alpha=0.4\), \(\beta=0.5\) (c)}\label{\detokenize{02.optimization_intro:id8}}\end{figure}

\sphinxAtStartPar
Like many 3D plots it’s hard to get a good understanding

\sphinxAtStartPar
Let’s try again with contours plus heat map

\begin{figure}[htbp]
\centering
\capstart

\noindent\sphinxincludegraphics[width=0.800\linewidth]{{prodcontour}.png}
\caption{Production function with \(\alpha=0.4\), \(\beta=0.5\), contours}\label{\detokenize{02.optimization_intro:id9}}\end{figure}

\sphinxAtStartPar
In this context the contour lines are called \sphinxstyleemphasis{\sphinxstylestrong{isoquants}}

\sphinxAtStartPar
Can you see how \(\alpha < \beta\) shows up in the slope of the contours?

\sphinxAtStartPar
We can drop the colours to see the numbers more clearly

\begin{figure}[htbp]
\centering
\capstart

\noindent\sphinxincludegraphics[width=0.800\linewidth]{{prodcontour2}.png}
\caption{Production function with \(\alpha=0.4\), \(\beta=0.5\)}\label{\detokenize{02.optimization_intro:id10}}\end{figure}

\begin{sphinxadmonition}{note}{Example: log\sphinxhyphen{}utility}

\sphinxAtStartPar
Let \(u(x_1,x_2)\) be “utility” gained from \(x_1\) units of good 1 and \(x_2\) units of good 2

\sphinxAtStartPar
We take
\begin{equation*}
\begin{split}
u(x_1, x_2) = \alpha \log(x_1) + \beta \log(x_2)
\end{split}
\end{equation*}
\sphinxAtStartPar
where
\begin{itemize}
\item {} 
\sphinxAtStartPar
\(\alpha\) and \(\beta\) are parameters

\item {} 
\sphinxAtStartPar
we assume \(\alpha>0, \, \beta > 0\)

\item {} 
\sphinxAtStartPar
The log functions mean “diminishing returns” in each good

\end{itemize}
\end{sphinxadmonition}

\begin{figure}[htbp]
\centering
\capstart

\noindent\sphinxincludegraphics[width=0.800\linewidth]{{log_util}.png}
\caption{Log utility with \(\alpha=0.4\), \(\beta=0.5\)}\label{\detokenize{02.optimization_intro:id11}}\end{figure}

\sphinxAtStartPar
Let’s look at the contour lines

\sphinxAtStartPar
For utility functions, contour lines called \sphinxstyleemphasis{\sphinxstylestrong{indifference curves}}

\begin{figure}[htbp]
\centering
\capstart

\noindent\sphinxincludegraphics[width=0.800\linewidth]{{log_util_contour}.png}
\caption{Indifference curves of log utility with \(\alpha=0.4\), \(\beta=0.5\)}\label{\detokenize{02.optimization_intro:id12}}\end{figure}

\begin{sphinxadmonition}{note}{Example: quasi\sphinxhyphen{}linear utility}
\begin{equation*}
\begin{split}
u(x_1, x_2) = x_1 + \log(x_2)
\end{split}
\end{equation*}\begin{itemize}
\item {} 
\sphinxAtStartPar
Called quasi\sphinxhyphen{}linear because linear in good 1

\end{itemize}
\end{sphinxadmonition}

\begin{figure}[htbp]
\centering
\capstart

\noindent\sphinxincludegraphics[width=0.800\linewidth]{{ql_utility}.png}
\caption{Quasi\sphinxhyphen{}linear utility}\label{\detokenize{02.optimization_intro:id13}}\end{figure}

\begin{figure}[htbp]
\centering
\capstart

\noindent\sphinxincludegraphics[width=0.800\linewidth]{{ql_utility_contour}.png}
\caption{Indifference curves of quasi\sphinxhyphen{}linear utility}\label{\detokenize{02.optimization_intro:id14}}\end{figure}

\begin{sphinxadmonition}{note}{Example: quadratic utility}
\begin{equation*}
\begin{split}
u(x_1, x_2) = - (x_1 - b_1)^2 - (x_2 - b_2)^2
\end{split}
\end{equation*}
\sphinxAtStartPar
Here
\begin{itemize}
\item {} 
\sphinxAtStartPar
\(b_1\) is a “satiation” or “bliss” point for \(x_1\)

\item {} 
\sphinxAtStartPar
\(b_2\) is a “satiation” or “bliss” point for \(x_2\)

\end{itemize}
\end{sphinxadmonition}

\sphinxAtStartPar
Dissatisfaction increases with deviations from the bliss points

\begin{figure}[htbp]
\centering
\capstart

\noindent\sphinxincludegraphics[width=0.800\linewidth]{{quad_util}.png}
\caption{Quadratic utility with \(b_1 = 3\) and \(b_2 = 2\)}\label{\detokenize{02.optimization_intro:id15}}\end{figure}

\begin{figure}[htbp]
\centering
\capstart

\noindent\sphinxincludegraphics[width=0.800\linewidth]{{quad_util_contour}.png}
\caption{Indifference curves quadratic utility with \(b_1 = 3\) and \(b_2 = 2\)}\label{\detokenize{02.optimization_intro:id16}}\end{figure}


\section{Bivariate Optimization}
\label{\detokenize{02.optimization_intro:bivariate-optimization}}
\sphinxAtStartPar
Consider \(f \colon I \to \mathbb{R}\) where \(I \subset \mathbb{R}^2\)

\sphinxAtStartPar
The set \(\mathbb{R}^2\) is all \((x_1, x_2)\) pairs

\begin{sphinxadmonition}{note}{Definition}

\sphinxAtStartPar
A point \((x_1^*, x_2^*) \in I\) is called a \sphinxstyleemphasis{\sphinxstylestrong{maximizer}} of \(f\) on \(I\) if
\begin{equation*}
\begin{split}
f(x_1^*, x_2^*) \geq f(x_1, x_2) 
\quad \text{for all} \quad
(x_1, x_2) \in I
\end{split}
\end{equation*}\end{sphinxadmonition}

\begin{sphinxadmonition}{note}{Definition}

\sphinxAtStartPar
A point \((x_1^*, x_2^*) \in I\) is called a \sphinxstyleemphasis{\sphinxstylestrong{minimizer}} of \(f\) on \(I\) if
\begin{equation*}
\begin{split}
f(x_1^*, x_2^*) \leq f(x_1, x_2) 
\quad \text{for all} \quad
(x_1, x_2) \in I
\end{split}
\end{equation*}\end{sphinxadmonition}

\sphinxAtStartPar
When they exist, the partial derivatives at \((x_1, x_2) \in I\) are
\begin{equation*}
\begin{split}
f_1(x_1, x_2) = \frac{\partial}{\partial x_1} f(x_1, x_2)
\\
f_2(x_1, x_2) = \frac{\partial}{\partial x_2} f(x_1, x_2)
\end{split}
\end{equation*}
\begin{sphinxadmonition}{note}{Example}

\sphinxAtStartPar
When \(f(k, \ell) = k^\alpha \ell^\beta\),
\begin{equation*}
\begin{split}
f_1(k, \ell) 
= \frac{\partial}{\partial k} f(k, \ell)
= \frac{\partial}{\partial k} k^\alpha \ell^\beta
= \alpha k^{\alpha-1} \ell^\beta
\end{split}
\end{equation*}\end{sphinxadmonition}

\begin{sphinxadmonition}{note}{Definition}

\sphinxAtStartPar
An interior point \((x_1, x_2) \in I\) is called \sphinxstyleemphasis{\sphinxstylestrong{stationary}} for \(f\) if
\begin{equation*}
\begin{split}
f_1(x_1, x_2) = f_2(x_1, x_2) = 0
\end{split}
\end{equation*}\end{sphinxadmonition}

\begin{sphinxadmonition}{note}{Fact}

\sphinxAtStartPar
Let \(f \colon I \to \mathbb{R}\) be a continuously differentiable function.
If \((x_1^*, x_2^*)\) is either
\begin{itemize}
\item {} 
\sphinxAtStartPar
an interior maximizer of \(f\) on \(I\), or

\item {} 
\sphinxAtStartPar
an interior minimizer of \(f\) on \(I\),

\end{itemize}

\sphinxAtStartPar
then \((x_1^*, x_2^*)\) is a stationary point of \(f\)
\end{sphinxadmonition}

\sphinxAtStartPar
Usage, for maximization:
\begin{enumerate}
\sphinxsetlistlabels{\arabic}{enumi}{enumii}{}{.}%
\item {} 
\sphinxAtStartPar
Compute partials

\item {} 
\sphinxAtStartPar
Set partials to zero to find \(S =\) all stationary points

\item {} 
\sphinxAtStartPar
Evaluate candidates in \(S\) and boundary of \(I\)

\item {} 
\sphinxAtStartPar
Select point \((x^*_1, x_2^*)\) yielding highest value

\end{enumerate}

\begin{sphinxadmonition}{note}{Example}
\begin{equation*}
\begin{split}
f(x_1, x_2) = x_1^2 + 4 x_2^2 \rightarrow \min
\quad \mathrm{s.t.} \quad
x_1 + x_2 \leq 1
\end{split}
\end{equation*}\end{sphinxadmonition}

\sphinxAtStartPar
Setting
\begin{equation*}
\begin{split}
f_1(x_1, x_2) = 2 x_1 = 0 
\quad \text{and} \quad
f_2(x_1, x_2) = 8 x_2 = 0 
\end{split}
\end{equation*}
\sphinxAtStartPar
gives the unique stationary point \((0, 0)\), at which \(f(0, 0) = 0\)

\sphinxAtStartPar
On the boundary we have \(x_1 + x_2 = 1\), so
\begin{equation*}
\begin{split}
f(x_1, x_2) 
= f(x_1, 1 - x_1) 
= x_1^2 + 4 (1 - x_1)^2
\end{split}
\end{equation*}
\sphinxAtStartPar
\sphinxstylestrong{Exercise:} Show right hand side \(> 0\) for any \(x_1\)

\sphinxAtStartPar
Hence minimizer is \((x_1^*, x_2^*) = (0, 0)\)


\subsection{Nasty secrets}
\label{\detokenize{02.optimization_intro:nasty-secrets}}
\sphinxAtStartPar
Solving for \((x_1, x_2)\) such that \(f_1(x_1, x_2) = 0\) and \(f_2(x_1, x_2) = 0\) can be hard
\begin{itemize}
\item {} 
\sphinxAtStartPar
System of nonlinear equations

\item {} 
\sphinxAtStartPar
Might have no analytical solution

\item {} 
\sphinxAtStartPar
Set of solutions can be a continuum

\end{itemize}

\begin{sphinxadmonition}{note}{Example}

\sphinxAtStartPar
(Don’t) try to find all stationary  points of
\begin{equation*}
\begin{split}
f(x_1, x_2) = \frac{\cos(x_1^2 + x_2^2) + x_1^2 + x_1}{2 +
p(-x_1^2) + \sin^2(x_2)}
\end{split}
\end{equation*}\end{sphinxadmonition}

\sphinxAtStartPar
Also:
\begin{itemize}
\item {} 
\sphinxAtStartPar
Boundary is often a continuum, not just two points

\item {} 
\sphinxAtStartPar
Things get even harder in higher dimensions

\end{itemize}

\sphinxAtStartPar
On the other hand:
\begin{itemize}
\item {} 
\sphinxAtStartPar
Most classroom examples are chosen to avoid these problems

\item {} 
\sphinxAtStartPar
Life is still pretty easy if we have concavity / convexity

\item {} 
\sphinxAtStartPar
Clever tricks have been found for certain kinds of problems

\end{itemize}


\section{Second Order Partials}
\label{\detokenize{02.optimization_intro:second-order-partials}}
\sphinxAtStartPar
Let \(f \colon I \to \mathbb{R}\) and, when they exist, denote
\begin{equation*}
\begin{split}
f_{11}(x_1, x_2) 
= \frac{\partial^2}{\partial x_1^2} 
f(x_1, x_2)
\end{split}
\end{equation*}\begin{equation*}
\begin{split}
f_{12}(x_1, x_2) 
= \frac{\partial^2}{\partial x_1 \partial x_2} 
f(x_1, x_2)
\end{split}
\end{equation*}\begin{equation*}
\begin{split}
f_{21}(x_1, x_2) 
= \frac{\partial^2}{\partial x_2 \partial x_1} 
f(x_1, x_2)
\end{split}
\end{equation*}\begin{equation*}
\begin{split}
f_{22}(x_1, x_2) 
= \frac{\partial^2}{\partial x_2^2} 
f(x_1, x_2)
\end{split}
\end{equation*}
\begin{sphinxadmonition}{note}{Example: Cobb\sphinxhyphen{}Douglas technology with linear costs}

\sphinxAtStartPar
If \(\pi(k, \ell) = p k^{\alpha} \ell^{\beta} - w \ell - r k\) then
\begin{equation*}
\begin{split}
\pi_{11}(k, \ell) = p \alpha(\alpha-1) k^{\alpha-2} \ell^{\beta}
\end{split}
\end{equation*}\begin{equation*}
\begin{split}
\pi_{12}(k, \ell) = p \alpha\beta k^{\alpha-1} \ell^{\beta-1}
\end{split}
\end{equation*}\begin{equation*}
\begin{split}
\pi_{21}(k, \ell) = p \alpha\beta k^{\alpha-1} \ell^{\beta-1}
\end{split}
\end{equation*}\begin{equation*}
\begin{split}
\pi_{22}(k, \ell) = p \beta(\beta-1) k^{\alpha} \ell^{\beta-2}
\end{split}
\end{equation*}\end{sphinxadmonition}

\begin{sphinxadmonition}{note}{Fact}

\sphinxAtStartPar
If \(f \colon I \to \mathbb{R}\) is twice continuously differentiable at \((x_1, x_2)\), then
\begin{equation*}
\begin{split}
f_{12}(x_1, x_2) = f_{21}(x_1, x_2)
\end{split}
\end{equation*}\end{sphinxadmonition}

\sphinxAtStartPar
\sphinxstylestrong{Exercise:} Confirm the results in the exercise above.


\section{Shape conditions in 2D}
\label{\detokenize{02.optimization_intro:shape-conditions-in-2d}}
\sphinxAtStartPar
Let \(I\) be an “open” set (only interior points – formalities next week)

\sphinxAtStartPar
Let \(f \colon I \to \mathbb{R}\) be twice continuously differentiable

\sphinxAtStartPar
The function \(f\) is strictly \sphinxstylestrong{concave} on \(I\) if, for any \((x_1, x_2) \in I\)
\begin{enumerate}
\sphinxsetlistlabels{\arabic}{enumi}{enumii}{}{.}%
\item {} 
\sphinxAtStartPar
\(f_{11}(x_1, x_2) < 0\)

\item {} 
\sphinxAtStartPar
\(f_{11}(x_1, x_2) \, f_{22}(x_1, x_2) >  f_{12}(x_1, x_2)^2\)

\end{enumerate}

\sphinxAtStartPar
The function \(f\) is strictly \sphinxstylestrong{convex} on \(I\) if, for any \((x_1, x_2) \in I\)
\begin{enumerate}
\sphinxsetlistlabels{\arabic}{enumi}{enumii}{}{.}%
\item {} 
\sphinxAtStartPar
\(f_{11}(x_1, x_2) > 0\)

\item {} 
\sphinxAtStartPar
\(f_{11}(x_1, x_2) \, f_{22}(x_1, x_2) >  f_{12}(x_1, x_2)^2\)

\end{enumerate}

\sphinxAtStartPar
When is stationarity sufficient?

\begin{sphinxadmonition}{note}{Fact}

\sphinxAtStartPar
If \(f\) is differentiable and strictly concave on \(I\), then any
stationary point of \(f\) is also a unique maximizer of \(f\) on \(I\)
\end{sphinxadmonition}

\begin{sphinxadmonition}{note}{Fact}

\sphinxAtStartPar
If \(f\) is differentiable and strictly convex on \(I\), then any
stationary point of \(f\) is also a unique minimizer of \(f\) on \(I\)
\end{sphinxadmonition}

\begin{figure}[htbp]
\centering
\capstart

\noindent\sphinxincludegraphics[width=0.800\linewidth]{{concave_max}.png}
\caption{Maximizer of a concave function}\label{\detokenize{02.optimization_intro:id17}}\end{figure}

\begin{figure}[htbp]
\centering
\capstart

\noindent\sphinxincludegraphics[width=0.800\linewidth]{{convex_min}.png}
\caption{Minimizer of a convex function}\label{\detokenize{02.optimization_intro:id18}}\end{figure}

\begin{sphinxadmonition}{note}{Example: unconstrained maximization of quadratic utility}
\begin{equation*}
\begin{split}
u(x_1, x_2) = - (x_1 - b_1)^2 - (x_2 - b_2)^2
\rightarrow \max_{x_1, x_2}
\end{split}
\end{equation*}\end{sphinxadmonition}

\sphinxAtStartPar
Intuitively the solution is \(x_1^*=b_1\) and \(x_2^*=b_2\)

\sphinxAtStartPar
Analysis above leads to the same conclusion

\sphinxAtStartPar
First let’s check first order conditions (\sphinxstyleemphasis{F.O.C.})
\begin{equation*}
\begin{split}
\frac{\partial}{\partial x_1}
u(x_1, x_2) = -2 (x_1 - b_1) = 0
\quad \implies \quad
x_1 = b_1
\end{split}
\end{equation*}\begin{equation*}
\begin{split}
\frac{\partial}{\partial x_2}
u(x_1, x_2) = -2 (x_2 - b_2) = 0
\quad \implies \quad
x_2 = b_2
\end{split}
\end{equation*}
\sphinxAtStartPar
How about (strict) concavity?

\sphinxAtStartPar
Sufficient condition is
\begin{enumerate}
\sphinxsetlistlabels{\arabic}{enumi}{enumii}{}{.}%
\item {} 
\sphinxAtStartPar
\(u_{11}(x_1, x_2) < 0\)

\item {} 
\sphinxAtStartPar
\(u_{11}(x_1, x_2)u_{22}(x_1, x_2) > u_{12}(x_1, x_2)^2\)

\end{enumerate}

\sphinxAtStartPar
We have
\begin{itemize}
\item {} 
\sphinxAtStartPar
\(u_{11}(x_1, x_2) = -2\)

\item {} 
\sphinxAtStartPar
\(u_{11}(x_1, x_2)u_{22}(x_1, x_2) = 4 > 0 = u_{12}(x_1, x_2)^2\)

\end{itemize}

\begin{sphinxadmonition}{note}{Example: Profit maximization with two inputs}
\begin{equation*}
\begin{split}
\pi(k, \ell) 
= p k^{\alpha} \ell^{\beta} - w \ell - r k
\rightarrow \max_{k, \ell}
\end{split}
\end{equation*}
\sphinxAtStartPar
where \( \alpha, \beta, p, w\) are all positive and \(\alpha + \beta < 1\)
\end{sphinxadmonition}

\sphinxAtStartPar
Derivatives:
\begin{itemize}
\item {} 
\sphinxAtStartPar
\(\pi_1(k, \ell) = p \alpha k^{\alpha-1} \ell^{\beta} - r\)

\item {} 
\sphinxAtStartPar
\(\pi_2(k, \ell) = p \beta k^{\alpha} \ell^{\beta-1} - w\)

\item {} 
\sphinxAtStartPar
\(\pi_{11}(k, \ell) = p \alpha(\alpha-1) k^{\alpha-2} \ell^{\beta}\)

\item {} 
\sphinxAtStartPar
\(\pi_{22}(k, \ell) = p \beta(\beta-1) k^{\alpha} \ell^{\beta-2}\)

\item {} 
\sphinxAtStartPar
\(\pi_{12}(k, \ell) = p \alpha \beta k^{\alpha-1} \ell^{\beta-1}\)

\end{itemize}

\sphinxAtStartPar
First order conditions: set
\begin{equation*}
\begin{split}
\pi_1(k, \ell) = 0
\\
\pi_2(k, \ell) = 0
\end{split}
\end{equation*}
\sphinxAtStartPar
and solve simultaneously for \(k, \ell\) to get
\begin{equation*}
\begin{split}
k^* =
\left[ 
p (\alpha/r)^{1 - \beta}  (\beta/w)^{\beta}
\right]^{1 / (1 - \alpha - \beta)}
\\
\ell^* =
\left[ 
p (\beta/w)^{1 - \alpha}  (\alpha/r)^{\alpha}
\right]^{1 / (1 - \alpha - \beta)}
\end{split}
\end{equation*}
\sphinxAtStartPar
\sphinxstylestrong{Exercise:} Verify

\sphinxAtStartPar
Now we check second order conditions, hoping for strict concavity

\sphinxAtStartPar
What we need: for any \(k, \ell > 0\)
\begin{enumerate}
\sphinxsetlistlabels{\arabic}{enumi}{enumii}{}{.}%
\item {} 
\sphinxAtStartPar
\(\pi_{11}(k, \ell) < 0\)

\item {} 
\sphinxAtStartPar
\(\pi_{11}(k, \ell) \, \pi_{22}(k, \ell) >  \pi_{12}(k, \ell)^2\)

\end{enumerate}

\sphinxAtStartPar
\sphinxstylestrong{Exercise:} Show both inequalities satisfied when \(\alpha + \beta < 1\)

\begin{figure}[htbp]
\centering
\capstart

\noindent\sphinxincludegraphics[width=0.800\linewidth]{{optprod}.png}
\caption{Profit function when \(p=5\), \(r=w=2\), \(\alpha=0.4\), \(\beta=0.5\)}\label{\detokenize{02.optimization_intro:id19}}\end{figure}

\begin{figure}[htbp]
\centering
\capstart

\noindent\sphinxincludegraphics[width=0.800\linewidth]{{optprod_contour}.png}
\caption{Optimal choice, \(p=5\), \(r=w=2\), \(\alpha=0.4\), \(\beta=0.5\)}\label{\detokenize{02.optimization_intro:id20}}\end{figure}


\section{Exercises and materials for self study}
\label{\detokenize{02.optimization_intro:exercises-and-materials-for-self-study}}
\sphinxAtStartPar
{\hyperref[\detokenize{02.exercises::doc}]{\sphinxcrossref{\DUrole{doc,std,std-doc}{Exercise set A}}}}

\sphinxstepscope


\chapter{Exercise set A}
\label{\detokenize{02.exercises:exercise-set-a}}\label{\detokenize{02.exercises::doc}}
\sphinxAtStartPar
\sphinxstylestrong{General comment on the tutorial exercises}
\begin{itemize}
\item {} 
\sphinxAtStartPar
these questions are not directly assessable and solutions are provided in a separate document

\item {} 
\sphinxAtStartPar
the aim is to help you better understand the material we have covered so far and start to prepare for actual assessments

\item {} 
\sphinxAtStartPar
if you are having no particular problems with this course then
please carry on to the questions below; if, on the other hand, you are having difficulty with the material, then please read on

\item {} 
\sphinxAtStartPar
this is an upper level course on mathematics for economists that pushes you beyond the boundaries of the kind of things we do in high school or first year university

\item {} 
\sphinxAtStartPar
many people find this material hard at first, however, the experience is that anyone who works diligently and consistently can and will do well

\end{itemize}

\sphinxAtStartPar
Here are few tips on getting through and doing well:
\begin{enumerate}
\sphinxsetlistlabels{\arabic}{enumi}{enumii}{}{.}%
\item {} 
\sphinxAtStartPar
It’s hard to “wing” this course, even if you did well at maths in high school. It’s also very hard to follow everything just from the lectures.
It takes practice to do well, just like playing guitar or learning a language. The first resource is the lecture slides, and the more often you read them the more the definitions will stick and the material will gel in your head.

\item {} 
\sphinxAtStartPar
Each time you are having difficulty with a new concept try googling it. Have a look at Wikipedia, find a video on YouTube, or some of the other online resources. They might phrase or explain the concept in a way that fits better with your brain.

\item {} 
\sphinxAtStartPar
Work consistently throughout the semester. Concepts become clearer and more familiar the more times that you go over them—with at least one sleep in between to allow your brain to organize neurons and synapses to store and categorize this new information.

\item {} 
\sphinxAtStartPar
Make use of your tutors. They are very knowledgable and are willing to put in time to help anyone who is genuinely trying (although much less inclined to help those who aren’t).

\item {} 
\sphinxAtStartPar
Send me feedback if you think it’s something I can help with (e.g., more practice questions on a certain topic) or drop in during my office hours to discuss.

\item {} 
\sphinxAtStartPar
Above all, remember that the course material is nontrivial for a reason. Doing straightforward calculations applying well known rules or memorizing “cookbooks” of facts are not particularly useful, mainly because computers are far, far better than humans at these kinds of activities. What is still very useful—probably more than ever—is understanding concepts and how they relate to each other, and building up your ability to digest technical material and think in a logical way.
If you complete this course successfully you will have significantly upgraded your mathematical skills.

\end{enumerate}

\sphinxAtStartPar
Finally, here are some tips on doing proofs:
\begin{enumerate}
\sphinxsetlistlabels{\arabic}{enumi}{enumii}{}{.}%
\item {} 
\sphinxAtStartPar
In many instances there will be an easy way to do things, if you can spot it. A question that seems to require a long calculation will likely have an easy answer if you know the relevant fact.

\item {} 
\sphinxAtStartPar
If you feel stuck, remember that the hardest step is getting started, and for proofs the best place to start is always the relevant definitions. If you are asked to show that the range of a given function is a linear subspace, start by writing down those two definitions. They will tell you more specifically what you need to show. If you’re still stuck, review any facts from the lecture slides related to those definitions. Is there a different way to describe the range of this function? Is there some fact related to linear subspaces that might be helpful?

\item {} 
\sphinxAtStartPar
If you’re still stuck, try flipping the problem around. In the previous example, suppose that the range of the function is not a linear subspace. What would that imply? Can you show that such an outcome is impossible?

\item {} 
\sphinxAtStartPar
Be patient and don’t rush. You’ll get quicker naturally, with practice.

\end{enumerate}


\section{Question A.1}
\label{\detokenize{02.exercises:question-a-1}}
\sphinxAtStartPar
Let \(f \colon [-1, 1] \to \mathbb{R}\) be defined by \(f(x) = 1 - |x|\), where \(|x|\) is the absolute value of \(x\).
\begin{itemize}
\item {} 
\sphinxAtStartPar
Is the point \(x = 0\) a maximizer of \(f\) on \([-1, 1]\)?

\item {} 
\sphinxAtStartPar
Is it a unique maximizer?

\item {} 
\sphinxAtStartPar
Is it an interior maximizer?

\item {} 
\sphinxAtStartPar
Is it stationary?

\end{itemize}


\section{Question A.2}
\label{\detokenize{02.exercises:question-a-2}}
\sphinxAtStartPar
Let \(f \colon \mathbb{R} \to \mathbb{R}\) be defined by \(f(x) = \sin(x)\).
\begin{itemize}
\item {} 
\sphinxAtStartPar
Write down the set of stationary points of this function.

\item {} 
\sphinxAtStartPar
Which of these, if any, are maximizers, and which are minimizers?

\end{itemize}

\begin{sphinxadmonition}{tip}{Tip:}
\sphinxAtStartPar
When we discussed these kinds of problems it was for functions of the form \(f \colon [a, b] \to \mathbb{R}\).  Now the domain is all of \(\mathbb{R}\).
However you can apply the same definitions and use similar reasoning.  Also, feel free to look up and use any helpful facts on trigonometric functions.
\end{sphinxadmonition}


\section{Question A.3: Profit maximization with Cobb\sphinxhyphen{}Douglas production and linear costs}
\label{\detokenize{02.exercises:question-a-3-profit-maximization-with-cobb-douglas-production-and-linear-costs}}
\sphinxAtStartPar
A firm uses capital and labor to produce output.
When it employs \(k\) units of capital and \(\ell\) units of labor, its output is \(A k^{\alpha} \ell^{\beta}\) units, where \(A\) is a positive number, and \(\alpha + \beta < 1\).

\sphinxAtStartPar
The unit price of capital is \(r\), and the unit price of labor is \(w\); both are non\sphinxhyphen{}negative.
The firm would like to maximize the profits taking the price \(p\) of the output as given.

\sphinxAtStartPar
The firm’s chief economist \sphinxstyleemphasis{Bob} presented the following formulation of the firm’s optimization problem to the CEO \sphinxstyleemphasis{Alice}:
\begin{equation*}
\begin{split}
\text{Choose } k, \ell, w \text{ and } r \text{ to maximize } 
p k^{\alpha} \ell^{\beta} - w \ell - r k 
\quad \mathrm{s.t.} \quad
\alpha + \beta < 1
\end{split}
\end{equation*}
\sphinxAtStartPar
Questions:
\begin{enumerate}
\sphinxsetlistlabels{\arabic}{enumi}{enumii}{}{.}%
\item {} 
\sphinxAtStartPar
Is this formulation of the firm’s optimization problem correct?

\end{enumerate}
\begin{itemize}
\item {} 
\sphinxAtStartPar
What part reflects the revenue?

\item {} 
\sphinxAtStartPar
What part reflects the costs?

\item {} 
\sphinxAtStartPar
What are the choice variables?

\item {} 
\sphinxAtStartPar
Are there any constraints to be taken into account?

\end{itemize}
\begin{enumerate}
\sphinxsetlistlabels{\arabic}{enumi}{enumii}{}{.}%
\setcounter{enumi}{1}
\item {} 
\sphinxAtStartPar
Right down the problem after \sphinxstyleemphasis{Alice} have updated the formulation.

\item {} 
\sphinxAtStartPar
Approach the problem as unconstrained maximization, and follow the steps in the lecture to find find all stationary points (solve the FOCs).

\item {} 
\sphinxAtStartPar
Write down second order partial derivatives and verify the shape conditions for the profit function.

\item {} 
\sphinxAtStartPar
What is the optimal strategy for the firm? Is the maximizer unique? Why?

\end{enumerate}


\section{Question A.4*}
\label{\detokenize{02.exercises:question-a-4}}
\begin{sphinxadmonition}{note}{Note:}
\sphinxAtStartPar
Exercises marked with an asterisk \sphinxstylestrong{(*)} are optional and more difficult.
\end{sphinxadmonition}
\begin{enumerate}
\sphinxsetlistlabels{\arabic}{enumi}{enumii}{}{.}%
\item {} 
\sphinxAtStartPar
Find all stationary points of the function
\(f(x, y) = \frac{\cos(x^2 + y^2)}{1 + x^2 + y^2}\).

\item {} 
\sphinxAtStartPar
Find all maximizers and minimizers of this function on \(\mathbb{R}^2\).

\end{enumerate}

\begin{sphinxadmonition}{hint}{Hint:}
\sphinxAtStartPar
Is there a convenient change of variable to convert the problem to a univariate one.
\end{sphinxadmonition}

\sphinxstepscope


\chapter{Solutions for the Exercise set A}
\label{\detokenize{02.exercises_solutions:solutions-for-the-exercise-set-a}}\label{\detokenize{02.exercises_solutions::doc}}

\section{Question A.1}
\label{\detokenize{02.exercises_solutions:question-a-1}}
\sphinxAtStartPar
The point \(x=0\) is indeed a maximizer, since \(f(x) = 1 -|x| \leq 1 = f(0)\) for any \(x \in [-1, 1]\) (\(|x|=0\) if and only if \(x=0\)).
It is also a unique maximizer, since no other point is a maximizer (because \(1 -|x| < 1\) for any other \(x\)).
It is an interior maximizer since \(0\) is not an end point of \([-1, 1]\).
It is not stationary because \(f\) is not differentiable at this point (sketch the graph if you like) and hence cannot satisfy \(f'(x)=0\).


\section{Question A.2}
\label{\detokenize{02.exercises_solutions:question-a-2}}
\sphinxAtStartPar
The set \(S\) of stationary points of \(f\) are the points \(x \in \mathbb{R}\) such
that \(f'(x) = \cos(x) = 0\). By the definition of the cosine function this
is the set
\begin{equation*}
\begin{split}
S := \{ x \in \mathbb{R} : x = \pi/2 + k \pi \text{ for } k \in \mathbb{Z} \}
\end{split}
\end{equation*}
\sphinxAtStartPar
Every point in the domain \(\mathbb{R}\) is interior (i.e, not an end point) and
the function \(f\) is differentiable, so the set of maximizers will be
contained in the set of stationary points. The same is true of the set of
minimizers. From the definition of the sine function, we have
\begin{equation*}
\begin{split}
f(\pi/2 + k \pi) =
\begin{cases}
    1 & \text{ if $k$ is even} \\
    -1 & \text{ if $k$ is odd} \\
\end{cases}
\end{split}
\end{equation*}
\sphinxAtStartPar
Hence the set of maximizers is
\begin{equation*}
\begin{split}
M^* := \{ x \in \mathbb{R} : x = \pi/2 + k \pi \text{ for  \} k \text{ an
even integer}}
\end{split}
\end{equation*}
\sphinxAtStartPar
The set of minimizers is
\begin{equation*}
\begin{split}
M_* := \{ x \in \mathbb{R} : x = \pi/2 + k \pi \text{ for  \} k \text{ an
odd integer}}
\end{split}
\end{equation*}

\section{Question A.3}
\label{\detokenize{02.exercises_solutions:question-a-3}}\begin{enumerate}
\sphinxsetlistlabels{\arabic}{enumi}{enumii}{}{.}%
\item {} 
\sphinxAtStartPar
The formulation is not correct. The revenue (after reincerting constant \(A\)) is \(p A k^{\alpha} \ell^{\beta}\), the costs are \(w \ell + r k\), and the choice variables are \(k\) and \(\ell\) (\(w\) and \(r\) are not chosen by the firm).\\
The constraint \(\alpha + \beta < 1\) is irrelevant for the optimization problem, instead it is a constraint on the parameters for the problem to be well posed.
Relevant constraints on the optimization problem are \(k>0\) and \(\ell>0\), they can be first ignored and checked after we solve the unconstrained version of the problem.

\item {} 
\sphinxAtStartPar
The correct formulation is (\(A, p, \alpha, \beta, w, r\) are parameters and should be fixed/found out before the firm solves the optimization problem)
\begin{equation*}
\begin{split}
   p A k^{\alpha} \ell^{\beta} - w \ell - r k
   \rightarrow \max_{k, \ell}\\
   \quad \mathrm{s.t.} \quad
   k \ge 0, \ell \ge 0
   \end{split}
\end{equation*}
\item {} 
\sphinxAtStartPar
See lecture notes

\item {} 
\sphinxAtStartPar
See lecture notes

\item {} 
\sphinxAtStartPar
Optimal strategy \(k^*, \ell^*\) are given in the lecture notes. The maximizer is unique because the objective function is strictly concave when \(\alpha+\beta < 1\).
Proof:
We check second order conditions for strict concavity.

\sphinxAtStartPar
What we need: for any \(k, \ell > 0\)
\begin{enumerate}
\sphinxsetlistlabels{\arabic}{enumii}{enumiii}{}{.}%
\item {} 
\sphinxAtStartPar
\(\pi_{11}(k, \ell) < 0\)

\item {} 
\sphinxAtStartPar
\(\pi_{11}(k, \ell) \, \pi_{22}(k, \ell) >  \pi_{12}(k, \ell)^2\)roof: \sphinxstyleemphasis{coming next}

\end{enumerate}

\sphinxAtStartPar
The second order derivatives are
\begin{equation*}
\begin{split}
   \pi_{11}(k,\ell) = (\alpha-1)\alpha pA k^{\alpha-2} \ell^\beta \\
   \pi_{22}(k,\ell) = (\beta-1)\beta pA k^{\alpha} \ell^{\beta-2}\\
   \pi_{12}(k, \ell) = \alpha \beta pA k^{\alpha-1} \ell^{\beta-1}.
   \end{split}
\end{equation*}
\sphinxAtStartPar
Since \(\alpha+\beta<1\) and \(\alpha, \beta \geq 0\), we have \(\alpha-1<0\), which implies \(\pi_{11}(k,\ell)<0\) for all \(k, \ell >0\).\\
Moreover, the second order differentials imply
\begin{equation*}
\begin{split}
   \pi_{11}(k, \ell) \, \pi_{22}(k, \ell) = (\alpha-1)(\beta-1)\alpha \beta p^2 A^2 k^{2\alpha-2} \ell^{2\beta-2}\\
   (\pi_{22}(k, \ell))^2 = \alpha^2 \beta^2 p^2 A^2 k^{2\alpha-2} \ell^{2\beta-2}.
   \end{split}
\end{equation*}
\sphinxAtStartPar
Assuming that all parameters and variables are positive.
Then, we obtain \(\pi_{11}(k, \ell) \, \pi_{22}(k, \ell) >  \pi_{12}(k, \ell)^2\) if and only if \((\alpha-1)(\beta-1) > \alpha \beta\) if and only if \(1 > \alpha + \beta\).

\end{enumerate}


\section{Question A.4}
\label{\detokenize{02.exercises_solutions:question-a-4}}
\sphinxAtStartPar
The graph of \(f(x,y)\) is

\begin{sphinxuseclass}{cell}\begin{sphinxVerbatimInput}

\begin{sphinxuseclass}{cell_input}
\begin{sphinxVerbatim}[commandchars=\\\{\}]
\PYG{k+kn}{import} \PYG{n+nn}{matplotlib}\PYG{n+nn}{.}\PYG{n+nn}{pyplot} \PYG{k}{as} \PYG{n+nn}{plt}
\PYG{k+kn}{from} \PYG{n+nn}{mpl\PYGZus{}toolkits}\PYG{n+nn}{.}\PYG{n+nn}{mplot3d}\PYG{n+nn}{.}\PYG{n+nn}{axes3d} \PYG{k+kn}{import} \PYG{n}{Axes3D}
\PYG{k+kn}{import} \PYG{n+nn}{numpy} \PYG{k}{as} \PYG{n+nn}{np}
\PYG{k+kn}{import} \PYG{n+nn}{scipy} \PYG{k}{as} \PYG{n+nn}{sp}
\PYG{k+kn}{from} \PYG{n+nn}{matplotlib} \PYG{k+kn}{import} \PYG{n}{cm}
\PYG{n}{f} \PYG{o}{=} \PYG{k}{lambda} \PYG{n}{x}\PYG{p}{,} \PYG{n}{y}\PYG{p}{:} \PYG{n}{np}\PYG{o}{.}\PYG{n}{cos}\PYG{p}{(}\PYG{n}{x}\PYG{o}{*}\PYG{o}{*}\PYG{l+m+mi}{2} \PYG{o}{+} \PYG{n}{y}\PYG{o}{*}\PYG{o}{*}\PYG{l+m+mi}{2}\PYG{p}{)} \PYG{o}{/} \PYG{p}{(}\PYG{l+m+mi}{1} \PYG{o}{+} \PYG{n}{x}\PYG{o}{*}\PYG{o}{*}\PYG{l+m+mi}{2} \PYG{o}{+} \PYG{n}{y}\PYG{o}{*}\PYG{o}{*}\PYG{l+m+mi}{2}\PYG{p}{)}
\PYG{n}{xgrid} \PYG{o}{=} \PYG{n}{np}\PYG{o}{.}\PYG{n}{linspace}\PYG{p}{(}\PYG{o}{\PYGZhy{}}\PYG{l+m+mi}{3}\PYG{p}{,} \PYG{l+m+mi}{3}\PYG{p}{,} \PYG{l+m+mi}{50}\PYG{p}{)}
\PYG{n}{ygrid} \PYG{o}{=} \PYG{n}{xgrid}
\PYG{n}{x}\PYG{p}{,} \PYG{n}{y} \PYG{o}{=} \PYG{n}{np}\PYG{o}{.}\PYG{n}{meshgrid}\PYG{p}{(}\PYG{n}{xgrid}\PYG{p}{,} \PYG{n}{ygrid}\PYG{p}{)}
\PYG{n}{fig} \PYG{o}{=} \PYG{n}{plt}\PYG{o}{.}\PYG{n}{figure}\PYG{p}{(}\PYG{n}{figsize}\PYG{o}{=}\PYG{p}{(}\PYG{l+m+mi}{8}\PYG{p}{,} \PYG{l+m+mi}{6}\PYG{p}{)}\PYG{p}{)}
\PYG{n}{ax} \PYG{o}{=} \PYG{n}{fig}\PYG{o}{.}\PYG{n}{add\PYGZus{}subplot}\PYG{p}{(}\PYG{l+m+mi}{111}\PYG{p}{,} \PYG{n}{projection}\PYG{o}{=}\PYG{l+s+s1}{\PYGZsq{}}\PYG{l+s+s1}{3d}\PYG{l+s+s1}{\PYGZsq{}}\PYG{p}{)}
\PYG{n}{ax}\PYG{o}{.}\PYG{n}{plot\PYGZus{}surface}\PYG{p}{(}\PYG{n}{x}\PYG{p}{,}
                \PYG{n}{y}\PYG{p}{,}
                \PYG{n}{f}\PYG{p}{(}\PYG{n}{x}\PYG{p}{,} \PYG{n}{y}\PYG{p}{)}\PYG{p}{,}
                \PYG{n}{rstride}\PYG{o}{=}\PYG{l+m+mi}{2}\PYG{p}{,} \PYG{n}{cstride}\PYG{o}{=}\PYG{l+m+mi}{2}\PYG{p}{,}
                \PYG{n}{cmap}\PYG{o}{=}\PYG{n}{cm}\PYG{o}{.}\PYG{n}{jet}\PYG{p}{,}
                \PYG{n}{alpha}\PYG{o}{=}\PYG{l+m+mf}{0.7}\PYG{p}{,}
                \PYG{n}{linewidth}\PYG{o}{=}\PYG{l+m+mf}{0.25}\PYG{p}{)}
\PYG{n}{ax}\PYG{o}{.}\PYG{n}{set\PYGZus{}zlim}\PYG{p}{(}\PYG{o}{\PYGZhy{}}\PYG{l+m+mf}{0.5}\PYG{p}{,} \PYG{l+m+mf}{1.0}\PYG{p}{)}
\PYG{n}{plt}\PYG{o}{.}\PYG{n}{show}\PYG{p}{(}\PYG{p}{)}
\end{sphinxVerbatim}

\end{sphinxuseclass}\end{sphinxVerbatimInput}
\begin{sphinxVerbatimOutput}

\begin{sphinxuseclass}{cell_output}
\noindent\sphinxincludegraphics{{7e667ededfed90e7cfdd5e04185b793e6f2f689cfecb6d4850b534d03f325af8}.png}

\end{sphinxuseclass}\end{sphinxVerbatimOutput}

\end{sphinxuseclass}
\sphinxAtStartPar
Let \(t = x^2+y^2 \geq 0\).
The function becomes \$\(f(x,y) = \frac{\cos(x^2+y^2)}{1 + x^2+y^2} = \frac{\cos(t)}{1 + t} =: g(t)  \quad (t \geq 0) .\)\(
First note that since \)t \textbackslash{}geq 0\( and \)\textbackslash{}cos(t) \textbackslash{}leq 1\(, we have \)g(t)\textbackslash{}leq 1\( and \)g(0)=1\(.
Hence, \)t=0\( is a maximizer for \)g\(, or \)(x,y)=(0,0)\( is the maximizer for \)f\(.
It is a unique maximizer, since if \)g(t) < 1\( for \)t >0\$.

\sphinxAtStartPar
Next, we find the stationary points of \(f\) by finding the stationary points of \(g\).
The FOC is
\begin{equation*}
\begin{split}g'(t) = \frac{-\sin(t)(1+t) - \cos(t)}{(1+t)^2} = 0.\end{split}
\end{equation*}
\sphinxAtStartPar
Since \((1+t)^2>0\), it must be
\begin{equation*}
\begin{split}-\sin(t)(1+t) - \cos(t) = 0 ⇔ t\sin(t) + \sin(t) + \cos(t)=0.\end{split}
\end{equation*}
\sphinxAtStartPar
The numerical solutions for the smallest stationary point \(t_m\) such that \(\cos(t_m)<0\) are

\begin{sphinxuseclass}{cell}\begin{sphinxVerbatimInput}

\begin{sphinxuseclass}{cell_input}
\begin{sphinxVerbatim}[commandchars=\\\{\}]
\PYG{n}{g} \PYG{o}{=} \PYG{k}{lambda} \PYG{n}{t}\PYG{p}{:} \PYG{n}{np}\PYG{o}{.}\PYG{n}{cos}\PYG{p}{(}\PYG{n}{t}\PYG{p}{)} \PYG{o}{/} \PYG{p}{(}\PYG{l+m+mi}{1}\PYG{o}{+}\PYG{n}{t}\PYG{p}{)}
\PYG{n}{gp} \PYG{o}{=} \PYG{k}{lambda} \PYG{n}{t}\PYG{p}{:} \PYG{n}{t} \PYG{o}{*} \PYG{n}{np}\PYG{o}{.}\PYG{n}{sin}\PYG{p}{(}\PYG{n}{t}\PYG{p}{)} \PYG{o}{+} \PYG{n}{np}\PYG{o}{.}\PYG{n}{sin}\PYG{p}{(}\PYG{n}{t}\PYG{p}{)} \PYG{o}{+} \PYG{n}{np}\PYG{o}{.}\PYG{n}{cos}\PYG{p}{(}\PYG{n}{t}\PYG{p}{)}
\PYG{n}{xgrid} \PYG{o}{=} \PYG{n}{np}\PYG{o}{.}\PYG{n}{linspace}\PYG{p}{(}\PYG{l+m+mi}{0}\PYG{p}{,} \PYG{l+m+mi}{10}\PYG{p}{,} \PYG{l+m+mi}{200}\PYG{p}{)}
\PYG{n}{fig} \PYG{o}{=} \PYG{n}{plt}\PYG{o}{.}\PYG{n}{figure}\PYG{p}{(}\PYG{n}{figsize}\PYG{o}{=}\PYG{p}{(}\PYG{l+m+mi}{8}\PYG{p}{,} \PYG{l+m+mi}{6}\PYG{p}{)}\PYG{p}{)}
\PYG{n}{plt}\PYG{o}{.}\PYG{n}{plot}\PYG{p}{(}\PYG{n}{xgrid}\PYG{p}{,} \PYG{n}{gp}\PYG{p}{(}\PYG{n}{xgrid}\PYG{p}{)}\PYG{p}{)}
\PYG{n}{plt}\PYG{o}{.}\PYG{n}{hlines}\PYG{p}{(}\PYG{n}{y}\PYG{o}{=}\PYG{l+m+mi}{0}\PYG{p}{,} \PYG{n}{xmin}\PYG{o}{=}\PYG{l+m+mi}{0}\PYG{p}{,} \PYG{n}{xmax}\PYG{o}{=}\PYG{l+m+mi}{10}\PYG{p}{)}
\PYG{n}{plt}\PYG{o}{.}\PYG{n}{show}\PYG{p}{(}\PYG{p}{)}

\PYG{n}{res} \PYG{o}{=} \PYG{n}{sp}\PYG{o}{.}\PYG{n}{optimize}\PYG{o}{.}\PYG{n}{root\PYGZus{}scalar}\PYG{p}{(}\PYG{n}{gp}\PYG{p}{,} \PYG{n}{bracket}\PYG{o}{=}\PYG{p}{[}\PYG{l+m+mi}{2}\PYG{p}{,} \PYG{l+m+mi}{4}\PYG{p}{]}\PYG{p}{,} \PYG{n}{method}\PYG{o}{=}\PYG{l+s+s1}{\PYGZsq{}}\PYG{l+s+s1}{brentq}\PYG{l+s+s1}{\PYGZsq{}}\PYG{p}{)}
\PYG{n+nb}{print}\PYG{p}{(}\PYG{l+s+s2}{\PYGZdq{}}\PYG{l+s+s2}{The smallest stationary point is tm= }\PYG{l+s+s2}{\PYGZdq{}}\PYG{p}{,} \PYG{n}{res}\PYG{o}{.}\PYG{n}{root}\PYG{p}{)}
\PYG{n+nb}{print}\PYG{p}{(}\PYG{l+s+s2}{\PYGZdq{}}\PYG{l+s+s2}{The minimum is }\PYG{l+s+s2}{\PYGZdq{}}\PYG{p}{,} \PYG{n}{g}\PYG{p}{(}\PYG{n}{res}\PYG{o}{.}\PYG{n}{root}\PYG{p}{)}\PYG{p}{)}
\end{sphinxVerbatim}

\end{sphinxuseclass}\end{sphinxVerbatimInput}
\begin{sphinxVerbatimOutput}

\begin{sphinxuseclass}{cell_output}
\noindent\sphinxincludegraphics{{9347b4f115fd6333b69174adffbecc626c3ff4ff315e4503a61f9b76d6794b9a}.png}

\begin{sphinxVerbatim}[commandchars=\\\{\}]
The smallest stationary point is tm=  2.889969697678371
The minimum is  \PYGZhy{}0.24897613487740497
\end{sphinxVerbatim}

\end{sphinxuseclass}\end{sphinxVerbatimOutput}

\end{sphinxuseclass}
\sphinxAtStartPar
The minimizers are \(\{(x,y)\in\mathbb{R}: x^2+y^2 = t_m\}\).
To verify that \(t_m\) is the unique minimizer for \(g\), since \(\cos^2(t) + \sin^2(t)=1\), we rewrite FOC to get
\begin{equation*}
\begin{split}
t\sin(t) + \sin(t) = \pm \sqrt{1-\sin^2(t)} \\
⇔ \sin^2(t) = \frac{1}{2 + 2t + t^2} \\
⇔ \cos^2(t) = 1 - \frac{1}{2+2t + t^2}=\frac{(1+t)^2}{2+2t + t^2}\\
⇒ g(t) = \frac{\cos(t)}{1+t} = \pm \frac{1}{\sqrt{2+2t +t^2}}  \qquad (\text{$t$ is stationary point}).\end{split}
\end{equation*}
\sphinxAtStartPar
Therefore, the smallest stationary point such that \(\cos(t) < 0\) will be the unique minimizer for \(g\).
\$\$

\sphinxstepscope


\chapter{Elements of set theory}
\label{\detokenize{03.set_theory:elements-of-set-theory}}\label{\detokenize{03.set_theory::doc}}
\sphinxAtStartPar
\sphinxstylestrong{ECON2125/6012 Lecture 3}\\
Fedor Iskhakov


\section{Announcements \& Reminders}
\label{\detokenize{03.set_theory:announcements-reminders}}

\section{Plan for this lecture}
\label{\detokenize{03.set_theory:plan-for-this-lecture}}
\sphinxAtStartPar
We now turn to more formal / foundational ideas
\begin{enumerate}
\sphinxsetlistlabels{\arabic}{enumi}{enumii}{}{.}%
\item {} 
\sphinxAtStartPar
Logic and proofs

\item {} 
\sphinxAtStartPar
Sets, operations with sets

\item {} 
\sphinxAtStartPar
Sequences, limits, operations with limits

\item {} 
\sphinxAtStartPar
Functions, properties of functions

\item {} 
\sphinxAtStartPar
Differentiation, Taylor series\\
+

\item {} 
\sphinxAtStartPar
Analysis in \(\mathbb{R}^n\)

\end{enumerate}

\sphinxAtStartPar
Mainly review of key ideas

\sphinxAtStartPar
\sphinxstylestrong{Supplementary reading:}
\begin{itemize}
\item {} 
\sphinxAtStartPar
Simon \& Blume:

\item {} 
\sphinxAtStartPar
Sundaram:

\end{itemize}

\begin{sphinxadmonition}{note}{Common symbols}
\begin{itemize}
\item {} 
\sphinxAtStartPar
\(P \implies Q\) means “\(P\) implies \(Q\)”

\item {} 
\sphinxAtStartPar
\(P \iff Q\) means “\(P \implies Q\) and \(Q \implies P\)”

\item {} 
\sphinxAtStartPar
\(\exists\) means “there exists”

\item {} 
\sphinxAtStartPar
\(\forall\) means “for all”

\item {} 
\sphinxAtStartPar
s.t. means “such that”

\item {} 
\sphinxAtStartPar
\(\because\) means “because” (not used very often)

\item {} 
\sphinxAtStartPar
\(\therefore\) means “therefore” (not used very often)

\item {} 
\sphinxAtStartPar
\(a := 1\) means “\(a\) is defined to be equal to 1” (alternatively \(a \equiv 1\) or
\(a \stackrel{def.}{=} 1 \))

\item {} 
\sphinxAtStartPar
\(\mathbb{R}\) means all real numbers

\item {} 
\sphinxAtStartPar
\(\mathbb{N}\) means the natural numbers \(\{1, 2, \ldots \}\)

\item {} 
\sphinxAtStartPar
\(\mathbb{Z}\) means integers \(\{\ldots, -2,-1,0,1, 2, \ldots \}\)

\item {} 
\sphinxAtStartPar
\(\mathbb{Q}\) means the rational numbers (ratios of two integers)

\end{itemize}
\end{sphinxadmonition}


\section{Logic}
\label{\detokenize{03.set_theory:logic}}
\sphinxAtStartPar
Let \(P\) and \(Q\) be statements, such as
\begin{itemize}
\item {} 
\sphinxAtStartPar
\(x\) is a negative integer

\item {} 
\sphinxAtStartPar
\(x\) is an odd number

\item {} 
\sphinxAtStartPar
the area of any circle in the plane is \(-2 \pi R\)

\end{itemize}

\sphinxAtStartPar
\sphinxhref{https://en.wikipedia.org/wiki/Law\_of\_excluded\_middle}{Law of the excluded middle}: Every mathematical statement is either \sphinxcode{\sphinxupquote{true}} or \sphinxcode{\sphinxupquote{false}}

\sphinxAtStartPar
Statement “\(P \implies Q\)” means “\(P\) implies \(Q\)”

\begin{sphinxadmonition}{note}{Example}

\sphinxAtStartPar
\(k\) is even \(\implies\) \(k = 2n\) for some integer \(n\)
\end{sphinxadmonition}

\sphinxAtStartPar
Equivalent forms of \(P \implies Q\):
\begin{enumerate}
\sphinxsetlistlabels{\arabic}{enumi}{enumii}{}{.}%
\item {} 
\sphinxAtStartPar
If \(P\) is true then \(Q\) is true

\item {} 
\sphinxAtStartPar
\(P\) is a \sphinxstyleemphasis{sufficient condition} for \(Q\)

\item {} 
\sphinxAtStartPar
\(Q\) is a necessary condition for \(P\)

\item {} 
\sphinxAtStartPar
If \(Q\) fails then \(P\) fails

\end{enumerate}

\begin{figure}[htbp]
\centering

\noindent\sphinxincludegraphics[scale=0.5]{{subset}.png}
\end{figure}

\sphinxAtStartPar
Equivalent ways of saying \(P \nRightarrow Q\) (\sphinxstyleemphasis{does not imply}):
\begin{enumerate}
\sphinxsetlistlabels{\arabic}{enumi}{enumii}{}{.}%
\item {} 
\sphinxAtStartPar
\(P\) does not imply \(Q\)

\item {} 
\sphinxAtStartPar
\(P\) is not sufficient for \(Q\)

\item {} 
\sphinxAtStartPar
\(Q\) is not necessary for \(P\)

\item {} 
\sphinxAtStartPar
Even if \(Q\) fails, \(P\) can still hold

\end{enumerate}

\begin{figure}[htbp]
\centering

\noindent\sphinxincludegraphics[scale=0.5]{{notsubset}.png}
\end{figure}

\begin{sphinxadmonition}{note}{Example}

\sphinxAtStartPar
Let
\begin{itemize}
\item {} 
\sphinxAtStartPar
\(P := \) “\(n \in \mathbb{N}\) and even”

\item {} 
\sphinxAtStartPar
\(Q := \) “\(n\) even”

\end{itemize}

\sphinxAtStartPar
Then
\begin{enumerate}
\sphinxsetlistlabels{\arabic}{enumi}{enumii}{}{.}%
\item {} 
\sphinxAtStartPar
\(P \implies Q\)

\item {} 
\sphinxAtStartPar
\(P\) is sufficient for \(Q\)

\item {} 
\sphinxAtStartPar
\(Q\) is necessary for \(P\)

\item {} 
\sphinxAtStartPar
If \(Q\) fails then \(P\) fails

\end{enumerate}
\end{sphinxadmonition}

\begin{sphinxadmonition}{note}{Example}

\sphinxAtStartPar
Let
\begin{itemize}
\item {} 
\sphinxAtStartPar
\(P := \) “\(R\) is a rectangle”

\item {} 
\sphinxAtStartPar
\(Q := \) “\(R\) is a square”

\end{itemize}

\sphinxAtStartPar
Then
\begin{enumerate}
\sphinxsetlistlabels{\arabic}{enumi}{enumii}{}{.}%
\item {} 
\sphinxAtStartPar
\(P \not \Rightarrow Q\)

\item {} 
\sphinxAtStartPar
\(P\) is not sufficient for \(Q\)

\item {} 
\sphinxAtStartPar
\(Q\) is not necessary for \(P\)

\item {} 
\sphinxAtStartPar
Just because \(Q\) fails does not mean that \(P\) fails

\end{enumerate}
\end{sphinxadmonition}


\section{Proof by contradiction}
\label{\detokenize{03.set_theory:proof-by-contradiction}}
\sphinxAtStartPar
Suppose we wish to prove a statement such as \(P \implies Q\)
\begin{enumerate}
\sphinxsetlistlabels{\arabic}{enumi}{enumii}{}{.}%
\item {} 
\sphinxAtStartPar
A proof by contradiction starts by \sphinxstylestrong{assuming the opposite}: \(P\) holds and yet \(Q\) fails.

\item {} 
\sphinxAtStartPar
We then show that this scenario leads to a contradiction

\end{enumerate}

\begin{sphinxadmonition}{note}{Examples of contradictions}
\begin{itemize}
\item {} 
\sphinxAtStartPar
\(1 < 0\)

\item {} 
\sphinxAtStartPar
\(10\) is an odd number

\end{itemize}
\end{sphinxadmonition}

\sphinxAtStartPar
We then conclude that \(P \implies Q\) is valid after all.

\begin{sphinxadmonition}{note}{Example: proof by contradiction}

\sphinxAtStartPar
Suppose that island X is populated only by pirates and knights:
\begin{itemize}
\item {} 
\sphinxAtStartPar
pirates always lie

\item {} 
\sphinxAtStartPar
knights always tell the truth

\end{itemize}

\sphinxAtStartPar
Claim to prove: If person Y says \sphinxcode{\sphinxupquote{"I'm a pirate"}} then person Y is \sphinxstyleemphasis{\sphinxstylestrong{not}} a native of island X
\end{sphinxadmonition}

\sphinxAtStartPar
Strategy for the \sphinxstylestrong{Proof:}
\begin{enumerate}
\sphinxsetlistlabels{\arabic}{enumi}{enumii}{}{.}%
\item {} 
\sphinxAtStartPar
Suppose person Y is a native of the island

\item {} 
\sphinxAtStartPar
Show that this leads to a contradiction

\item {} 
\sphinxAtStartPar
Conclude that Y is not a native of island X, as claimed

\end{enumerate}

\sphinxAtStartPar
\sphinxstylestrong{Proof:}
Suppose to the contrary that person Y \sphinxstyleemphasis{\sphinxstylestrong{is}} a native of island X
\begin{itemize}
\item {} 
\sphinxAtStartPar
then Y is either a pirate or a knight

\end{itemize}
\begin{enumerate}
\sphinxsetlistlabels{\arabic}{enumi}{enumii}{}{.}%
\item {} 
\sphinxAtStartPar
Suppose first that Y is knight

\end{enumerate}
\begin{itemize}
\item {} 
\sphinxAtStartPar
Y is a knight who claims to be a pirate

\item {} 
\sphinxAtStartPar
This is impossible, since knights always tell the truth

\end{itemize}
\begin{enumerate}
\sphinxsetlistlabels{\arabic}{enumi}{enumii}{}{.}%
\setcounter{enumi}{1}
\item {} 
\sphinxAtStartPar
Suppose next that Y is pirate

\end{enumerate}
\begin{itemize}
\item {} 
\sphinxAtStartPar
Y is a pirate who claims to be a pirate

\item {} 
\sphinxAtStartPar
Since pirates always lie, they would not make such a statement

\end{itemize}

\sphinxAtStartPar
Either way we get a contradiction \(\implies\) Y is not a native of the island!

\begin{sphinxadmonition}{note}{Example}

\sphinxAtStartPar
There is \sphinxstyleemphasis{\sphinxstylestrong{no}} \(x \in \mathbb{R}\) such that \(0 < x < 1/n\), \(\forall n \in \mathbb{N}\).
\end{sphinxadmonition}

\sphinxAtStartPar
\sphinxstylestrong{Proof:} Suppose to the contrary that such an \(x\) exists

\begin{figure}[htbp]
\centering

\noindent\sphinxincludegraphics{{no_x}.png}
\end{figure}

\sphinxAtStartPar
Since \(x > 0\) the number \(1/x\) is finite

\sphinxAtStartPar
Let \(k\) be the smallest integer such that \(k \geq 1/x\)
\begin{itemize}
\item {} 
\sphinxAtStartPar
if \(x = 0.3\) then \(1/x = 3.333\cdots\), so set \(k = 4 \in \mathbb{N}\)

\item {} 
\sphinxAtStartPar
if \(x = 0.02\) then \(1/x = 50\cdots\), so set \(k = 50 \in \mathbb{N}\)

\end{itemize}

\sphinxAtStartPar
Since \(k \geq 1/x\) we also have \(1/k \leq x\)

\sphinxAtStartPar
On the other hand, since \(k \in \mathbb{N}\), we have \(x < 1/k\)

\sphinxAtStartPar
But then \(1/k \leq x < 1/k\), and in particular \(1/k < 1/k\), which is impossible — a contradiction!

\begin{sphinxadmonition}{note}{Example}

\sphinxAtStartPar
Let \(n \in \mathbb{N}\). Show that \(n^2\) odd \(\implies\) \(n\) odd
\end{sphinxadmonition}

\sphinxAtStartPar
\sphinxstylestrong{Proof:} Suppose to the contrary that is:
\begin{enumerate}
\sphinxsetlistlabels{\arabic}{enumi}{enumii}{}{.}%
\item {} 
\sphinxAtStartPar
\(n \in \mathbb{N}\) and \(n^2\) is odd

\item {} 
\sphinxAtStartPar
but \(n\) is even

\end{enumerate}

\sphinxAtStartPar
Then \(n = 2k\) for some \(k \in \mathbb{N}\)

\sphinxAtStartPar
Hence \(n^2 = (2k)^2\)

\sphinxAtStartPar
But then \(n^2 = 2m\) for \(m := 2k^2 \in \mathbb{N}\), and thus \(n^2\) is even!

\sphinxAtStartPar
Contradiction


\section{Sets}
\label{\detokenize{03.set_theory:sets}}
\sphinxAtStartPar
Will often refer to the \sphinxstyleemphasis{\sphinxstylestrong{real numbers}},  \(\mathbb{R}\)

\sphinxAtStartPar
Understand it to contain “all of the numbers” on the “real line”

\begin{figure}[htbp]
\centering

\noindent\sphinxincludegraphics{{real_line}.png}
\end{figure}

\sphinxAtStartPar
Contains both the rational and the irrational numbers

\sphinxAtStartPar
\(\mathbb{R}\) is an example of a \sphinxstyleemphasis{\sphinxstylestrong{set}}

\sphinxAtStartPar
A set is a collection of objects viewed as a whole

\sphinxAtStartPar
(In case of \(\mathbb{R}\), the objects are numbers)

\sphinxAtStartPar
Other examples of sets:
\begin{itemize}
\item {} 
\sphinxAtStartPar
set of all rectangles in the plane

\item {} 
\sphinxAtStartPar
set of all prime numbers

\item {} 
\sphinxAtStartPar
set of students in the class

\end{itemize}

\sphinxAtStartPar
Notation:
\begin{itemize}
\item {} 
\sphinxAtStartPar
Sets: \(A, B, C\)

\item {} 
\sphinxAtStartPar
Elements: \(x,y,z\)

\end{itemize}

\sphinxAtStartPar
Important sets:
\begin{itemize}
\item {} 
\sphinxAtStartPar
\(\mathbb{N} := \{1, 2, 3, \ldots \}\)

\item {} 
\sphinxAtStartPar
\(\mathbb{Z} := \{\ldots, -2, -1, 0, 1, 2, \ldots \}\)

\item {} 
\sphinxAtStartPar
\(\mathbb{Q} := \{ p/q : p, q \in \mathbb{Z}, \; q \ne 0 \}\)

\item {} 
\sphinxAtStartPar
\(\mathbb{R} := \mathbb{Q} \cup \{ \text{ irrationals } \}\)

\end{itemize}

\begin{sphinxadmonition}{note}{Definition of a set}

\sphinxAtStartPar
A set \(A\) can be defined by either
\begin{itemize}
\item {} 
\sphinxAtStartPar
direct enumeration of its elements

\item {} 
\sphinxAtStartPar
defining a formula for infinite number of elements

\item {} 
\sphinxAtStartPar
as a \sphinxstyleemphasis{subset} of already defined set \(B\) and known function \(\psi(x)\)

\end{itemize}
\begin{equation*}
\begin{split}
A = \{ \psi(x), x \in B \colon \text{condition on x}\}
\end{split}
\end{equation*}\end{sphinxadmonition}


\section{Intervals of \protect\(\mathbb{R}\protect\)}
\label{\detokenize{03.set_theory:intervals-of-mathbb-r}}
\sphinxAtStartPar
Common notation:
\begin{equation*}
\begin{split}
(a, b)  := \{ x \in \mathbb{R} : a < x < b \}
\end{split}
\end{equation*}\begin{equation*}
\begin{split}
(a, b]  := \{ x \in \mathbb{R} : a < x \leq b \}
\end{split}
\end{equation*}\begin{equation*}
\begin{split}
[a, b)  := \{ x \in \mathbb{R} : a \leq x < b \}
\end{split}
\end{equation*}\begin{equation*}
\begin{split}
[a, b]  := \{ x \in \mathbb{R} : a \leq x \leq b \}
\end{split}
\end{equation*}\begin{equation*}
\begin{split}
[a, \infty) := \{ x \in \mathbb{R} : a \leq x  \}
\end{split}
\end{equation*}\begin{equation*}
\begin{split}
(-\infty, b) := \{ x \in \mathbb{R} :  x < b  \}
\end{split}
\end{equation*}
\sphinxAtStartPar
Etc.


\section{Operations with sets}
\label{\detokenize{03.set_theory:operations-with-sets}}
\sphinxAtStartPar
Let \(A\) and \(B\) be any sets

\sphinxAtStartPar
Statement \(x \in A\) means that \(x\) is an element of \(A\)

\sphinxAtStartPar
\(A \subset B\) means that any element of \(A\) is also an element of \(B\)

\begin{sphinxadmonition}{note}{Example}
\begin{itemize}
\item {} 
\sphinxAtStartPar
\(\mathbb{N} \subset \mathbb{Z}\)

\item {} 
\sphinxAtStartPar
irrational numbers are a subset of \(\mathbb{R}\)

\end{itemize}
\end{sphinxadmonition}

\sphinxAtStartPar
\sphinxstyleemphasis{\sphinxstylestrong{Equality}} of \(A\) and \(B\)

\sphinxAtStartPar
Let \(S\) be any set and \(A\) and \(B\) be subsets of \(S\)

\sphinxAtStartPar
\(A = B\) means that \(A\) and \(B\) contain the same elements

\sphinxAtStartPar
Equivalently, \(A = B\) \(\iff\) \(A \subset B\) and \(B \subset A\)

\sphinxAtStartPar
\sphinxstyleemphasis{\sphinxstylestrong{Union}} of \(A\) and \(B\)
\begin{equation*}
\begin{split}
A \cup B := 
\{ x \in S : x \in A \text{ or } x \in B \}
\end{split}
\end{equation*}
\sphinxAtStartPar
\sphinxstyleemphasis{\sphinxstylestrong{Intersection}} of \(A\) and \(B\)
\begin{equation*}
\begin{split}
A \cap B := 
\{ x \in S : x \in A \text{ and } x \in B \}
\end{split}
\end{equation*}
\sphinxAtStartPar
\sphinxstyleemphasis{\sphinxstylestrong{Set theoretic difference}} of \(A\) and \(B\)
\begin{equation*}
\begin{split}
A \setminus B := 
\{ x \in S : x \in A \text{ and } x \notin B \}
\end{split}
\end{equation*}
\sphinxAtStartPar
In other words, all points in \(A\) that are not points in \(B\)

\begin{sphinxadmonition}{note}{Example}
\begin{itemize}
\item {} 
\sphinxAtStartPar
\(\mathbb{Z} \setminus \mathbb{N} = \{\ldots, -2, -1, 0\}\)

\item {} 
\sphinxAtStartPar
\(\mathbb{R} \setminus \mathbb{Q} = \) the set of irrational numbers

\item {} 
\sphinxAtStartPar
\(\mathbb{R} \setminus [0, \infty) = (-\infty, 0)\)

\item {} 
\sphinxAtStartPar
\(\mathbb{R} \setminus (a, b) = (-\infty, a] \cup [b, \infty)\)

\end{itemize}
\end{sphinxadmonition}

\sphinxAtStartPar
\sphinxstyleemphasis{\sphinxstylestrong{Complement}} of \(A\)

\sphinxAtStartPar
All elements of \(S\) that are not in \(A\):
\begin{equation*}
\begin{split}
A^c := S \setminus A :=: \{ x \in S : x \notin A \}
\end{split}
\end{equation*}
\sphinxAtStartPar
Remarks:
\begin{itemize}
\item {} 
\sphinxAtStartPar
Need to know what \(S\) is before we can determine \(A^c\)

\item {} 
\sphinxAtStartPar
If not clear better write \(S \setminus A\)

\end{itemize}

\begin{sphinxadmonition}{note}{Example}

\sphinxAtStartPar
\((a,\infty)^c\) generally understood to be \((-\infty, a]\)
\end{sphinxadmonition}

\begin{figure}[htbp]
\centering
\capstart

\noindent\sphinxincludegraphics{{allsets}.png}
\caption{\textbackslash{}label\{f:allsets\} Unions, intersections and complements}\label{\detokenize{03.set_theory:allsets}}\end{figure}


\section{Set operations properties}
\label{\detokenize{03.set_theory:set-operations-properties}}
\sphinxAtStartPar
If \(A\) and \(B\) subsets of \(S\), then
\begin{enumerate}
\sphinxsetlistlabels{\arabic}{enumi}{enumii}{}{.}%
\item {} 
\sphinxAtStartPar
\(A \cup B = B \cup A\) and \(A \cap B = B \cap A\)

\end{enumerate}
\begin{itemize}
\item {} 
\sphinxAtStartPar
\((A \cup B)^c = B^c \cap A^c\) and \((A \cap B)^c = B^c \cup A^c\)

\item {} 
\sphinxAtStartPar
\(A \setminus B = A \cap B^c\)

\end{itemize}
\begin{enumerate}
\sphinxsetlistlabels{\arabic}{enumi}{enumii}{}{.}%
\setcounter{enumi}{8}
\item {} 
\sphinxAtStartPar
\((A^c)^c = A\)

\end{enumerate}

\sphinxAtStartPar
The \sphinxstyleemphasis{\sphinxstylestrong{empty set}} \(\emptyset\) is the set containing no elements

\sphinxAtStartPar
If \(A \cap B = \emptyset\), then \(A\) and \(B\) said to be \sphinxstyleemphasis{\sphinxstylestrong{disjoint}}


\section{Infinite Unions and Intersections}
\label{\detokenize{03.set_theory:infinite-unions-and-intersections}}
\sphinxAtStartPar
Given a family of sets \(K_{\lambda} \subset S\) with \(\lambda \in \Lambda\),
\begin{equation*}
\begin{split}
\bigcap_{\lambda \in \Lambda} K_{\lambda} 
:= \{ x \in S : x\in K_{\lambda}
\textnormal{ for all } \lambda \in \Lambda \}
\end{split}
\end{equation*}\begin{equation*}
\begin{split}
\bigcup_{\lambda \in \Lambda} K_{\lambda}  
:= \{x \in S \colon \textnormal{there
exists an } \lambda \in \Lambda \textnormal{ such that } x\in K_{\lambda} \}
\end{split}
\end{equation*}\begin{itemize}
\item {} 
\sphinxAtStartPar
“there exists” means “there exists \sphinxstyleemphasis{\sphinxstylestrong{at least}} one”

\end{itemize}

\begin{sphinxadmonition}{note}{Example}

\sphinxAtStartPar
Let \(A := \cap_{n \in \mathbb{N}} (0, 1/n)\)

\sphinxAtStartPar
Claim: \(A = \emptyset\)
\end{sphinxadmonition}

\sphinxAtStartPar
\sphinxstylestrong{Proof:} We need to show that \(A\) contains no elements

\sphinxAtStartPar
Suppose to the contrary that \(x \in A = \cap_{n \in \mathbb{N}} (0, 1/n)\)

\sphinxAtStartPar
Then \(x\) is a number satisfying \(0 < x < 1/n\) for all \(n \in \mathbb{N}\)

\sphinxAtStartPar
No such \(x\) exists as we showed above. Contradiction.

\begin{sphinxadmonition}{note}{Example}

\sphinxAtStartPar
For any \(a < b\) we have \(\cup_{\epsilon > 0 } \; [a + \epsilon, b) = (a, b)\)
\end{sphinxadmonition}

\sphinxAtStartPar
\sphinxstylestrong{Proof:} To show equality of the sets, we show that RHS \(\subset\) LHS and LHS \(\subset\) RHS

\sphinxAtStartPar
Pick any \(a < b\)

\sphinxAtStartPar
Suppose first that \(x \in \cup_{\epsilon > 0 } \; [a + \epsilon, b)\)

\sphinxAtStartPar
This means there exists \(\epsilon > 0\) such that \(a + \epsilon \leq x < b\)

\sphinxAtStartPar
Clearly \(a < x < b\), and hence \(x \in (a, b)\)

\sphinxAtStartPar
Conversely, if \(a < x < b\), then \(\exists \, \epsilon > 0\) s.t. \(a +
\epsilon \leq x < b\)

\sphinxAtStartPar
Hence \(x \in \cup_{\epsilon > 0 } \; [a + \epsilon, b)\)

\begin{sphinxadmonition}{note}{Fact: de Morgan’s laws}

\sphinxAtStartPar
Let \(S\) be any set and let \(K_{\lambda} \subset S\) for all \(\lambda \in \Lambda\). Then
\begin{equation*}
\begin{split}
\left[ \bigcup_{\lambda \in \Lambda} K_{\lambda}  \right]^{c}  =
\bigcap_{\lambda \in \Lambda} K_{\lambda}^{c}
\quad \text{and} \quad
\left[ \bigcap_{\lambda \in \Lambda}
K_{\lambda}  \right]^{c}  = \bigcup_{\lambda \in \Lambda} K_{\lambda}^{c}
\end{split}
\end{equation*}\end{sphinxadmonition}

\sphinxAtStartPar
Let’s prove that \(A := \left( \cup_{\lambda \in \Lambda} K_{\lambda}  \right)^{c}
= \cap_{\lambda \in \Lambda} K_{\lambda}^{c} =: B\)

\sphinxAtStartPar
Suffices to show that \(A \subset B\) and \(B \subset A\)

\sphinxAtStartPar
Let’s just do  \(A \subset B\)

\sphinxAtStartPar
Must show that every \(x \in A\) is also in \(B\)

\sphinxAtStartPar
Fix \(x \in A\)

\sphinxAtStartPar
Since \(x \in A\), it must be that \(x\) is not in \(\cup_{\lambda \in \Lambda} K_{\lambda}\)
\begin{equation*}
\begin{split}
\text{therefore } \text{ $x$ is not in any $K_{\lambda}$ }
\end{split}
\end{equation*}\begin{equation*}
\begin{split}
\text{therefore } x \in K_{\lambda}^c \text{ for each } \lambda \in \Lambda
\end{split}
\end{equation*}\begin{equation*}
\begin{split}
\text{therefore } x \in \cap_{\lambda \in \Lambda} K_{\lambda}^{c} =: B
\end{split}
\end{equation*}

\section{Tuples}
\label{\detokenize{03.set_theory:tuples}}
\sphinxAtStartPar
We often organize collections with natural order into “tuples”

\begin{sphinxadmonition}{note}{Definition}

\sphinxAtStartPar
A \sphinxstyleemphasis{\sphinxstylestrong{tuple}} is
\begin{itemize}
\item {} 
\sphinxAtStartPar
a finite ordered sequence of terms

\item {} 
\sphinxAtStartPar
denoted using notation such as \((a_1, a_2)\) or \((x_1, x_2, x_3)\)

\end{itemize}
\end{sphinxadmonition}

\begin{sphinxadmonition}{note}{Example}

\sphinxAtStartPar
Flip a coin 10 times and let
\begin{itemize}
\item {} 
\sphinxAtStartPar
\(0\) represent tails and \(1\) represent heads

\end{itemize}

\sphinxAtStartPar
Typical outcome \((1, 1, 0, 0, 0, 0, 1, 0, 1, 1)\)

\sphinxAtStartPar
Generic outcome \((b_1, b_2, \ldots, b_{10})\)  for \(b_n \in \{0, 1\}\)
\end{sphinxadmonition}


\section{Cartesian Products}
\label{\detokenize{03.set_theory:cartesian-products}}
\sphinxAtStartPar
We make collections of tuples using Cartesian products

\begin{sphinxadmonition}{note}{Definition}

\sphinxAtStartPar
The \sphinxstyleemphasis{\sphinxstylestrong{Cartesian product}} of \(A_1, \ldots, A_N\) is the set
\begin{equation*}
\begin{split}
A_1 \times \cdots \times A_N
:= \{ (a_1, \ldots, a_N) : a_n \in A_n \text{ for } n =1, \ldots, N \}
\end{split}
\end{equation*}\end{sphinxadmonition}

\begin{sphinxadmonition}{note}{Example}
\begin{equation*}
\begin{split}
```[0, 8] \times [0, 1] = \{ (x_1,x_2) : 0 \leq x_1 \leq 8, \, 0 \leq x_2 \leq 1 \}
\end{split}
\end{equation*}\end{sphinxadmonition}

\begin{figure}[htbp]
\centering

\noindent\sphinxincludegraphics{{cart_prod}.png}
\end{figure}

\begin{sphinxadmonition}{note}{Example}

\sphinxAtStartPar
Set of all outcomes from flip experiment is
\end{sphinxadmonition}
\begin{equation*}
\begin{split}
B := \Big\{ (b_1, \ldots, b_{10}) : b_n \in \{0, 1\} \text{ for } n = 1, \ldots, 10 \Big\}
\end{split}
\end{equation*}\begin{equation*}
\begin{split}
= \{0, 1\} \times \cdots \times \{0, 1\} \quad (10 \text{ products})
\end{split}
\end{equation*}
\begin{sphinxadmonition}{note}{Example}

\sphinxAtStartPar
The \sphinxstyleemphasis{\sphinxstylestrong{vector space}} \(\mathbb{R}^N\) is the Cartesian product
\begin{equation*}
\begin{split}
\mathbb{R}^N = \mathbb{R} \times \cdots \times \mathbb{R} \quad (N \text{ times})
\end{split}
\end{equation*}\begin{equation*}
\begin{split}
= \{ 
\, \text{all tuples } (x_1, \ldots, x_N) \text{ with } x_n \in \mathbb{R}
\}
\end{split}
\end{equation*}\end{sphinxadmonition}


\section{Counting Finite Sequences}
\label{\detokenize{03.set_theory:counting-finite-sequences}}
\sphinxAtStartPar
Counting methods answer common questions such as
\begin{itemize}
\item {} 
\sphinxAtStartPar
How many arrangements of a sequence?

\item {} 
\sphinxAtStartPar
How many subsets of a set?

\end{itemize}

\sphinxAtStartPar
They also address deeper problems such as
\begin{itemize}
\item {} 
\sphinxAtStartPar
How “large” is a given set?

\item {} 
\sphinxAtStartPar
Can we compare size of sets even when they are infinite?

\end{itemize}

\sphinxAtStartPar
The key rule is: multiply possibilities

\begin{sphinxadmonition}{note}{Example}

\sphinxAtStartPar
Can travel from Sydney to Tokyo in 3 ways and Tokyo to NYC in 8 ways
\(\implies\) can travel from Sydney to NYC in 24 ways
\end{sphinxadmonition}

\begin{sphinxadmonition}{note}{Example}

\sphinxAtStartPar
Number of 10 letter passwords from the lowercase letters \sphinxcode{\sphinxupquote{a,b,...,z}} is
\$\(
26^{10} = 141,167,095,653,376
\)\$
\end{sphinxadmonition}

\begin{sphinxadmonition}{note}{Example}

\sphinxAtStartPar
Number of possible distinct outcomes \((i, j)\) from 2 rolls of a dice is
\$\(
6 \times 6 = 36
\)\$
\end{sphinxadmonition}


\section{Counting Cartesian Products}
\label{\detokenize{03.set_theory:counting-cartesian-products}}
\begin{sphinxadmonition}{note}{Fact}

\sphinxAtStartPar
If \(A_n\) are finite for \(n=1, \ldots,N\), then
\begin{equation*}
\begin{split}
\#(A_1 \times \cdots \times A_N) = (\# A_1) \times \cdots \times (\# A_N)
\end{split}
\end{equation*}\end{sphinxadmonition}

\sphinxAtStartPar
That is, number of possible tuples \(=\) product of the number of
possibilities for each element

\begin{sphinxadmonition}{note}{Example}

\sphinxAtStartPar
Number of binary sequences of length \(10\) is
\$\(
\# [\{0, 1\} \times \cdots \times \{0, 1\}]
= 2 \times \cdots \times 2 = 2^{10}
\)\$
\end{sphinxadmonition}

\sphinxAtStartPar
\sphinxstylestrong{Infinite Cartesian Products}

\sphinxAtStartPar
If \(\{A_n\}\) is a collection of sets, one
for each \(n \in \mathbb{N}\), then
\begin{equation*}
\begin{split}
A_1 \times A_2 \times \cdots 
:= \{ (a_1, a_2, \ldots) : a_n \in A_n \text{ for each } n \in \mathbb{N} \}
\end{split}
\end{equation*}
\sphinxAtStartPar
Sometimes denoted \(\times_{n=1}^{\infty} A_n\)

\sphinxAtStartPar
If \(A_n = A\) for all \(n\), then \(\times_{n=1}^{\infty} A\) also written as \(A^{\mathbb{N}}\)

\begin{sphinxadmonition}{note}{Example}

\sphinxAtStartPar
The set of all binary sequences \(\{0, 1\}^{\mathbb{N}}\)
\end{sphinxadmonition}


\section{Functions}
\label{\detokenize{03.set_theory:functions}}
\begin{sphinxadmonition}{note}{Definition}

\sphinxAtStartPar
A \sphinxstyleemphasis{\sphinxstylestrong{function}} \(f \colon A \rightarrow B\) from set \(A\) to set \(B\) is a rule that
associates to each element of \(A\) a uniquely determined element of \(B\)
\end{sphinxadmonition}
\begin{itemize}
\item {} 
\sphinxAtStartPar
\(f \colon A \to B\) means that \(f\) is a function from \(A\) to \(B\)

\end{itemize}

\begin{figure}[htbp]
\centering

\noindent\sphinxincludegraphics{{function}.png}
\end{figure}

\sphinxAtStartPar
\(A\) is called the \sphinxstyleemphasis{\sphinxstylestrong{domain}} of \(f\) and \(B\) is called the \sphinxstyleemphasis{\sphinxstylestrong{codomain}}

\begin{sphinxadmonition}{note}{Example}

\sphinxAtStartPar
\(f\) defined by
\begin{equation*}
\begin{split}
f(x) = \exp(-x^2)
\end{split}
\end{equation*}
\sphinxAtStartPar
is a function from \(\mathbb{R}\) to \(\mathbb{R}\)

\sphinxAtStartPar
Sometimes we write the whole thing like this
\begin{equation*}
\begin{split}
f \colon \mathbb{R} \to \mathbb{R} \\
x \mapsto \exp(-x^2), \text{ or}\\
f \colon \mathbb{R} \ni x \mapsto \exp(-x^2) \in \mathbb{R} 
\end{split}
\end{equation*}\end{sphinxadmonition}

\begin{figure}[htbp]
\centering

\noindent\sphinxincludegraphics{{allfunctions}.png}
\end{figure}

\begin{sphinxadmonition}{note}{Example: not a function}

\begin{figure}[H]
\centering

\noindent\sphinxincludegraphics{{xy_non_func}.png}
\end{figure}
\end{sphinxadmonition}

\sphinxAtStartPar
For each \(a \in A\), \(f(a) \in B\) is called the \sphinxstyleemphasis{\sphinxstylestrong{image of \(a\)}} under \(f\)

\begin{figure}[htbp]
\centering

\noindent\sphinxincludegraphics{{xy_func}.png}
\end{figure}

\sphinxAtStartPar
If \(f(a) = b\) then \(a\) is called a \sphinxstyleemphasis{\sphinxstylestrong{preimage of \(b\)}} under \(f\)

\begin{figure}[htbp]
\centering

\noindent\sphinxincludegraphics{{preimage}.png}
\end{figure}

\sphinxAtStartPar
A point in \(B\) can have one, many or zero preimages

\begin{figure}[htbp]
\centering

\noindent\sphinxincludegraphics{{preimage2}.png}
\end{figure}

\sphinxAtStartPar
The codomain of a function is not uniquely pinned down

\begin{sphinxadmonition}{note}{Example}

\sphinxAtStartPar
Consider the mapping defined by
\$\(f(x) = \exp(-x^2)\)\$

\sphinxAtStartPar
Both of these statements are valid:
\begin{itemize}
\item {} 
\sphinxAtStartPar
\(f\) a function from \(\mathbb{R}\) to \(\mathbb{R}\)

\item {} 
\sphinxAtStartPar
\(f\) a function from \(\mathbb{R}\) to \((0, \infty)\)

\end{itemize}
\end{sphinxadmonition}

\sphinxAtStartPar
The smallest possible codomain is called the \sphinxstyleemphasis{\sphinxstylestrong{range}} of \(f \colon A \to B\):
\begin{equation*}
\begin{split}
\mathrm{rng}(f) := \{ b \in B : f(a) = b \text{ for some } a \in A \} 
\end{split}
\end{equation*}
\begin{figure}[htbp]
\centering

\noindent\sphinxincludegraphics{{range}.png}
\end{figure}

\begin{sphinxadmonition}{note}{Example}

\sphinxAtStartPar
Let \(f \colon [-1, 1] \to \mathbb{R}\) be defined by
\$\(
f(x) =  0.6 \cos(4 x) + 1.4
\)\(
Then \)\textbackslash{}mathrm\{rng\}(f) = {[}0.8, 2.0{]}\$
\end{sphinxadmonition}

\begin{figure}[htbp]
\centering

\noindent\sphinxincludegraphics{{range2}.png}
\end{figure}

\begin{sphinxadmonition}{note}{Example}

\sphinxAtStartPar
If \( f \colon [0, 1] \to \mathbb{R}\) is defined by
\$\(
f(x) = 2x
\)\(
then \)\textbackslash{}mathrm\{rng\}(f) = {[}0, 2{]}\$
\end{sphinxadmonition}

\begin{sphinxadmonition}{note}{Example}

\sphinxAtStartPar
If \(f \colon \mathbb{R} \to \mathbb{R}\) is defined by
\$\(
f(x) = \exp(x) 
\)\(
then \)\textbackslash{}mathrm\{rng\}(f) = (0, \textbackslash{}infty)\$
\end{sphinxadmonition}

\sphinxAtStartPar
The \sphinxstyleemphasis{\sphinxstylestrong{composition}} of \(f \colon A \to B\) and \(g \colon B \to C\) is the
function \(g \circ f\) from \(A\) to \(C\) defined by
\begin{equation*}
\begin{split}
(g \circ f)(a) = g(f(a)) \quad (a \in A)
\end{split}
\end{equation*}
\begin{figure}[htbp]
\centering

\noindent\sphinxincludegraphics{{composition}.png}
\end{figure}


\section{Onto Functions}
\label{\detokenize{03.set_theory:onto-functions}}
\sphinxAtStartPar
A function \(f \colon A \to B\) is called \sphinxstyleemphasis{\sphinxstylestrong{onto}} if every element of \(B\)
is the image under \(f\) of at least one point in \(A\).

\sphinxAtStartPar
Equivalently, \(\mathrm{rng}(f) = B\)

\begin{figure}[htbp]
\centering

\noindent\sphinxincludegraphics{{function1}.png}
\end{figure}

\begin{sphinxadmonition}{note}{Fact}

\sphinxAtStartPar
\(f \colon A \to B\) is onto if and only if each element of \(B\)
has at least one preimage under \(f\)
\end{sphinxadmonition}

\begin{figure}[htbp]
\centering
\capstart

\noindent\sphinxincludegraphics[scale=0.5]{{bijection3}.png}
\caption{Onto}\label{\detokenize{03.set_theory:bijection3}}\end{figure}

\begin{figure}[htbp]
\centering
\capstart

\noindent\sphinxincludegraphics[scale=0.5]{{function3}.png}
\caption{Not \sphinxstyleemphasis{onto}!}\label{\detokenize{03.set_theory:function3}}\end{figure}

\begin{figure}[htbp]
\centering
\capstart

\noindent\sphinxincludegraphics[scale=0.5]{{bijection2}.png}
\caption{Not \sphinxstyleemphasis{onto}!}\label{\detokenize{03.set_theory:bijection22}}\end{figure}

\begin{sphinxadmonition}{note}{Example}

\sphinxAtStartPar
The function \(f \colon \mathbb{R} \to \mathbb{R}\) defined by
\begin{equation*}
\begin{split}
f(x) = ax^3 + b x^2 + cx + d
\end{split}
\end{equation*}
\sphinxAtStartPar
is onto whenever \(a \ne 0\)
\end{sphinxadmonition}

\sphinxAtStartPar
To see this pick any \(y \in \mathbb{R}\)

\sphinxAtStartPar
We claim \(\exists \; x \in \mathbb{R}\) such that \(f(x) = y\)

\sphinxAtStartPar
Equivalent:
\begin{equation*}
\begin{split}
\exists \; x \in \mathbb{R} \; \mathrm{s.t.} \;
ax^3 + b x^2 + cx + d - y = 0
\end{split}
\end{equation*}
\begin{sphinxadmonition}{note}{Fact}

\sphinxAtStartPar
Every cubic equation with \(a \ne 0\) has at least one real root
\end{sphinxadmonition}

\begin{figure}[htbp]
\centering
\capstart

\noindent\sphinxincludegraphics{{cubic}.png}
\caption{Cubic functions from \(\mathbb{R}\) to \(\mathbb{R}\) are always onto}\label{\detokenize{03.set_theory:cubic}}\end{figure}


\section{One\sphinxhyphen{}to\sphinxhyphen{}One Functions}
\label{\detokenize{03.set_theory:one-to-one-functions}}
\sphinxAtStartPar
A function \(f \colon A \to B\) is called \sphinxstyleemphasis{\sphinxstylestrong{one\sphinxhyphen{}to\sphinxhyphen{}one}} if distinct
elements of \(A\) are always mapped into distinct elements of \(B\).

\sphinxAtStartPar
That is, \(f\) is one\sphinxhyphen{}to\sphinxhyphen{}one if
\begin{equation*}
\begin{split}
a \in A, \; a' \in A  \text{ and } a \ne a' 
\implies f(a) \ne f(a')
\end{split}
\end{equation*}
\sphinxAtStartPar
Equivalently,
\begin{equation*}
\begin{split}
f(a) = f(a') \implies a = a'
\end{split}
\end{equation*}
\begin{sphinxadmonition}{note}{Fact}

\sphinxAtStartPar
\(f \colon A \to B\) is one\sphinxhyphen{}to\sphinxhyphen{}one if and only if each element of \(B\)
has at most one preimage under \(f\)
\end{sphinxadmonition}

\begin{figure}[htbp]
\centering
\capstart

\noindent\sphinxincludegraphics[scale=0.5]{{bijection2}.png}
\caption{One\sphinxhyphen{}to\sphinxhyphen{}one}\label{\detokenize{03.set_theory:bijection2}}\end{figure}

\begin{figure}[htbp]
\centering
\capstart

\noindent\sphinxincludegraphics[scale=0.5]{{bijection3}.png}
\caption{One\sphinxhyphen{}to\sphinxhyphen{}one}\label{\detokenize{03.set_theory:bijection3b}}\end{figure}

\begin{figure}[htbp]
\centering
\capstart

\noindent\sphinxincludegraphics[scale=0.5]{{bijection1}.png}
\caption{Not one\sphinxhyphen{}to\sphinxhyphen{}one}\label{\detokenize{03.set_theory:bijection1}}\end{figure}


\section{Bijections}
\label{\detokenize{03.set_theory:bijections}}
\sphinxAtStartPar
A function that is
\begin{enumerate}
\sphinxsetlistlabels{\arabic}{enumi}{enumii}{}{.}%
\item {} 
\sphinxAtStartPar
one\sphinxhyphen{}to\sphinxhyphen{}one and

\item {} 
\sphinxAtStartPar
onto

\end{enumerate}

\sphinxAtStartPar
is called a \sphinxstyleemphasis{\sphinxstylestrong{bijection}} or \sphinxstyleemphasis{\sphinxstylestrong{one\sphinxhyphen{}to\sphinxhyphen{}one correspondence}}

\begin{figure}[htbp]
\centering

\noindent\sphinxincludegraphics[scale=0.5]{{bijection3}.png}
\end{figure}

\begin{sphinxadmonition}{note}{Fact}

\sphinxAtStartPar
\(f \colon A \to B\) is a bijection if and only if each \(b \in B\) has
\end{sphinxadmonition}

\sphinxAtStartPar
one and only one preimage in \(A\)

\begin{sphinxadmonition}{note}{Example}

\sphinxAtStartPar
\(x \mapsto 2x\) is a bijection from \(\mathbb{R}\) to \(\mathbb{R}\)
\end{sphinxadmonition}

\begin{figure}[htbp]
\centering

\noindent\sphinxincludegraphics{{x_to_2x}.png}
\end{figure}

\begin{sphinxadmonition}{note}{Example}

\sphinxAtStartPar
\(k \mapsto -k\) is a bijection from \(\mathbb{Z}\) to \(\mathbb{Z}\)
\end{sphinxadmonition}

\begin{figure}[htbp]
\centering

\noindent\sphinxincludegraphics{{k_to_minus_k}.png}
\end{figure}

\begin{sphinxadmonition}{note}{Example}

\sphinxAtStartPar
\(x \mapsto x^2\) is \sphinxstyleemphasis{\sphinxstylestrong{not}} a bijection from \(\mathbb{R}\) to \(\mathbb{R}
\)
\end{sphinxadmonition}

\begin{figure}[htbp]
\centering

\noindent\sphinxincludegraphics{{x_squared}.png}
\end{figure}

\begin{sphinxadmonition}{note}{Fact}

\sphinxAtStartPar
If \(f \colon A \to B\) a bijection, then there exists a unique
function \(\phi \colon B \to A\) such that
\begin{equation*}
\begin{split}
\phi(f(a)) = a, \quad \forall \; a \in A
\end{split}
\end{equation*}
\sphinxAtStartPar
That function \(\phi\) is called the \sphinxstyleemphasis{\sphinxstylestrong{inverse}} of \(f\) and written \(f^{-1}\)
\end{sphinxadmonition}

\begin{figure}[htbp]
\centering

\noindent\sphinxincludegraphics{{bijec}.png}
\end{figure}

\begin{sphinxadmonition}{note}{Example}

\sphinxAtStartPar
Let
\begin{itemize}
\item {} 
\sphinxAtStartPar
\(f \colon \mathbb{R} \to (0, \infty)\) be defined by \(f(x) = \exp(x) :=
e^x\)

\item {} 
\sphinxAtStartPar
\(\phi \colon (0, \infty) \to \mathbb{R}\) be defined by \(\phi(x) = \log(x)\)

\end{itemize}

\sphinxAtStartPar
Then \(\phi = f^{-1}\) because, for any \(a \in \mathbb{R}\),
\begin{equation*}
\begin{split}
\phi(f(a)) = \log(\exp(a)) = a
\end{split}
\end{equation*}\end{sphinxadmonition}

\begin{sphinxadmonition}{note}{Fact}

\sphinxAtStartPar
If \(f \colon A \to B\) is one\sphinxhyphen{}to\sphinxhyphen{}one, then \(f \colon A \to \mathrm{rng}(f)\) is a bijection
\end{sphinxadmonition}

\begin{sphinxadmonition}{note}{Fact}

\sphinxAtStartPar
Let \(f \colon A \to B\) and \(g \colon B \to C\) be bijections
\begin{enumerate}
\sphinxsetlistlabels{\arabic}{enumi}{enumii}{}{.}%
\item {} 
\sphinxAtStartPar
\(f^{-1}\) is a bijection and its inverse is \(f\)

\end{enumerate}
\begin{itemize}
\item {} 
\sphinxAtStartPar
\(f^{-1}(f(a)) = a\) for all \(a \in A\)

\item {} 
\sphinxAtStartPar
\(f(f^{-1}(b)) = b\) for all \(b \in B\)

\item {} 
\sphinxAtStartPar
\(g \circ f\) is a bijection from \(A\) to \(C\) and \((g \circ f)^{-1}
= f^{-1} \circ g^{-1}\)
\textbackslash{}end\{enumerate\}

\end{itemize}
\end{sphinxadmonition}

\begin{figure}[htbp]
\centering
\capstart

\noindent\sphinxincludegraphics{{bij_inv}.png}
\caption{Illustration of \((g \circ f)^{-1} = f^{-1} \circ g^{-1}\)}\label{\detokenize{03.set_theory:bij-inv}}\end{figure}


\section{Cardinality}
\label{\detokenize{03.set_theory:cardinality}}
\sphinxAtStartPar
If a bijection exists between sets \(A\) and \(B\) they are said to have the \sphinxstyleemphasis{\sphinxstylestrong{same cardinality}}, and we write \(|A| = |B|\)

\begin{sphinxadmonition}{note}{Fact}

\sphinxAtStartPar
If \(|A| = |B|\) and \(A\) and \(B\) are finite then \(A\) and \(B\) have the same number of elements (same cardinality).
\end{sphinxadmonition}

\sphinxAtStartPar
\sphinxstylestrong{Exercise:} Convince yourself this is true

\sphinxAtStartPar
Hence “same cardinality” is analogous to “same size”
\begin{itemize}
\item {} 
\sphinxAtStartPar
But cardinality applies to infinite sets as well!

\end{itemize}

\begin{sphinxadmonition}{note}{Fact}

\sphinxAtStartPar
If \(|A| = |B|\) and \(|B| = |C|\) then \(|A| = |C|\)
\end{sphinxadmonition}

\sphinxAtStartPar
\sphinxstylestrong{Proof:}
\begin{itemize}
\item {} 
\sphinxAtStartPar
Since \(|A| = |B|\), there exists a bijection \(f \colon A \to B\)

\item {} 
\sphinxAtStartPar
Since \(|B| = |C|\), there exists a bijection \(g \colon B \to C\)

\end{itemize}

\sphinxAtStartPar
Let \(h := g \circ f\)

\sphinxAtStartPar
Then \(h\) is a bijection from \(A\) to \(C\)

\sphinxAtStartPar
Hence \(|A| = |C|\)

\sphinxAtStartPar
A nonempty set \(A\) is called \sphinxstyleemphasis{\sphinxstylestrong{finite}} if
\$\(
|A| = |\{1, 2, \ldots, n\}|
\quad \text{ for some } \quad
n \in \mathbb{N}
\)\$

\sphinxAtStartPar
Otherwise called \sphinxstyleemphasis{\sphinxstylestrong{infinite}}

\sphinxAtStartPar
Sets that either
\begin{enumerate}
\sphinxsetlistlabels{\arabic}{enumi}{enumii}{}{.}%
\item {} 
\sphinxAtStartPar
are finite, or

\item {} 
\sphinxAtStartPar
have the same cardinality as \(\mathbb{N}\)

\end{enumerate}

\sphinxAtStartPar
are called \sphinxstyleemphasis{\sphinxstylestrong{countable}}, denoted \(|A| = \aleph_0\)

\begin{sphinxadmonition}{note}{Example}

\sphinxAtStartPar
\(-\mathbb{N} := \{\ldots, -4, -3, -2, -1\}\) is countable
\end{sphinxadmonition}
\begin{equation*}
\begin{split}
\begin{array}{ccc}
-1 & \leftrightarrow & 1 \\
-2 & \leftrightarrow & 2 \\
-3 & \leftrightarrow & 3 \\
& \vdots &  \\
-n & \leftrightarrow & n \\
& \vdots &  
\end{array}
\end{split}
\end{equation*}
\sphinxAtStartPar
Formally: \(f(k) = -k\) is a bijection from \(-\mathbb{N}\) to \(\mathbb{N}\)

\begin{sphinxadmonition}{note}{Example}

\sphinxAtStartPar
\(E := \{2, 4, \ldots\}\) is countable
\end{sphinxadmonition}
\begin{equation*}
\begin{split}
\begin{array}{ccc}
2 & \leftrightarrow & 1 \\
4 & \leftrightarrow & 2 \\
6 & \leftrightarrow & 3 \\
& \vdots &  \\
2n & \leftrightarrow & n \\
& \vdots &  
\end{array}
\end{split}
\end{equation*}
\sphinxAtStartPar
Formally: \(f(k) = k/2\) is a bijection from \(E\) to \(\mathbb{N}\)

\begin{sphinxadmonition}{note}{Example}

\sphinxAtStartPar
\(\{100, 200, 300, \ldots\}\) is countable
\end{sphinxadmonition}
\begin{equation*}
\begin{split}
\begin{array}{ccc}
100 & \leftrightarrow & 1 \\
200 & \leftrightarrow & 2 \\
300 & \leftrightarrow & 3 \\
& \vdots &  \\
100n & \leftrightarrow & n \\
& \vdots &  
\end{array}
\end{split}
\end{equation*}
\begin{sphinxadmonition}{note}{Fact}

\sphinxAtStartPar
Nonempty subsets of countable sets are countable
\end{sphinxadmonition}

\begin{sphinxadmonition}{note}{Fact}

\sphinxAtStartPar
Finite unions of countable sets are countable
\end{sphinxadmonition}

\sphinxAtStartPar
Sketch of proof, for
\begin{itemize}
\item {} 
\sphinxAtStartPar
\(A\) and \(B\) countable \(\implies A \cup B\) countable

\item {} 
\sphinxAtStartPar
\(A\) and \(B\) are disjoint and infinite

\end{itemize}

\sphinxAtStartPar
By assumption, can write \(A = \{a_1, a_2, \ldots\}\) and \(B = \{b_1, b_2,
\ldots\}\)

\sphinxAtStartPar
Now count it like so:
\begin{equation*}
\begin{split}
\begin{matrix}
a_1 &      & a_2 &      & a_3 &      & a_4 & \cdots\\
\downarrow & \nearrow & \downarrow & \nearrow & \downarrow & \nearrow & \downarrow & \nearrow \\
b_1 &      & b_2 &      & b_3 &      & b_4 & \cdots  
\end{matrix}
\end{split}
\end{equation*}
\begin{sphinxadmonition}{note}{Example}

\sphinxAtStartPar
\(\mathbb{Z} = \{\ldots, -2, -1\} \cup \{ 0 \} \cup \{1, 2, \ldots\}\) is countable
\end{sphinxadmonition}

\begin{sphinxadmonition}{note}{Fact}

\sphinxAtStartPar
Finite Cartesian products of countable sets is countable
\end{sphinxadmonition}

\sphinxAtStartPar
Sketch of proof, for
\begin{itemize}
\item {} 
\sphinxAtStartPar
\(A\) and \(B\) countable \(\implies A \times B\) countable

\item {} 
\sphinxAtStartPar
\(A\) and \(B\) are disjoint and infinite

\end{itemize}

\sphinxAtStartPar
Now count like so:
\begin{equation*}
\begin{split}
\begin{matrix}
(a_{1},b_{1})&\to &(a_{1},b_{2})&       & (a_{1},b_{3})&\to\cdots\\
&\swarrow&             &\nearrow   &              & \\
(a_{2},b_{1})&    &(a_{2},b_{2})&       &\cdots        & \\
\downarrow   &\nearrow&             &       &              & \\
(a_{3},b_{1})&    &\vdots       &       &              & \\
\vdots       &    &             &       &              & 
\end{matrix}
\end{split}
\end{equation*}
\begin{sphinxadmonition}{note}{Example}

\sphinxAtStartPar
\(\mathbb{Z} \times \mathbb{Z} = \{ (p,q) : p \in \mathbb{Z}, q \in \mathbb{Z} \}\) is countable
\end{sphinxadmonition}

\begin{figure}[htbp]
\centering

\noindent\sphinxincludegraphics{{lattice}.png}
\end{figure}

\begin{sphinxadmonition}{note}{Fact}

\sphinxAtStartPar
\(\mathbb{Q}\) is countable
\end{sphinxadmonition}

\sphinxAtStartPar
\sphinxstylestrong{Proof:} By definition
\begin{equation*}
\begin{split}
\mathbb{Q}:= 
\left\{ \, 
\text{all } \frac{p}{q}
\text{ where } p \in \mathbb{Z} \text{ and }  q \in \mathbb{N} \,
\right\}
\end{split}
\end{equation*}
\sphinxAtStartPar
Consider the function \(\phi\) defined by \(\phi(p/q) = (p, q)\)
\begin{itemize}
\item {} 
\sphinxAtStartPar
A one\sphinxhyphen{}to\sphinxhyphen{}one function from \(\mathbb{Q}\) to \(\mathbb{Z} \times \mathbb{N}\)

\item {} 
\sphinxAtStartPar
A bijection from \(\mathbb{Q}\) to \(\mathrm{rng}(\phi)\)

\end{itemize}

\sphinxAtStartPar
Since \(\mathbb{Z} \times \mathbb{N}\) is countable, so is \(\mathrm{rng}(\phi) \subset \mathbb{Z} \times \mathbb{N}\)

\sphinxAtStartPar
Hence \(\mathbb{Q}\) is also countable

\begin{sphinxadmonition}{note}{Example}

\sphinxAtStartPar
An example of an \sphinxstyleemphasis{\sphinxstylestrong{uncountable}} set is all binary sequences
\$\(
\{0,1\}^{\mathbb{N}} := \big\{ (b_1,b_2,\ldots) :  \, b_n \in \{0,1\ \} \text{
for each } n \big\}
\)\$
\end{sphinxadmonition}

\sphinxAtStartPar
\sphinxstylestrong{Sketch of proof:} If this set were countable then it could be listed as follows:
\begin{equation*}
\begin{split}
\begin{matrix}
1      & \leftrightarrow & {\bf a_1}, \; a_2, \; a_3, \; a_4, \ldots \\
2      & \leftrightarrow & b_1, \;{\bf b_2}, \; b_3, \; b_4, \ldots \\
3      & \leftrightarrow & c_1, \; c_2, \;{\bf c_3}, \; c_4, \ldots \\
4      & \leftrightarrow & d_1, \; d_2, \; d_3, \;{\bf d_4}, \ldots \\
\vdots &                 & \vdots
\end{matrix}
\end{split}
\end{equation*}
\sphinxAtStartPar
Such a list is never complete: Cantor’s diagonalization argument

\sphinxAtStartPar
Cardinality of  \(\{0,1\}^{\mathbb{N}}\) called the \sphinxstyleemphasis{\sphinxstylestrong{power of the continuum}}

\sphinxAtStartPar
Other sets with the power of the continuum
\begin{itemize}
\item {} 
\sphinxAtStartPar
\(\mathbb{R}\)

\item {} 
\sphinxAtStartPar
\((a, b)\) for any \(a < b\)

\item {} 
\sphinxAtStartPar
\([a, b]\) for any \(a < b\)

\item {} 
\sphinxAtStartPar
\(\mathbb{R}^N\) for any finite \(N \in \mathbb{N}\)

\end{itemize}

\begin{sphinxadmonition}{note}{Continuum hypothesis}

\sphinxAtStartPar
Every nonempty subset of \(\mathbb{R}\) is either
countable or has the power of the continuum
\end{sphinxadmonition}
\begin{itemize}
\item {} 
\sphinxAtStartPar
\sphinxhref{https://en.wikipedia.org/wiki/Continuum\_hypothesis}{Not a homework exercise}!

\end{itemize}

\begin{sphinxadmonition}{note}{Example}

\sphinxAtStartPar
\(\mathbb{R}\) and \((-1, 1)\) have the same cardinality because \(x \mapsto 2\arctan(x)/\pi\) is a bijection from \(\mathbb{R}\) to \((-1, 1)\)
\end{sphinxadmonition}

\begin{figure}[htbp]
\centering
\capstart

\noindent\sphinxincludegraphics{{arctan}.png}
\caption{Same cardinality}\label{\detokenize{03.set_theory:arctan}}\end{figure}

\sphinxstepscope


\chapter{Exercise set B}
\label{\detokenize{03.exercises:exercise-set-b}}\label{\detokenize{03.exercises::doc}}
\sphinxAtStartPar
Please, see the
{\hyperref[\detokenize{02.exercises::doc}]{\sphinxcrossref{\DUrole{doc,std,std-doc}{\sphinxstylestrong{general comment on the tutorial exercises}}}}}


\section{Question B.1}
\label{\detokenize{03.exercises:question-b-1}}
\sphinxAtStartPar
Each of the definitions below is an attempt to define a set. Determine whether a set is indeed defined in each case, and if not explain why.
\begin{enumerate}
\sphinxsetlistlabels{\arabic}{enumi}{enumii}{}{.}%
\item {} 
\sphinxAtStartPar
\(\{1,e,-2,-\pi\}\)

\item {} 
\sphinxAtStartPar
\(\{15,\{1,2,3\},\text{ANU},\text{Europe},\text{USA}\}\)

\item {} 
\sphinxAtStartPar
\(\{1,2,\dots,99 \}\)

\item {} 
\sphinxAtStartPar
\(\{1,4,7,91, \dots \}\)

\item {} 
\sphinxAtStartPar
\(\{x \in \mathbb{R} \colon x^2 \le 5\}\)

\item {} 
\sphinxAtStartPar
\(\{(x,y) \in \mathbb{R}^2 \colon 5x^2 + y^2 \le 10\}\)

\item {} 
\sphinxAtStartPar
\(\{f \colon [0,1] \rightarrow \mathbb{R} \colon f \text{ is one-to-one} \}\)

\item {} 
\sphinxAtStartPar
\(\{ f_n(x) \colon [0,1] \rightarrow \mathbb{R} \colon f_n(x) = x^n \}\)

\item {} 
\sphinxAtStartPar
\(\{A \subset S : x_0 \in A \}\) for given \(S\) and \(x_0 \in S\)

\end{enumerate}


\section{Question B.2}
\label{\detokenize{03.exercises:question-b-2}}
\sphinxAtStartPar
Let \(A\),\(B\) and \(C\) be any three sets.
Show that \(A \cap (B \cup C) = (A\cap B) \cup (A \cap C)\).

\begin{sphinxadmonition}{hint}{Hint:}
\sphinxAtStartPar
Hint, if you need it: One way to show that \(E=F\) is show that a arbitrary element of \(E\) must also be in \(F\) and vice versa.
\end{sphinxadmonition}


\section{Question B.3}
\label{\detokenize{03.exercises:question-b-3}}
\sphinxAtStartPar
Let \(A\), \(B\), \(C\) and \(D\) be some set such that \(A \subset C\) and \(B \subset C\).
Let \(f\colon D \rightarrow C\) be a function.

\sphinxAtStartPar
Show that \(f^{-1}(A \setminus B) = f^{-1}(A) \setminus f^{-1}(B)\).


\section{Question B.4}
\label{\detokenize{03.exercises:question-b-4}}
\sphinxAtStartPar
Find the composition \(g \circ f\) of two functions \(f\) and \(g\), if it exists:
\begin{enumerate}
\sphinxsetlistlabels{\arabic}{enumi}{enumii}{}{.}%
\item {} 
\sphinxAtStartPar
\(f \colon \mathbb{R} \rightarrow \mathbb{R}\) defined by \(f(x)=\sin(x)\) and \(g \colon \mathbb{R} \rightarrow \mathbb{R}\) defined by \(g(x)= \frac{x}{1+x^2}\)

\item {} 
\sphinxAtStartPar
\(f \colon \mathbb{R} \rightarrow \mathbb{R}\) defined by \(f(x)= 1-x^4\) and \(g \colon (1,\infty) \rightarrow \mathbb{R}\) defined by \(g(x)= \log(x-1)\)

\item {} 
\sphinxAtStartPar
\(f \colon \mathbb{R} \rightarrow \mathbb{R}\) defined by \(f(x)=\cos(x)\) and \(g \colon \mathbb{R}\setminus\{1\} \rightarrow \mathbb{R}\) defined by \(g(x)= \frac{x}{1-x}\)

\end{enumerate}

\begin{sphinxadmonition}{hint}{Hint:}
\sphinxAtStartPar
Is there a composition in each case?
\end{sphinxadmonition}


\section{Question B.5}
\label{\detokenize{03.exercises:question-b-5}}
\sphinxAtStartPar
Let \(f\) and \(g\) be any two functions from \(\mathbb{R}\) to \(\mathbb{R}\).  Is it true that
\(g \circ f = f \circ g\) always holds?

\begin{sphinxadmonition}{hint}{Hint:}
\sphinxAtStartPar
There are two
things implicit in this question.  First, there is an implicit final
sentence here, which is: If yes, prove it.  If no, give a
counterexample.  Second, an equality sign between two functions means
that they are the same function.  Hence to show equality you need to
show that they agree everywhere on the domain.  To show inequality,
you need to give just one point in the domain where the function
values differ.
\end{sphinxadmonition}


\section{Question B.6}
\label{\detokenize{03.exercises:question-b-6}}
\begin{sphinxadmonition}{note}{Fact: the sufficient conditions for concavity/convexity in 2D}

\sphinxAtStartPar
Let \(z = f(x,y)\) be a twice continuously differentiable function defined for all
\((x, y) \in R^2\).

\sphinxAtStartPar
Then it holds:
\begin{itemize}
\item {} 
\sphinxAtStartPar
\(f \text{ is convex } \iff f''_{1,1} \ge 0, \; f''_{2,2} \ge 0 , \text{ and } f''_{1,1} f''_{2,2} − (f′′_{1,2})^2 \ge 0\)

\item {} 
\sphinxAtStartPar
\(f \text{ is concave } \iff f''_{1,1} \le 0, \; f''_{2,2} \le 0 , \text{ and } f''_{1,1} f''_{2,2} − (f′′_{1,2})^2 \ge 0\)

\item {} 
\sphinxAtStartPar
\(f''_{1,1} > 0 \text{ and } f''_{1,1} f''_{2,2} −implies f \text{ is strictly convex}\)

\item {} 
\sphinxAtStartPar
\(f''_{1,1} < 0 \text{ and } f''_{1,1} f''_{2,2} −implies f \text{ is strictly concave}\)

\end{itemize}
\end{sphinxadmonition}
\begin{enumerate}
\sphinxsetlistlabels{\arabic}{enumi}{enumii}{}{.}%
\item {} 
\sphinxAtStartPar
Find the largest domain \(S\) on which
\(f(x, y) = x^2 − y^2 − xy − x^3\) is concave.

\item {} 
\sphinxAtStartPar
How about strictly concave?

\end{enumerate}

\sphinxstepscope


\chapter{Elements of linear algebra}
\label{\detokenize{04.linear_algebra:elements-of-linear-algebra}}\label{\detokenize{04.linear_algebra::doc}}
\noindent{\hspace*{\fill}\sphinxincludegraphics[scale=1.0]{{coming_soon}.png}\hspace*{\fill}}

\sphinxstepscope


\chapter{Elements of probability}
\label{\detokenize{05.probability:elements-of-probability}}\label{\detokenize{05.probability::doc}}
\noindent{\hspace*{\fill}\sphinxincludegraphics[scale=1.0]{{coming_soon}.png}\hspace*{\fill}}

\sphinxstepscope


\chapter{Fundamentals of optimization}
\label{\detokenize{06.optimization_fundamentals:fundamentals-of-optimization}}\label{\detokenize{06.optimization_fundamentals::doc}}
\noindent{\hspace*{\fill}\sphinxincludegraphics[scale=1.0]{{coming_soon}.png}\hspace*{\fill}}

\sphinxstepscope


\chapter{Unconstrained optimization}
\label{\detokenize{07.unconstrained:unconstrained-optimization}}\label{\detokenize{07.unconstrained::doc}}
\noindent{\hspace*{\fill}\sphinxincludegraphics[scale=1.0]{{coming_soon}.png}\hspace*{\fill}}

\sphinxstepscope


\chapter{Constrained optimization}
\label{\detokenize{08.constrained:constrained-optimization}}\label{\detokenize{08.constrained::doc}}
\noindent{\hspace*{\fill}\sphinxincludegraphics[scale=1.0]{{coming_soon}.png}\hspace*{\fill}}

\sphinxstepscope


\chapter{Practical session}
\label{\detokenize{09.practical_session:practical-session}}\label{\detokenize{09.practical_session::doc}}
\noindent{\hspace*{\fill}\sphinxincludegraphics[scale=1.0]{{coming_soon}.png}\hspace*{\fill}}

\sphinxstepscope


\chapter{Envelope and maximum theorems}
\label{\detokenize{10.envelope_maximum:envelope-and-maximum-theorems}}\label{\detokenize{10.envelope_maximum::doc}}
\noindent{\hspace*{\fill}\sphinxincludegraphics[scale=1.0]{{coming_soon}.png}\hspace*{\fill}}

\sphinxstepscope


\chapter{Dynamic optimization}
\label{\detokenize{11.dynamic:dynamic-optimization}}\label{\detokenize{11.dynamic::doc}}
\noindent{\hspace*{\fill}\sphinxincludegraphics[scale=1.0]{{coming_soon}.png}\hspace*{\fill}}

\sphinxstepscope


\chapter{Revision}
\label{\detokenize{12.revision:revision}}\label{\detokenize{12.revision::doc}}
\noindent{\hspace*{\fill}\sphinxincludegraphics[scale=1.0]{{coming_soon}.png}\hspace*{\fill}}







\renewcommand{\indexname}{Index}
\printindex
\end{document}